\documentclass[12pt,letterpaper]{article}
\usepackage{amsmath,amsthm,amsfonts,amssymb,amscd}
\usepackage{fullpage}
\usepackage{fancyhdr}
\usepackage{geometry}
\geometry{left=2.5cm,right=2.5cm,top=0.5cm,bottom=1cm}
\pagestyle{fancyplain}
\headheight 15pt
%\lhead{Xiaojun Sun\ (u0741290)}
%\chead{\textbf{\Large Homework \asgnname}}
%\rhead{\today}
%\headsep 10pt

\begin{document}

\title{Proposal: Arithmetic circuits verification using Network Flow model}
\author{Xiaojun Sun}
\maketitle

\section{Theory}
This paper discussed a new method modeling each gate and fanout branch by arithmetic signature expressions.
First, the compatibility for each cut set is computed using weight propagation. Second, based on a valid 
assignment on input signal that makes the circuit compatible, total flow conducting into fanout blocks is
equal to the flow consuming by floating signals, i.e.\\
\textbf{Theorem 1}\ \ \ The circuit is functionally correct if and only if:
\\
(i) There exists a compatible assignment of weights consistent with the input signature $Sig_{in}$;
\\(ii) The amount of the flow introduced by fanouts $\Delta_{fn}$ is equal to the flow consumed by floating signals $\Sigma_{fi}$.

Both the consistency checking and fanouts flow conservation checking are equivalence checking, which is 
accomplished by TED method in original paper. My work is to find a substitution method such as word-level
compatible computer algebraic method to verify the equivalence, thus check the functional correctness of
the arithmetic circuits.

\section{Implementation}
The test was running on as large as $62\times 62$ multiplier. The netlist is transformed to network of
half-adders, full-adders and other logic gates remained, which is easy to deduce signature expressions
to primary inputs signals.

Several cuts are selected, and for each cut the signature (flow) equation is checked equality with its higher 
level. If any difference detected, the residual expression must contain affected signals by the potential bug(s).

After the compatibility/consistency checking is completed, the flow generated by fanouts and consumed by floating
signals are compared to check equivalence. Then the functional correctness is asserted based on the conservative
condition of network flow model.


\begin{thebibliography}{1}
\bibitem{ref1}
Ciesielski, Maciej, Walter Brown, and André Rossi. \emph{"Arithmetic Bit-Level Verification Using Network Flow Model."} Hardware and Software: Verification and Testing. Springer International Publishing, 2013. 327-343.
\end{thebibliography}

\end{document}