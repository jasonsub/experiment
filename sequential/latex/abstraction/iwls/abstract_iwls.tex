\begin{abstract}
A combinational circuit with $k$-inputs and $k$-outputs implements 
Boolean functions $f: \B^k \rightarrow \B^k$, where $\B = \{0, 1\}$.
The function can also be construed as a mapping $f: \Fkk
\rightarrow \Fkk$,  where $\Fkk$ denotes a Galois field of $2^k$
elements. Every function over $\Fkk$ is a polynomial function ---
i.e. there exists a unique, minimal, canonical polynomial
$\F$ that describes $f$. This paper presents novel techniques based on 
computer-algebra and algebraic-geometry to derive the canonical
(word-level) polynomial representation of the circuit as $Y = \F (A)$
over $\Fkk$, where $A$ and $Y$ denote, respectively, the input and 
output bit-vectors of the circuit. We show that this can be achieved
by computing a Gr\"obner basis of a set of polynomials derived from
the circuit, using a specific elimination term order. Our approach can
be generalized to circuits with arbitrary number of inputs and
outputs, as polyfunctions $f: \Fkk^d \rightarrow \Fkk$ or $f:
{\mathbb{F}}_{2^n} \rightarrow {\mathbb{F}}_{2^m}$. 

Computing Gr\"obner bases using elimination orders is, however,
practically infeasible for large circuits. To overcome this
limitation, we present improvements to the computation
based on: i) term orders derived from the topological
traversal of the given circuit; and ii) the use of FGLM algorithm from 
algebraic geometry. We apply our approach to ``reverse-engineer''
hardware implementations of Elliptic Curve Cryptography (ECC)
primitives over Galois fields $\Fkk$ --- with applications to design
verification and function identification. 
\end{abstract}
