\section{Introduction}

Abstraction of word-level functionality from bit-level descriptions of
digital circuits is an important fundamental problem in Electronic
Design Automation (EDA) and design verification. Word-level abstractions
find applications in high-level/RTL datapath synthesis
%\cite{demicheli:iccad_98} \cite{demicheli:dac_99}
\cite{demicheli:tcad_03}, RTL verification \cite{WLS} %\cite{arditi:bmd}
\cite{lpsat}, word-level SMT and constraint solving \cite{ms:research}
\cite{boolector}, %\cite{tew:iccad08}, 
etc. Functional abstraction can
also be helpful in detecting malicious modifications to a circuit --
potentially inserted as hardware Trojan horses -- by
{\it reverse-engineering} the function implemented by the circuit\footnote{A
hardware Trojan modifies the functionality of the 
circuit. Abstraction of the function at word-level can perhaps give
some insights into the functional {\it intent} of the Trojan. This can
also be helpful in Trojan classification; such efforts are
currently underway within the EDA community \cite{Bhunia:Trojan}.}. 
It is further desirable that the obtained word-level abstraction be a
{\it canonical} representation of the function, to facilitate
equivalence verification between a specification model and an
implementation circuit. 

This paper presents a symbolic method to derive the {\it word-level,
  canonical, polynomial representation} from a given combinational
circuit. Our approach models the problem over Galois fields, and employs
concepts from commutative algebra and algebraic geometry ---
notably, Gr\"obner bases \cite{ideals:book} theory and
technology --- to derive the word-level abstraction.
%Buchberger's algorithm
%\cite{buchberger_thesis}, elimination ideals 
%and the FGLM algorithm \cite{fglm}. 

The main motivation for this work is to ``reverse engineer'' (or
to identify) the function implemented by a given Galois field
arithmetic circuit as used in Elliptic Curve Cryptography (ECC). The
main operations of encryption, decryption and authentication in ECC
rely on point-addition and point-doubling operations on elliptic
curves defined over Galois fields $\Fkk$. These primitive operations
are, in turn, implemented as circuits that perform polynomial
computations over $k$-bit vectors \cite{eccld}. 
%Moreover, the datapath size $k$ in these applications
%can be very large, {\it e.g.} $k = 163, 233$ and larger, as specified
%by the US National Institute for Standards and Technology (NIST) for
%ECC. 
The objective is to apply our approach to a given circuit and
{\it extract  the word-level polynomial function (polyfunction)}
implemented by it for design verification. While our approach can be
applied to any arbitrary combinational circuit, it is most beneficial
for arithmetic circuits where canonical Boolean representations prove
to be infeasible. 


{\bf Problem Statement:} Given a combinational circuit $C$ with
$k$-bit inputs and $k$-bit outputs, such that the circuit implements 
Boolean functions that are mappings between $k$-dimensional Boolean
spaces: $f: \B^k \rightarrow \B^k$. Let $\{a_0, \dots, a_{k-1}\}$
denote the primary inputs of the circuit, and let $\{y_0, \dots,
y_{k-1}\}$ denote the bit-level primary outputs. Suppose that we do
not know the function implemented by the circuit. We wish to
derive a {\it word-level, symbolic representation} of the function as
$Y = \F(A)$ over $\Fkk$, where $Y = \{y_0, \dots, y_{k-1}\}$, $A =
\{a_0, \dots,  a_{k-1}\}$ are, respectively, the output and input
bit-vectors (words), and $\F$ denotes a polynomial representation of
the function $f$. We wish to further generalize our approach to any
circuit with arbitrary number  of inputs ($n$) and outputs ($m$), as
polyfunctions $f: {\mathbb{F}}_{2^n} \rightarrow
{\mathbb{F}}_{2^m}$. 

{\bf Approach:} Canonical symbolic DAG 
representations of Boolean functions, such
as ROBDDs \cite{BRYA86}, FDDs \cite{okfdd}, BMDs \cite{bmd}, etc., are
based on (variants of) the Shannon's  expansion, which is a bit-wise,
Boolean decomposition. These are inapplicable to 
word-level abstraction of modulo-arithmetic circuits over Galois
extension fields $\Fkk$. Therefore, we approach this problem from the
perspective of {\it symbolic computer algebra and algebraic geometry}. 

The function $f: \B^k \rightarrow \B^k$ is a
mapping among $2^k$ elements. Therefore, $f$ can also be interpreted,
algebraically, in the following two ways: i) as a function $f:
\Z_{2^k} \rightarrow \Z_{2^k}$, i.e. as a function over finite integer
rings $\pmod{ 2^k}$; and ii) as a function $f: \Fkk \rightarrow \Fkk$,
where $\Fkk$ represents the Galois field of $2^k$
elements. In this work, we will analyze $f$ as a polynomial function
over $\Fkk$, for the following reasons. 

First of all, {\it not every
  function} of the type $f: \Z_{2^k} \rightarrow \Z_{2^k}$, is a
polynomial function; i.e. every function cannot be represented by a
polynomial $\F \pmod{ 2^k}$.  In commutative algebra, identification of
such polynomial functions is an important problem: given a function
$f: \Z_n \rightarrow \Z_n$, $n \in {\mathbb{N}}$, identify if $f$ has
a polynomial representation. If so, derive a unique,
minimal, canonical polynomial $\F$, such that $f \equiv  \F \pmod{
  n}$. 
%that represents $f$. 
%This
%requires to analyze $f$ at each of the $n$ points, and to setup a
%system of $n$ linear congruences to solve. If this system of
%congruences has a solution, then $f$ is a polynomial function; 
%otherwise, $f$ is not a polynomial function. And the feasible
%solutions to these linear congruences correspond to the coefficients
%of a polynomial $\F$ that represents $f$.  
The work of 
\cite{singmaster} 
%\cite{chen_95} \cite{chen_96} 
addresses such problems in number theory and algebra. {\it
 Shekhar et al.} \cite{shekhar:tcad07} address RTL verification
of polynomial datapaths over $k$-bit vectors using the above
concepts. However, in their work, the RTL datapath is already known to
be a polynomial function $\pmod{ 2^k}$. Unfortunately, an arbitrary
$k$-input/$k$-output combinational circuit cannot always be modeled as
a polynomial function over $\Z_{2^k}$; therefore, the $\Z_{2^k}$ model
is not considered in this work. 

On the other hand, there is a well-known ``textbook'' result
\cite{ff:1997} which states that over Galois fields ($\Fq$) of $q$
elements, {\it every function} $f: \Fq \rightarrow \Fq$ is a
polynomial function. 
%By analyzing $f$ over each of the $q$ points, one can apply
%Langrange's interpolation formula and interpolate a polynomial $\F(x)
%= \sum_{k=1} ^q  \frac{ \prod_{i \neq k}  (x -x_i)}{\prod_{i \neq k}
%  (x_k -x_i)} \cdot f(x_k)$, which is a polynomial of degree at 
%most $q-1$ in $x$. One can easily see that $\F(a)=f(a)$ for all $a \in
%\Fq$, and $\F(x)$ is therefore the polynomial function representation
%of $f$. 
Motivated by this fundamental result, we devise an approach
to derive a word-level polynomial function abstraction from the
circuit using the $f: \Fkk \rightarrow \Fkk$ polyfunction model. 
%Given the circuit $C$ which represents a function $f:
%\B^k \rightarrow \B^k$, we interpret $f$ as a function $f: \Fkk
%\rightarrow \Fkk$, and derive a unique, minimal, canonical polynomial
%representation as $Y = \F(A)$ over $\Fkk$, where $Y, A$ are,
%respectively, the output and input bit-vectors of the circuit
%$C$. 
%Note, however, that we are not given a function (mapping);
%rather, we are given a circuit, which is an encoding of the
%function. Interpolating a polynomial $\F$ over $2^k$ points is
%also going to be infeasible for large values of $k$. Therefore, 
We analyze the circuit, derive a specific set of polynomials modeled
over $\Fkk$, and compute a {\it Gr\"obner basis} of these polynomials
with a specific term order to abstract the canonical polynomial
representation. Our approach can be generalized to a circuit with
arbitrary number of inputs and outputs, i.e. to model polynomial
functions $f: \Fkk^d \rightarrow \Fkk$, and further to $f:
{\mathbb{F}}_{2^n} \rightarrow {\mathbb{F}}_{2^m}$.  


%\subsection{Contributions of this Thesis}
{\bf Approach \& Contributions:} 

\begin{enumerate}
\item Using polynomial abstractions, we analyze the circuit, and
  model the gate-level Boolean operators as elements of a multivariate
  polynomial ring with coefficients in $\Fkk$.

\item Based on the concepts of {\it Strong Nullstellensatz, Gr\"obner
  bases, elimination ideals and projections of varieties}
  \cite{ideals:book}, we deduce that the polynomial abstraction
  problem can be formulated as one of {\it computing a Gr\"obner
    basis} of a set of polynomials derived from the given circuit
  netlist, using a specific elimination term order $>$. 

\item Computing Gr\"obner bases using elimination term orders is
  infeasible for large circuits. To overcome this limitation, we 
  investigate the use of the FGLM algorithm \cite{fglm} to derive the
  polynomial representation. The FGLM algorithm takes an {\it already
    computed} Gr\"obner basis ($G_1$) w.r.t. an arbitrary term order
  $>_1$, and transforms it into another Gr\"obner basis ($G$)
  w.r.t. an elimination term order $>$. In the work of {\it
  Lv} \cite{lv:date2012} it was shown that for any
  given circuit, there exists a term order $>_1$ that renders the set
  of polynomials derived from the circuit itself a Gr\"obner
  basis. Moreover, $>_1$ can be easily derived by performing a reverse
  topological traversal of the circuit. Consequently, $G_1$ is readily
  available as a Gr\"obner basis directly by construction. Using FGLM,
  we can then translate $G_1$ into the Gr\"obner basis $G$
  w.r.t. the required elimination order $>$. 

\item The worst-case complexity of computing a Gr\"obner basis is
  known to be doubly exponential in $n$, the number of variables, and
  polynomial in $d$, the degree of the ideal. Most hardware/software
  verification applications do exhibit such high complexity, as the
  number of intermediate computations in the Gr\"obner basis algorithm
  (Buchbeger's algorithm) \cite{buchberger_thesis} 
  tend to explode. Therefore, we conjecture that using the FGLM
  algorithm --- which relies on {\it sparse linear algebra concepts}
  --- perhaps this complexity can be avoided. We are currently
  implementing a domain specific FGLM algorithm --- specifically for
  the abstraction problems related to Galois field
  crypto-circuits. Experimental results with FGLM are not yet available.

%\item As an application, we will use our CAD approach to
%  reverse-engineer the function implemented by a given Galois field
%  arithmetic circuits used in ECC   implementations. This can be used
%  for function identification from a given circuit for formal
%  verification. 

\end{enumerate}

{\bf Paper Organization:} The next section reviews previous work in
the area of canonical representations functions, word-level
abstractions, 
%formal design verification using computer algebra techniques, 
and also polynomial interpolation literature in symbolic
computing. Section \ref{sec:prelimgf} reviews 
Galois field theory and describes the arithmetic circuits that we wish
to verify using our approach. Section \ref{sec:ideals}
reviews preliminary computer-algebra concepts related to ideals,
varieties, Gr\"obner bases, elimination ideals and
Nullstellensatz. Section \ref{sec:theory} describes our results on
polynomial abstraction from circuits, and demonstrates the application
of our work using preliminary experiments. The application of the FGLM
algorithm is covered in Section \ref{sec:fglm}. Finally, Section
\ref{sec:concl} concludes the proposal. 
