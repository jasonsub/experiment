\begin{abstract}
A combinational circuit with $k$-inputs and $k$-outputs implements
Boolean functions $f: \B^k \rightarrow \B^k$, where $\B = \{0, 1\}$.
Such functions can also be construed as a mapping $f: \Fkk
\rightarrow \Fkk$,  where $\Fkk$ denotes the Galois field of $2^k$
elements. Every function over $\Fkk$ is a polynomial function ---
i.e. there exists a unique, minimal, canonical polynomial
$\F$ that describes $f$. In this master's thesis, we propose to
investigate computer algebra based methods to derive the canonical
(word-level) polynomial representation of the circuit as $Y = \F (A)$
over $\Fkk$, where $A$ and $Y$ denote, respectively, the input and 
output bit-vectors of the circuit. We show that this can be achieved
by computing a Gr\"obner basis of a set of polynomials derived from
the circuit, using a specific elimination term order. 
%This
%approach can be generalized to any circuit with arbitray number of
%inputs and outputs to abstract polynomial functions over $f:
%{\mathbb{F}}^{2^n} \rightarrow {\mathbb{F}}^{2^m}$.
Computing Gr\"obner bases using elimination orders is, however,
practically infeasible for large circuits. To overcome this
limitation, we propose to investigate the use of the FGLM algorithm 
--- which converts a given Gr\"obner basis w.r.t. one term order $>_1$
into another w.r.t. an elimination term order $>_2$ ---  to  
derive the required polynomial abstraction. The objective of this
research is to: i) design an efficient CAD tool based on our approach;
and ii) apply it to reverse-engineer hardware implementations of
Elliptic Curve Cryptography (ECC) primitives over Galois fields $\Fkk$
--- with applications to design verification and function
identification. 
\end{abstract}
