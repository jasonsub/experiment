% THIS IS SIGPROC-SP.TEX - VERSION 3.1
% WORKS WITH V3.2SP OF ACM_PROC_ARTICLE-SP.CLS
% APRIL 2009
%
% It is an example file showing how to use the 'acm_proc_article-sp.cls' V3.2SP
% LaTeX2e document class file for Conference Proceedings submissions.
% ----------------------------------------------------------------------------------------------------------------
% This .tex file (and associated .cls V3.2SP) *DOES NOT* produce:
%       1) The Permission Statement
%       2) The Conference (location) Info information
%       3) The Copyright Line with ACM data
%       4) Page numbering
% ---------------------------------------------------------------------------------------------------------------
% It is an example which *does* use the .bib file (from which the .bbl file
% is produced).
% REMEMBER HOWEVER: After having produced the .bbl file,
% and prior to final submission,
% you need to 'insert'  your .bbl file into your source .tex file so as to provide
% ONE 'self-contained' source file.
%
% Questions regarding SIGS should be sent to
% Adrienne Griscti ---> griscti@acm.org
%
% Questions/suggestions regarding the guidelines, .tex and .cls files, etc. to
% Gerald Murray ---> murray@hq.acm.org
%
% For tracking purposes - this is V3.1SP - APRIL 2009

\documentclass{acm_proc_article-sp}

\begin{document}

\title{Verification of Sequential Galois Field Circuits using Algebraic Geometry
\titlenote{First version, only contains abstract and introduction part.}}
%\subtitle{[Extended Abstract]
%\titlenote{A full version of this paper is available as
%\textit{Author's Guide to Preparing ACM SIG Proceedings Using
%\LaTeX$2_\epsilon$\ and BibTeX} at
%\texttt{www.acm.org/eaddress.htm}}}
%
% You need the command \numberofauthors to handle the 'placement
% and alignment' of the authors beneath the title.
%
% For aesthetic reasons, we recommend 'three authors at a time'
% i.e. three 'name/affiliation blocks' be placed beneath the title.
%
% NOTE: You are NOT restricted in how many 'rows' of
% "name/affiliations" may appear. We just ask that you restrict
% the number of 'columns' to three.
%
% Because of the available 'opening page real-estate'
% we ask you to refrain from putting more than six authors
% (two rows with three columns) beneath the article title.
% More than six makes the first-page appear very cluttered indeed.
%
% Use the \alignauthor commands to handle the names
% and affiliations for an 'aesthetic maximum' of six authors.
% Add names, affiliations, addresses for
% the seventh etc. author(s) as the argument for the
% \additionalauthors command.
% These 'additional authors' will be output/set for you
% without further effort on your part as the last section in
% the body of your article BEFORE References or any Appendices.

\numberofauthors{2} %  in this sample file, there are a *total*
% of EIGHT authors. SIX appear on the 'first-page' (for formatting
% reasons) and the remaining two appear in the \additionalauthors section.
%
\author{
% You can go ahead and credit any number of authors here,
% e.g. one 'row of three' or two rows (consisting of one row of three
% and a second row of one, two or three).
%
% The command \alignauthor (no curly braces needed) should
% precede each author name, affiliation/snail-mail address and
% e-mail address. Additionally, tag each line of
% affiliation/address with \affaddr, and tag the
% e-mail address with \email.
%
% 1st. author
\alignauthor
Xiaojun Sun\\
       \affaddr{University of Utah}\\
       \affaddr{Department of Electrical \& Computer Engineering}\\
       \affaddr{Salt Lake City, USA}\\
       \email{xiaojun.sun@utah.edu}
% 2nd. author
\alignauthor
Priyank Kalla\\
       \affaddr{University of Utah}\\
       \affaddr{Department of Electrical \& Computer Engineering}\\
       \affaddr{Salt Lake City, USA}\\
       \email{kalla@ece.utah.edu}
% 3rd. author
%\alignauthor Lars Th{\o}rv{\"a}ld\titlenote{This author is the
%one who did all the really hard work.}\\
%       \affaddr{The Th{\o}rv{\"a}ld Group}\\
%       \affaddr{1 Th{\o}rv{\"a}ld Circle}\\
%       \affaddr{Hekla, Iceland}\\
%       \email{larst@affiliation.org}
%\and  % use '\and' if you need 'another row' of author names
% 4th. author
%\alignauthor Lawrence P. Leipuner\\
%       \affaddr{Brookhaven Laboratories}\\
%       \affaddr{Brookhaven National Lab}\\
%       \affaddr{P.O. Box 5000}\\
%       \email{lleipuner@researchlabs.org}
% 5th. author
%\alignauthor Sean Fogarty\\
%       \affaddr{NASA Ames Research Center}\\
%       \affaddr{Moffett Field}\\
%       \affaddr{California 94035}\\
%       \email{fogartys@amesres.org}
% 6th. author
%\alignauthor Charles Palmer\\
%       \affaddr{Palmer Research Laboratories}\\
%      \affaddr{8600 Datapoint Drive}\\
%%      \affaddr{San Antonio, Texas 78229}\\
%       \email{cpalmer@prl.com}
}
% There's nothing stopping you putting the seventh, eighth, etc.
% author on the opening page (as the 'third row') but we ask,
% for aesthetic reasons that you place these 'additional authors'
% in the \additional authors block, viz.
%\additionalauthors{Additional authors: John Smith (The Th{\o}rv{\"a}ld Group,
%email: {\texttt{jsmith@affiliation.org}}) and Julius P.~Kumquat
%(The Kumquat Consortium, email: {\texttt{jpkumquat@consortium.net}}).}
\date{07 Nov 2013}
% Just remember to make sure that the TOTAL number of authors
% is the number that will appear on the first page PLUS the
% number that will appear in the \additionalauthors section.

\maketitle
\begin{abstract}
Circuits working in Galois fields are increasingly employed in designs like Ellptic Curve Cryptography (ECC). 
This work proposes a new method to effectively verify sequential circuits in Galois fields. Algebraic geometry
is intriduced to describe circuits behavior and redefine implicit state space traversal. Moreover, Gr\"obner basis
representation is adopted for word-level abstraction on circuit variables to address state explosion problem
with BDDs. Experiments are run with F4-style reduction engine to get more competitive results.
\end{abstract}

% A category with the (minimum) three required fields
\category{EDA5.1}{Verification}{Functional, transaction-level, RTL, and gate-level modeling and verification of hardware design}
%A category including the fourth, optional field follows...
%\category{D.2.8}{Software Engineering}{Metrics}[complexity measures, performance measures]

\terms{Verification, Algorithms}

\keywords{Verification, Sequential, Galois Fields, Algebraic Geometry} % NOT required for Proceedings

\section{Introduction}
Verification for sequential circuits has been extensively discussed for decade. Binary Decision Diagram (BDD) is the first
and most widely used canonical technique among those for sequential verification \cite{touati1990implicit}. It is easy to
manipulate any Boolean expressions with BDD and use them to enumerate implicit states. However, an issue that BDDs have to
face is the size explosion when Boolean variables increase significantly. Several modifications have been made to improve
this problem by optimizing original BDD \cite{gunther2001application} \cite{narayan1997reachability}, or by reducing Boolean
variables from circuits\cite{burch1991representing} \cite{jiang2003verification}. Some variants of BDD representation
have also been developed such as Linear Taylor Expansion Diagram (LTED) \cite{alizadeh2009sequential}.

Another sort of approaches are based on satisfiability theory (SAT). SAT-solver is applied to state space traversal and 
work through eliminations on Boolean constrains \cite{coudert1990verification}. By exploiting SAT-based algorithm, direct
reachability induction is proposed \cite{bjesse2000sat}. Still SAT-based approaches need to struggle with size explosion problem.

\section{Related Works}
G. Avrunin \cite{avrunin1996symbolic} introduced algebraic geometry into formal verification. In his paper, the corresponding
relation between varieties of ideals and circuit variables is discussed, and concepts such as intersect of varieties, ideal and
its generators and algebraic closure are explored. Yet, a self-contained system of algebraic geometry representation theory has
not been well-defined.

T. Pruss, el(which one should I cite?) developed a word-level abstraction method based on  Gr\"obner basis theory...

J. Lv, el (should I cite this part) gave out a F4-style reduction algorithm which can help speed up Gr\"obner basis calculation...

% The following two commands are all you need in the
% initial runs of your .tex file to
% produce the bibliography for the citations in your paper.
\bibliographystyle{abbrv}
\bibliography{seq_verif}  % sigproc.bib is the name of the Bibliography in this case
% You must have a proper ".bib" file
%  and remember to run:
% latex bibtex latex latex
% to resolve all references
%
\balancecolumns
% That's all folks!
\end{document}