\section{Introduction}
Verification for sequential circuits has been extensively discussed for decade. Binary Decision Diagram (BDD) is the first
and most widely used canonical technique among those for sequential verification \cite{touati1990implicit}. It is easy to
manipulate any Boolean expressions with BDD and use them to enumerate implicit states. However, an issue that BDDs have to
face is the size explosion when Boolean variables increase significantly. Several modifications have been made to improve
this problem by optimizing original BDD \cite{gunther2001application} \cite{narayan1997reachability}, or by reducing Boolean
variables from circuits\cite{burch1991representing} \cite{jiang2003verification}. Some variants of BDD representation
have also been developed such as Linear Taylor Expansion Diagram (LTED) \cite{alizadeh2009sequential}.

Another sort of approaches are based on satisfiability theory (SAT). SAT-solver is applied to state space traversal and 
work through eliminations on Boolean constrains \cite{coudert1990verification}. By exploiting SAT-based algorithm, direct
reachability induction is proposed \cite{bjesse2000sat}. Still SAT-based approaches need to struggle with size explosion problem.

\section{Related Works}
G. Avrunin \cite{avrunin1996symbolic} introduced algebraic geometry into formal verification. In his paper, the corresponding
relation between varieties of ideals and circuit variables is discussed, and concepts such as intersect of varieties, ideal and
its generators and algebraic closure are explored. Yet, a self-contained system of algebraic geometry representation theory has
not been well-defined.

T. Pruss, el(which one should I cite?) developed a word-level abstraction method based on  Gr\"obner basis theory...

J. Lv, el (should I cite this part) gave out a F4-style reduction algorithm which can help speed up Gr\"obner basis calculation...

