\section{Experiment Results}
Our experiment on different size of SMPO designs is performed with Singular symbolic algebra computation system.
The SMPO designs are given as gate-level netlists with registers, then translated to polynomials to compose
elimination ideal in Singular for Gr\"obner basis calculation. The experiment is conducted on desktop with
3.5GHz Intel $\text{Core}^\text{TM}$ i7 Quad-core CPU, 16 GB RAM and running 64-bit Linux OS.

\subsection{Computing Gr\"obner basis in Singular}
The Singular tool can read in scripts written in its own format similar to ANSI-C. For SMPO experiment, the main 
loop of our script file performs the same function as algorithm \ref{alg:modified} describes, while Gr\"obner basis
computation in main loop can be divided into 4 different function parts:

(i) Pre-process:

This step is executed only once before main loop starts. The function of pre-process is to compute following system
of equations for bit-level inputs $a_0 \sim a_{k-1}$:
\begin{displaymath}
\left\{
  \begin{array}{ll}
  a_0 & = f_0(A)\\
  a_1 & = f_1(A)\\
  \vdots & \ \\
  a_{k-1} & = f_{k-1}(A)
  \end{array} \right.
\end{displaymath}
The methodology has been discussed in section \ref{sec:blvs}. For 5-bit SMPO example, we start from word-level
expression polynomial
\begin{displaymath}
A + a_0\alpha^5+a_1\alpha^{10}+a_2\alpha^{20}+a_3\alpha^9+a_4\alpha^{18}
\end{displaymath}
and the result is
\begin{displaymath}
\left\{
  \begin{array}{lcl}
  a_0 & = & (\alpha+1)A^{16}+(\alpha^4+\alpha^3+\alpha)A^8+(\alpha^3+\alpha^2)A^4\\&&+(\alpha^4+1)A^2+(\alpha^2+1)A\\
  a_1 & = & (\alpha^2+1)A^{16}+(\alpha+1)A^8+(\alpha^4+\alpha^3+\alpha)A^4\\&&+(\alpha^3+\alpha^2)A^2+(\alpha^4+1)A\\
  a_2 & = & (\alpha^4+1)A^{16}+(\alpha^2+1)A^8+(\alpha+1)A^4\\&&+(\alpha^4+\alpha^3+\alpha)A^2+(\alpha^3+\alpha^2)A\\
  a_3 & = & (\alpha^3+\alpha^2)A^{16}+(\alpha^4+1)A^8+(\alpha^2+1)A^4\\&&+(\alpha+1)A^2+(\alpha^4+\alpha^3+\alpha)A\\
  a_4 & = & (\alpha^4+\alpha^3+\alpha)A^{16}+(\alpha^3+\alpha^2)A^8+(\alpha^4+1)A^4\\&&+(\alpha^2+1)A^2+(\alpha+1)A
  \end{array} \right.
\end{displaymath}
By replacing bit-level variable $a_i$ with $b_i, r_i$ or $R_i$, and word-level variable $A$ with $B, r, R$ respectively,
we can directly get bit-word relation functions for another operand input, pseudo input and pseudo output.

One limitation to Singular tool is the exponential cannot exceed $2^{63}$, so when doing experiments for SMPO larger than
62 bits, we use a little trick (the feasibility of this trick can also be verified in following steps). Since the BLVS method
only requires squaring of equations each time, the exponential of word $A$ can only be in the form $2^{i-1}$, i.e. 
$A^{2^0},A^{2^1},\dots,A^{2^{k-1}}$. To minimize the exponential presenting in Singular tool, we rewrite $2^{i-1}$ to $i$, i.e.
$(A^{2^0},A^{2^1},\dots,A^{2^{k-1}}) \to (A, A^2, \dots, A^k)$. In this way result is rewritten to be
$$a_0 = (\alpha+1)A^5+(\alpha^4+\alpha^3+\alpha)A^4+(\alpha^3+\alpha^2)A^3+(\alpha^4+1)A^2+(\alpha^2+1)A$$
Thus the exponential will not exceed the Singular data size limit.

This step requires limited substitution operations in Singular, so although we use the naive Gaussian elimination
method (whose time complexity is $O(k^3)$), the time cost is trivial comparing to following steps. For 33 bits experiment,
pre-process execution time is 2.7 sec; while for 100 bits experiment time cost is 36 sec.

(ii) Spoly reduction:

First, Spoly is calculated based on RATO, then reduced with the ideal composed by circuit description polynomials ($J$).
For already finished experiments, naive reduction (multi-division) is adopted, and this step takes largest portion of total 
time consumption.

For SMPO experiments, reduced Spoly has following generic form (all coefficients are omitted):
$$redSpoly = \sum r_i + \sum a_ib_i + \sum a_iB + \sum b_iA + R + r$$
Note there is no cross-term for bit-level or word-level variables from same side such as $a_ia_j, a_iA$, etc. Consider the necessary
condition of our trick, this property of reduced Spoly guarantees the word level variable can only exist in the form $A^{2^{i-1}}$,
after substituting bit-level variables with corresponding word-level variable.

(iii) Substitute bit-level variables in reduced Spoly:

Use the result from pre-process, get rid of $r_i, a_i$ and $b_i$ through substitution. This step yields following polynomial (consider the trick we used):
$$R + \sum r^i + \sum A^iB^j$$
all coefficients omitted.

(iv) Substitute present state word-level variable $r$ with inputs $A$ and $B$:

According to section \ref{sec:SMPOexperiment}, there is still a polynomial $r_{in}$ in the ideal we want to compute Gr\"obner basis.
This polynomial has form $r+f'(A,B)$, which is last clock cycle's output ($R+f'(A,B)$) with only leading term replaced in step "$from^i\gets to^i$" in 
algorithm \ref{alg:modified}. Basically this step has nothing different from last one, however, it must be taken good care of when using our
trick. There is power of $r$, $r^m$ is originally $r^{2^{m-1}}$; so if $r+f'(A,B)$ contains terms $A^iB^j$, the correct result after doing power
is $$(A^{2^{i-1}}B^{2^{j-1}})^{2^{m-1}} = A^{2^{((i+m-2)\bmod k)+1}}B^{2^{((j+m-2)\bmod k)+1}}$$
So the correct exponential for $A$ and $B$ in $(A^iB^j)^m$ should be $((i+m-2)\bmod k)+1$ and $((j+m-2)\bmod k)+1$, respectively.

Within one main loop, after finishing steps (ii) to (iv), the output should be intermediate multiplication result $R+f(A,B)$. After $k$ loops,
the output is $R+A\cdot B$ when SMPO is bug-free.

\subsection{Discussion}

I have concerns on this SMPO experiment because we use abstraction polynomial $R+f(A,B)$ instead of eliminating to univariate polynomial $g(R)$.
The only point of inconsistency to the FSM traversal we proposed is, currently the ideal quotient theory's correctness is only guaranteed under
univariate assumption. I think we have some options here: 1, expand the ideal quotient theory to multivariate; 2, argue that we use abstraction
only to make it easier to check the correctness of our experiment and let the output make sense to people; 3, abandon SMPO experiment in this paper
focusing on FSM traversal, try to do some new experiments on reachability checking, leave this SMPO experiment for some new paper concerning
functional verification.
