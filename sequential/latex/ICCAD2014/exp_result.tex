\section{Experimental Results}
\label{sec:result}
Using our approach described above, we have performed experiments to verify SMPO circuits over $\Fkk$ with \textsc{Singular} symbolic algebra computation system [v. 3-1-6]\cite{Singular}. 
Bugs are also introduced into SMPO designs for runtime comparison. To show the advantages of our approach, experiments using traditional methods such as SAT, 
BDD are also conducted. Our experiments run on a desktop with 3.5GHz Intel $\text{Core}^\text{TM}$ i7 Quad-core CPU, 16 GB RAM and 64-bit Linux.

\subsection{Evaluation of SAT/BDD based methods}
For every SMPO design, there is an equation in following form (this is a 5-bit example):
\begin{align}
R_i &= b_ia_{i+1} + b_{i+1}(a_i + a_{i+3}) + b_{i+2}(a_{i+3} + a_{i+4}) \nonumber\\
&+ b_{i+3}(a_{i+1} + a_{i+2}) + b_{i+4}
(a_{i+2} + a_{i+4}),\ 0\leq i\leq 4 \nonumber
\end{align}
This equation can be used as specification of SMPO circuit. For SAT/BDD based methods,
we first unroll our SMPO design, then create a \emph{miter} with the unrolled design and
circuit specification to check if it is unsatisfiable.

The SAT solver we use is \emph{Lingeling}\cite{Lingeling}. Single-output miter circuit is written
in DIMACS CNF\cite{dimacs} format and fed to Lingeling. For BDD based approach, we use the BDD module integrated
with VIS tool. The miter is described by a BLIF file and read by VIS\cite{VIS}. After BDD is built, if there is
no path to leaf "1", then the miter is unsatisfiable.

The runtime results given in seconds include experiments for 11, 18, 23, 33 and 50 bits unrolled SMPO
and their specifications. They show that both SAT and BDD based approach 
cannot verify SMPO beyond 33 bits.

\begin{table}[h]
\centering
\begin{tabular}{|c||c|c|c|c|c|} 
\hline
& \multicolumn{5}{|c|}{Word size of the operands $k$-bits}  \\
\hline
Solver & 11 & 18 & 23 & 33 & 50\\
\hline
\hline
Lingeling & 0.5  & 12.9  & 157.4 & 5931.2 & \emph{TO}\\
\hline
\hline
BDD & 0.02 & 5.7 & 237.9 & ?? & \emph{TO} \\
\hline
\end{tabular}
\caption{Runtime for verification of bug-free SMPO circuits over $\Fkk$ for SAT and BDD based methods. \emph{TO} = timeout of 4 hrs}\label{table:satbdd}  
\end{table}

\subsection{Evaluation of Our Approach}
Our approach requires only one time reduction for each clock cycle using RATO. If we use built-in function in \textsc{Singular} to directly
compute Gr\"obner basis, the high-degree word definition polynomial will exceed \textsc{Singular}'s capacity, even for 11 bits SMPO.

We also introduce bugs to our SMPO designs; these bugs are simply swapping an arbitrary pair of gate output signals, they will make the final result be
a far more complicated polynomial instead of $R+A\cdot B$, so the runtime will be slightly longer than bug-free one.