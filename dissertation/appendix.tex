\numberofappendices=1
\appendix
\chapter{}
\fixchapterheading
\section{Normal Basis Theory}
\label{append:NB}
In Chapter \ref{ch:prelim_GF} we briefly introduced the concept of normal basis (NB), and 
the benefits of using NB. In the section we describe more details about the normal basis theory by 
characterizing NB from linear algebra perspective, constructing a general NB in an 
arbitrary field and converting between NB and StdB. All theorems and lemmas refer to the 
dissertation of S. Gao \cite{gao:phd_normal_basis}, and we deduce all the proofs for them.
\subsection{Characterization of Normal Basis}
In order to depict the characterization of NB, we need the introduce some concepts from linear algebra domain.
\begin{Definition}[Frobenius Map]
Define map $\sigma:x\to x^p,~x\in \F_{p^n}$. This map denoting the linear map of field extension $\F_{p^n}$
over $\F_p$.
\end{Definition}
Additionally we can define a subspace based on linear map $T$:
\begin{Definition}
A subspace $W\subset V$ is called {\bf $T$-invariant} when 
$$Tu \in W, ~~ \forall \text{vector }u\in W$$

Subspace $Z(u,T) = \langle u,Tu,T^2u,\dots\rangle$ is called {\bf $T$-cyclic subspace} of $V$.
If $Z(u,T) = V$ holds, then $u$ is called a {\bf cyclic vector} of $V$ for $T$.
\end{Definition}
Field elements $x$ constitutes polynomials in ring $\F[x]$. Subsequently we can define the 
nullspace of polynomial:
\begin{Definition}[Nullspace of polynomial]
\label{def:nullspace}
For any polynomial $g(x) \in \F[x]$, the null space of $g(T)$ consists of all vectors $u$ such that
$g(T)u = 0$.
\end{Definition}
Finally we get to the most important concept: $T$-Order. Meanwhile it derives the construction of a ring extended by a field requires
a minimal polynomial (while we use stronger irreducible polynomial $P(x)$ in Chapter \ref{ch:prelim_GF}).
\begin{Definition}
For any vector $u \in V$, the monic polynomial $g(x) \in \F[x]$ with smallest degree such that
$g(T)u = 0$ is called the {\bf $T$-Order} of $u$ or {\bf minimal polynomial} of $u$.

That is, for an arbitrary element $\theta$ in $F_{q^n}$, find least positive integer $k$ such that
$\sigma^k\theta = \sum_{i=0}^{k-1} c_i\sigma^i\theta$, then the $\sigma$-Order of $\theta$ can be 
written as $$Ord_{\theta,\sigma}(x) = x^k - \sum_{i=0}^{k-1} c_ix^i$$
\end{Definition}

Using the concepts introduced above along with other basic concepts in linear algebra and finite field theory,
we can derive following theorems.

\begin{Lemma}
Given $g(x) \in \F[x]$ and $W$ is its nullspace. Let $d(x) = gcd(f(x),g(x)), e(x) = f(x)/d(x)$. 
Then $dim(W) = deg(d(x))$ and $W = \{e(T)u~|~u \in V\}$.
\end{Lemma}
\begin{Proof}
Assume $f(x)$ is the minimal and characteristic polynomial for $T$. Then according to Definition \ref{def:nullspace}
we obtain 
$$W = \{u\in V~|~g(T)u = 0\}$$

Let $k$ be a polynomial whose degree is larger than $f(x)$, i.e. $deg(k(x))>deg(f(x))$, then
$$f(x)|k(x)~~\text{iff}~k(T)=0$$

Since the construction of $f(x)$ relies on $T$, $f(T) = 0$, we get
$$\forall u, f(T)u = 0$$

Consider $W(u,T)$ subsequently relies on $g(T)$, we deduce 
$$dim(W) = deg(d(x))$$

According to the definition of $W$, we have
\begin{align*}
\forall u,~~& e(T)u \in W \Longleftrightarrow \\
& g(T)e(T)u = g(T)\frac{f(T)}{d(T)} u = h(T)\cdot f(T) u = 0
\end{align*}
\end{Proof}
\subsection{Construction of General Normal Basis}
\subsection{Bases Conversion}
% General \lambda matrix computation

\section{Optimal Normal Basis}
\label{append:ONB}
\subsection{Construction of Optimal Normal Basis}
\subsection{Optimal Normal Basis Multiplier Design}
% Customized Galois Field IEEE1363-2000
% Multi table <=> \lambda matrix
