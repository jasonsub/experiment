\numberofappendices=1
\appendix
\chapter{}
\fixchapterheading
\section{Normal Basis Theory}
\label{append:NB}
In Chapter \ref{ch:prelim_GF} we briefly introduced the concept of normal basis (NB), and 
the benefits of using NB. In the section we describe more details about the normal basis theory by 
characterizing NB from linear algebra perspective, constructing a general NB in an 
arbitrary field and converting between NB and StdB. All theorems and lemmas refer to the 
dissertation of S. Gao \cite{gao:phd_normal_basis}, and we deduce all the proofs for them.
\subsection{Characterization of Normal Basis}
In order to depict the characterization of NB, we need the introduce some concepts from linear algebra domain.
\begin{Definition}[Frobenius Map]
Define map $\sigma:x\to x^p,~x\in \F_{p^n}$. This map denoting the linear map of field extension $\F_{p^n}$
over $\F_p$.
\end{Definition}
Additionally we can define a subspace based on linear map $T$:
\begin{Definition}
A subspace $W\subset V$ is called {\bf $T$-invariant} when 
$$Tu \in W, ~~ \forall \text{vector }u\in W$$

Subspace $Z(u,T) = \langle u,Tu,T^2u,\dots\rangle$ is called {\bf $T$-cyclic subspace} of $V$.
If $Z(u,T) = V$ holds, then $u$ is called a {\bf cyclic vector} of $V$ for $T$.
\end{Definition}
Field elements $x$ constitutes polynomials in ring $\F[x]$. Subsequently we can define the 
nullspace of polynomial:
\begin{Definition}[Nullspace of polynomial]
\label{def:nullspace}
For any polynomial $g(x) \in \F[x]$, the null space of $g(T)$ consists of all vectors $u$ such that
$g(T)u = 0$.
\end{Definition}
Finally we get to the most important concept: $T$-Order. Meanwhile it derives the construction of a ring extended by a field requires
a minimal polynomial (while we use stronger irreducible polynomial $P(x)$ in Chapter \ref{ch:prelim_GF}).
\begin{Definition}
For any vector $u \in V$, the monic polynomial $g(x) \in \F[x]$ with smallest degree such that
$g(T)u = 0$ is called the {\bf $T$-Order} of $u$ or {\bf minimal polynomial} of $u$.

That is, for an arbitrary element $\theta$ in $F_{q^n}$, find least positive integer $k$ such that
$\sigma^k\theta = \sum_{i=0}^{k-1} c_i\sigma^i\theta$, then the $\sigma$-Order of $\theta$ can be 
written as $$Ord_{\theta,\sigma}(x) = x^k - \sum_{i=0}^{k-1} c_ix^i$$
\end{Definition}

Using the concepts introduced above along with other basic concepts in linear algebra and finite field theory,
we can derive following theorems.

\begin{Lemma}
\label{lem:app1}
Given $g(x) \in \F[x]$ and $W$ is its nullspace. Let $d(x) = gcd(f(x),g(x)), e(x) = f(x)/d(x)$. 
Then $dim(W) = deg(d(x))$ and $W = \{e(T)u~|~u \in V\}$.
\end{Lemma}
\begin{Proof}
Assume $f(x)$ is the minimal and characteristic polynomial for $T$. Then according to Definition \ref{def:nullspace}
we obtain 
$$W = \{u\in V~|~g(T)u = 0\}$$

Let $k$ be a polynomial whose degree is larger than $f(x)$, i.e. $deg(k(x))>deg(f(x))$, then
$$f(x)|k(x)~~\text{iff}~k(T)=0$$

Since the construction of $f(x)$ relies on $T$, $f(T) = 0$, we get
$$\forall u, f(T)u = 0$$

Consider $W(u,T)$ subsequently relies on $g(T)$, we deduce 
$$dim(W) = deg(d(x))$$

According to the definition of $W$, we have
\begin{align*}
\forall u,~~& e(T)u \in W \Longleftrightarrow \\
& g(T)e(T)u = g(T)\frac{f(T)}{d(T)} u = h(T)\cdot f(T) u = 0
\end{align*}
\end{Proof}

The $T$-Order and corresponding nullspace also affect the factorization of polynomials:
\begin{Lemma}[Factorization of $f(x)$]
Factorization $f(x) = \prod_{i=1}^{r} f_{i}^{d_i} (x)$, where each $f_i(x)$ is prime to others. 
Assume $V_i$ be nullspace of $f_{i}^{d_i} (x)$, then $V = V_1\oplus V_2 \oplus \dots\oplus V_r$.

Furthermore, we can define a polynomial $\Psi_i(x) = f(x)/f_{i}^{d_i} (x)$, where 
$$\forall u_j \in V_j, u_j \neq 0, \Psi_i(T)u_j \neq 0$$
only if $i = j$.
\end{Lemma}
\begin{Proof}
We can use Lemma \ref{lem:app1}:
$$gcd(f(x),f_{i}^{d_i} (x)) = f_{i}^{d_i} (x),~~dim(V_i) = deg(f_{i}^{d_i} (x))$$

Which implies 
$$dim(V_i) = deg\left(\prod_i f_{i}^{d_i} (x)\right) = \prod_i dim(V_i) \implies V = \bigoplus_i V_i$$

Assume $i\neq j$, $\Psi_i(x) = \frac{f(x)}{f_{i}^{d_i} (x)} = h(x)f_{j}^{d_j} (x)$. Then
\begin{align*}
&\forall u_j\in V_j, f_j(T)u_j = 0 \implies h(x)f_{j}^{d_j} (T)u_j = 0 \\
& \implies \Psi_i(x)u_j = 0
\end{align*}

Conversely, if $i=j$, then
\begin{align*}
& \Psi_i(x) = \frac{f(x)}{f_{i}^{d_i} (x)} \perp f_i(x) \\
& \implies \Psi_i(x)u_j \neq 0
\end{align*}
\end{Proof}

For a given Frobenius map $\sigma$, corresponding minimal polynomial is restricted:
\begin{Lemma}
\label{lem:3}
The minimal (and characteristic) polynomial for $\sigma$ is $x^n-1$.
\end{Lemma}
\begin{Proof}
Consider Fermat's little theorem in $\F_{p^n}$. Let $\beta$ be an element in the field, then
$$\sigma^n\beta = \beta^{p^n} = \beta \implies \sigma^n-I = 0$$
where $I$ is the identity map. For characteristic polynomial, assume $\exists f(x) = \sum_i f_ix^i \in \F_p[x]$, 
such that the degree of characteristic polynomial is lower than $n$:
$$\sum_if_i\sigma^i = 0~~~\text{and}~~~deg(f(x))<n$$

Then $\forall \beta\in\F_{p^n}$, 
$$\left(\sum_if_i\sigma^i\right)\beta = \sum_if_i\beta^{p^i} = 0$$

This equation denotes that $\beta$ is one of $p^n$ roots of polynomial $\Func(x) = \sum_if_i(x^{p^i})$.
However, the maximum number of roots allowed equals to the degree of $\Func(x)$, which is 
$p^{n-1}<p^n$, we find a contradiction. Thus we conclude that
$$\textit{Both characteristic and minimal polynomial is }x^n-1$$
\end{Proof}

From lemmas above we can deduce the following corollary:
\begin{Corollary}
\label{corol:1}
An element $\alpha \in \F_{p^n}$ is a {\bf normal element} if and only if $Ord_{\alpha,\sigma}(x) = x^n-1$.
\end{Corollary}
\begin{Proof}
Normal bases require $\{\beta,\beta^2,\dots,\beta^{2^{n-1}}\}$ to be linearly independent,
which is equivalent to the following expression:
$$\forall f(x)\in \F_p[x],~deg(f(x))<n$$
Furthermore, because there is no annihilators in the $T$-Order, this implies that
$$Ord_{\alpha,\sigma}(x) = x^n-1$$
\end{Proof}

From corollary above we can deduce another form of criterion of normal element:
\begin{Theorem}
\label{thm:1}
Given finite field $\F_{2^n}$, the field characteristic $p = 2$.
Define $t = p^e$ such that $n = kp^e, gcd(k,p) = 1$, so $t=1$ if $n$ is an odd integer. Then 
$x^n - 1$ can be factorized as 
$$(\varphi_1(x)\varphi_2(x)\cdot\cdot\cdot\varphi_r(x))^t$$
Additionally, define $\Phi_i(x) = (x^n - 1)/\varphi_i(x)$. We assert that

An element $\alpha \in 	F_{p^n}$ is a normal element if and only if $\Phi_i(\sigma)\alpha \neq 0, i = 1,2,\dots,r$.
\end{Theorem}
\begin{Proof}
Let us analyze the auxiliary polynomial $\Phi_i(x) = \frac{x^n - 1}{\varphi_i(x)}$ first.
$$\Phi_i(\sigma)\alpha \neq 0 \Leftrightarrow \text{No factor in }x^n-1\text{ annihilates }\alpha$$
On the other hand, according to Lemma \ref{lem:3}, it is a known 
fact that \emph{any annihilator always divide} $x^n-1$. As a result, the only possible situation is
$$Ord_{\alpha,\sigma}(x) = x^n-1$$
Using Corollary \ref{corol:1} we deduce that $\alpha$ is a normal element.
\end{Proof}

The normal element identification can also be described from the nullspace perspective:
\begin{Theorem}
\label{thm:2}
Let $W_i$ be the nullspace of $\varphi_{i}^{t} (x)$ and $\widetilde{W_i}$ the
nullspace of $\varphi_{i}^{t-1} (x)$. Let $\overline{W_i}$ be any subspace of
$W_i$ such that $W_i = \overline{W_i}\oplus \widetilde{W_i}$. Then
$$\F_{p^n} = \displaystyle\sum_{i=1}^{r} \overline{W_i}\oplus \widetilde{W_i}$$
is a direct sum where $dim(\overline{W_i}) = d_i$ and $dim(\widetilde{W_i}) = (t-1)d_i$.

Using above setup we define an element $\alpha \in \F_{p^n}$ with 
$\alpha = \sum_{i=1}^{r} (\overline{\alpha_i} + \widetilde{\alpha_i}), \overline{\alpha_i} \in \overline{W_i}, \widetilde{\alpha_i} \in \widetilde{W_i}$,
as a normal element if and only if $\overline{\alpha_i} \neq 0,~ \forall i = 1,2,\dots,r$.
\end{Theorem}

Ultimately, since there always exist at least one element fulfilling requirements in Theorem \ref{thm:1}
or Theorem \ref{thm:2}, we obtain following theorem:
\begin{Theorem}[Normal Basis Theorem over Finite Fields]
There always exists a normal basis of $\F_{p^n}$ over $\F_p$.
\end{Theorem}
\subsection{Construction of General Normal Bases}
After proving the existence of normal bases, the upcoming question is that how to find such a normal basis/element.
For general normal basis identification, there are two methods widely used: the L\"uneburg's algorithm
and Lenstra's algorithm.\\
\textbf{L\"uneburg's Algorithm:}
\begin{enumerate}[{1)}]
\item Randomly pick an element $\alpha$ from $\F_{p^n}$. For each $i = 0,1,\dots,n-1$, 
compute $\sigma$-Order $f_i = Ord_{\alpha^i}(x)$. Then $x^n - 1 = lcm(f_0,f_1,...,f_{n-1})$.
\item Apply factor refinement to set $\{f_0,\dots,f_{n-1}\}$ and obtain $f_i = \prod_{1\leq j\leq r} g_{j}^{e_{ij}}, i = 0,1,\dots,n-1$.
We can write the result as an $i\times j$ matrix.
\item For each $j$, find an index $i_j$ (denote as $i(j)$) such that $e_{ij}$ is maximum in the $j$-th column.
\item Let $h_j = f_{i(j)}/g_{j}^{e_{i(j)j}}$, take $\beta_j = h_j(\sigma)\alpha^{i(j)}$. Then
$$\beta = \displaystyle\sum_{j=1}^{r} \beta_j$$
is a normal element.
\end{enumerate}

L\"uneburg's algorithm starts with a random element and ends with a normal element. The justification of 
its soundness is as follows:
\begin{Proposition}
L\"uneburg's algorithm always generate a normal element over field $\F_{p^n}$.
\end{Proposition}

\begin{Proof}
First, according to the definition of $T$-Order:
$$f_i = Ord_{\alpha^i} \Leftrightarrow f_i(\sigma)\alpha^i = 0$$

Use Lemma \ref{lem:3}, the minimal/characteristic polynomial for $\sigma$ is $x^n-1$ implies that
any annihilator of $\alpha^i$ ({\it i.e.} $f_i(\sigma)$) divides $x^n-1$. Meanwhile, $\{\alpha^i~|~0<i<n-1\}$ forms a (standard) basis
of field $\F_{p^n}$, this means $\alpha_i$ are linearly independent with each other ($i=0,1,\dots,n-1$).
Since linear map $f(\sigma)$ guarantees that all elements in the field 
have order $lcm(f_0,\dots,f_{n-1})$, i.e. $f(\sigma)\gamma = f(\sigma)(\sum_i \alpha^i) \implies f(\sigma) = lcm(f_0,\dots,f_{n-1})$.
By contradiction we can prove that there is no factor of $x^n-1$ that can be divided by the product 
of elements in set $\{f_0,f_1,\dots,f_{n-1}\}$. This actually corresponds to the assertion in the first step of the algorithm:
$$x^n-1 = lcm(f_0,f_1,\dots,f_{n-1})$$

In the second step, after factorization of set $\{f_i\}$, we transform $f_i(\sigma)\alpha^i=0$ to
$$h_j(\sigma)\cdot g_{j}^{e_{i(j)j}}(\sigma) \cdot \alpha^{i(j)} = 0$$

Furthermore we have 
$$g_{j}^{e_{i(j)j}}(\sigma)\beta_j = 0 \Leftrightarrow Ord_{\sigma,\beta_j}(x) = g_{j}^{e_{i(j)j}}(x)$$

Consider the facts: 1) elements in set $\{g_j\}$ are relatively prime; 2) $g_{j}^{e_{i(j)j}}$ is the maximum factor; 
3) $x^n-1 = lcm(f_0,f_1,\dots,f_{n-1})$, we deduce that 
$$x^n-1 = \prod_j g_{j}^{e_{i(j)j}}(x) = \prod_jOrd_{\sigma,\beta_j}(x)$$

As a result
$$x^n-1 = Ord_{\sigma,\beta}(x) \implies \beta\text{ is a normal element.}$$
\end{Proof}

L\"uneburg's algorithm is an analytic method, which requires relatively high computational complexity 
mainly because of the factorization in its second step. To overcome the high cost, an inductive method 
is proposed, which is the Lenstra's algorithm. It allows for setup of heuristics to accelerate the procedure.
Before introducing the details of the algorithm, we demonstrate two preliminary lemmas.
\begin{Lemma}
\label{lem:4}
For an arbitrary element $\theta \in \F_{p^n}$ that $Ord_\theta(x) \neq x^n - 1$,
let $g(x) = (x^n - 1)/Ord_\theta(x)$. There exists another element $\beta$ such that $g(\sigma)\beta = \theta$.
\end{Lemma}
\begin{Proof}
Assume $\gamma$ is the desired normal element. From the definition of normal element we derive 
$$\exists f(x)\in \F_{p^n}[x],~~f(\sigma)\gamma = \theta$$

Then using the definition of $T$-Order, we derive
$$Ord_{\sigma}\theta = 0 \implies (Ord_{\sigma}f(\sigma))\gamma = 0$$

Since $\gamma$ is the normal element, $Ord_\gamma(x) = x^n-1\implies x^n-1~|~(Ord_\theta(x)f(x))$.
Furthermore,
$$g(x) = \frac{x^n-1}{Ord_\theta(x)} \implies x^n-1\Bigm|\left(\frac{x^n-1}{g(x)}\cdot f(x)\right)\implies g(x)~|~f(x)$$

Let $f(x)=h(x)g(x)$. Then
$$g(\sigma)(h(\sigma)\gamma) = \theta$$

Therefore, $\exists \beta$ {\it i.e. }$g(\sigma)\beta = \theta$. Concretely, $\beta = h(\sigma)\gamma$.
\end{Proof}

\begin{Lemma}
\label{lem:5}
Define $\theta$ and $g(x)$ as in Lemma \ref{lem:4}. Assume there exists a solution 
$\beta$ such that $deg(Ord_\beta(x)) \leq deg(Ord_\theta(x))$. Respectively there exists a non-zero element $\eta$ such that
$$g(\sigma)\eta = 0$$
and
$$deg(Ord_{\theta+\eta}(x)) > deg(Ord_\theta(x))$$
\end{Lemma}
\begin{Proof}
Assume $\gamma$ is the desired normal element. Similarly,
$$\exists \eta = Ord_\theta(\sigma)\gamma \neq 0,~~g(\sigma)\eta = 0$$

Follow the setup in Lemma \ref{lem:4}, we have
$$g(\sigma)\beta = \theta,~~\frac{x^n-1}{Ord_\theta(x)}\Bigm|_\sigma\cdot \beta = \theta$$

Subsequently,
$$Ord_\theta(x)\cdot \frac{x^n-1}{Ord_\theta(x)}\Bigm|_\sigma\beta = 0 = Ord_\beta(x)\beta$$
which implies
$$Ord_\theta(x)\bigm|Ord_\beta(x),~~deg(Ord_\theta(x)) \leq deg(Ord_\beta(x))$$
Combining with the assumption in the lemma: $deg(Ord_\beta(x)) \leq deg(Ord_\theta(x))$, we derive
$$deg(Ord_\beta(x)) = deg(Ord_\theta(x))\implies Ord_\theta(x) =Ord_\beta(x)$$

In the next part we need to prove $g(x)\perp Ord_\theta(x)$ by contradiction. Assume 
$h(x) = gcd(g(x),Ord_\theta(x))\neq 1$, and
$$g(\sigma)\beta = a(\sigma)h(\sigma)\beta = \theta,~~Ord_\theta(\sigma)\theta = a(\sigma)b(\sigma)h^2(\sigma)\beta$$
However, $Ord_\theta(x) =Ord_\beta(x)$ indicates that $Ord_\beta(x)\beta = b(x)h(x)a(x)\beta = 0$. Thus,
$Ord_\theta(x)=b(x)$. This is true if and only if $h(x)=1$. As a result, we assert that
$$g(x) \perp Ord_\theta(x)$$

Consider $g(\sigma)\eta = 0 \implies Ord_\eta(x)\bigm|g(x)$. It further implies that 
$$Ord_\theta(x)\perp Ord_\eta(x) \implies Ord_{\theta+\eta}(x) = Ord_\theta(x)\cdot Ord_\eta(x)$$

Since $\eta \neq 0$, we derive the result
$$deg(Ord_{\theta+\eta}(x)) > deg(Ord_\theta(x))$$
\end{Proof}

Based on Lemmas \ref{lem:4} and \ref{lem:5}, the Lenstra's algorithm is described below:\\
\textbf{Lenstra's Algorithm}
\begin{enumerate}[{1)}]
\item Take an arbitrary element $\theta \in \F_{p^n}$, determine $Ord_\theta(x)$.
\item If $Ord_\theta(x) = x^n - 1$ then algorithm terminates and return $\theta$ as a normal element.
\item Otherwise, compute $g(x) = (x^n - 1)/Ord_\theta(x)$, and solve $\beta$ from $g(\sigma)\beta = \theta$.
\item Determine $Ord_\beta(x)$. If $deg(Ord_\beta(x)) > deg(Ord_\theta(x))$ then replace $\theta$ by $\beta$ and go to 2);
otherwise if $deg(Ord_\beta(x)) \leq deg(Ord_\theta(x))$ then find a non-zero element $\eta$ such that $g(\sigma)\eta = 0$,
replace $\theta$ by $\theta + \eta$ and determine the order of new $\theta$, and go to 2).
\end{enumerate}

Lenstra's algorithm is an approximation algorithm in nature. For each iteration $Ord_\theta(x)$ monotonically increase,
so it will finally reach termination condition $Ord_\theta(x) = x^n-1$.
\subsection{Bases Conversion}
% General \lambda matrix computation

\section{Optimal Normal Basis}
\label{append:ONB}
\subsection{Construction of Optimal Normal Basis}
\subsection{Optimal Normal Basis Multiplier Design}
% Customized Galois Field IEEE1363-2000
% Multi table <=> \lambda matrix
In Section \ref{sec:nbdesign}, we introduce the concept of multiplication table and explained how its 
entries are transformed from the $0$-th $\lambda$-Matrix. Actually we can mathematically prove the relationship 
between them:
\begin{Theorem}
Multiplication table $T$ is actually a conjugate of the $\lambda$-Matrix $M$. 
We can use an equation $M_{i,j}^{(0)} = T_{j-i,-i}$ to demonstrate the relation between their entries.
\end{Theorem}
\begin{Proof}
Recall the definition of $\lambda$-Matrix over field $\F_{2^k}$:
$$C = A\times B = \left(\sum_i a_i\beta^{2^i}\right)\left(\sum_jb_j\beta^{2^j}\right) = \sum_i\sum_ja_ib_j\beta^{2^i}\beta^{2^j}$$
Then there always exist $\lambda_{ij}^{(l)}$, such that 
$$\beta^{2^i}\beta^{2^j} = \sum_l\lambda_{ij}^{(l)}\beta^{2^l}$$
We call it the cross-product term. The $l$-th bit of the product $C$ is
$$c_l = \sum_i\sum_ja_ib_j\lambda_{ij}^{(l)}$$
For the sake of simplification, we let $i=0$, thus 
$$\beta\cdot\beta^{2^j} = \sum_l\lambda_{0j}^{(l)}\beta^{2^l}$$
Since the multiplication table is defined as
\begin{equation}
\beta
\begin{bmatrix}
\beta \\ \beta^2 \\ \beta^{2^2} \\ \vdots \\ \beta^{2^{n-1}}
\end{bmatrix}
= {\bf T}
\begin{bmatrix}
\beta \\ \beta^2 \\ \beta^{2^2} \\ \vdots \\ \beta^{2^{n-1}}
\end{bmatrix}
\end{equation}
The $j$-th row of $T$ can be written as
\begin{align}
\beta\cdot\beta^{2^j} =& \sum_l\lambda_{0j}^{(l)}\beta^{2^l} & \text{(multi-table definition)} \nonumber\\
					=& \sum_l\lambda_{jl}^{(0)}\beta^{2^l} & \text{(cross-product term)} \label{eqn:mteq}
\end{align}
Notice that $\lambda$ corresponds to $\lambda$-Matrix entries. In Equation \ref{eqn:mteq} we assign $0$ to the 
row index $i$, the proof can be extended to all row indices $i<n$ but fix $l$ to 0 because of the 
conjugation generated by right-down cyclic-shift of $\lambda$-Matrix. Therefore we have 
$$M_{i,j}^{(0)} = \lambda_{ij}^{(0)} = \lambda_{j,0}^{(i)} = T_{j-i,0-i}$$
\end{Proof}