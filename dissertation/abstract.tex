\begin{center}{\bf ABSTRACT}\end{center}
Sequential equivalence and model checking techniques are widely
employed in hardware and software verification. With the increasing
size and complexity of hardware/software systems, abstraction has
become a key component to address scalability.
This dissertation investigates new abstraction techniques at word-level and their applications 
on sequential circuits verification. By bundling together $k$ bit-level state-variables into
one word-level constraint expression, the state-space can be construed
as a polynomial function over the finite (Galois) field of $2^k$
elements. As long as the state-space are modeled as polynomial ideals
over word-level variables, algebraic geometry techniques including ideal manipulations,
Gr\"obner bases reasoning and finite field algebra are researched to 
facilitate new word-level abstractions.

Algebraic geometry offers a very powerful set of tools to reason about
the properties of the solutions to a system of polynomial
constraints. Moreover, the algebraic model inherently provides a
framework for abstraction. 
The algebraic geometry algorithms (Groebner basis computations) exhibit
high complexity which prevent it from being widely used in conventional techniques. 
Interestingly, recent experience has shown that
by analyzing the topology of circuits in conjunction with Gr\"obner
basis computations helps to identify the structure and symmetry
inherent in the problem, this lower the computational complexity.

Formal verification of sequential circuits requires the analysis of
the underlying finite state machine. Reachability analysis becomes a
fundamental tool for this purpose. We devised traditional traversal algorithms 
using the new abstraction method to perform
the reachability analysis at word-level. 
The abstractions are also exploited on functional
verification of sequential Galois field arithmetic circuits, where conventional 
techniques fail to provide scalability. At last, we propose a new method to extract 
unsatisfiable cores and utilize the information about the cores to 
perform model checking with abstraction refinement.