\begin{center}{\bf ABSTRACT}\end{center}

% Verification of sequential circuits remains to be a topic of all time. With increasing size of new
% integrated circuits, sequential circuit designers may face much more complicated problems on 
% logic errors, unexpected delays and failures on critical path. 
% This dissertation addresses the problem of sequential circuit verification at the word-level
% and is based on concepts derived from algebraic geometry. 
% 
% Analyzing the sequential circuits at word level is an efficient way
% of {\it abstraction}, which may lower the complexity of verification by efficient representation of the
% state-space. We manage to model the verification properties and the gate-level sequential circuit 
% implementations over Galois fields of the type ${\mathbb{F}}_{2^k}$ by means of polynomial ideals and 
% their canonical representations --- Gr\"obner bases --- at the level of $k$-bit words. Subsequently, 
% techniques from algebraic geometry can be used to reason about the state-space on high-level.
% In algebraic geometry, ``bad" states are modeled as varieties while whole system is described using 
% polynomial ideals. Without actually solving the system, the states in state space can be investigated 
% by manipulating these ideals. 
% 
% We propose to apply these techniques to traverse a finite state machine (FSM) for reachability 
% analysis at the word-level, and also to implicitly unroll sequential arithmetic circuits to 
% verify their function. Moreover, as unsatisfiable (UNSAT) cores play an important role in modern 
% abstraction-refinement techniques for verification, we propose to investigate a word-level 
% analogue of the UNSAT core problem using the  Gr\"obner basis algorithm. The dissertation will 
% not only derive new algorithms and techniques, but will also consider efficient CAD 
% implementations to target sequential equivalence and model checking problems. 
% % \end{abstract}

This dissertation presents work on a new class of terahertz (THz) waveguides based on structured 
metal geometries. The waveguides are designed with the core idea that adoption of planar layout in 
fabrication can lead to exponential growth in device capabilities, analogous to the growth in device 
capabilities based in electronics. From a functional point of view, the waveguides rely upon 
propagation of surface waves along the surface of metals. This approach is preferred, since 
dielectrics tend to be lossy at THz and the loss parameters scales almost quadratically with 
frequency for most dielectrics. The loss in propagating wave is minimized by utilizing metals, 
which are highly conducting at THz frequencies. Structuring the metal surface with periodic array 
of apertures of sub-wavelength dimension allows bound surface wave to propagate as the wave can 
evanescently decay into the metal. This phenomenon is referred as the coupling of propagating wave 
to surface plasmon polariton (SPP) like mode at the interface of structured metal surface and air. 
Thus, these propagating THz waves are simply surface plasmon-polaritons (SPPs) . Similarly, 
complimentary structures that do not perforate the metal, but rather stand on the metal surface 
also support SPPs. The wavelength of SPPs can be controlled by changing the dimension of these 
apertures/structures, since the dispersion relationship of the medium depends on the geometrical size. 
This engineering capability has been exploited in creating all the waveguides presented in this thesis. 
The devices presented are categorized based on the fabrication technique. Each technique is unique 
in its own regard and can be selected based on functional needs. A commonly adopted process of laser 
ablation covers a wide set of waveguides presented here. In one of the waveguides fabricated using 
ablation technique, the role of disorder is discussed. The waveguide with introduction leads to 
observance of localized mode with spectral and spatial feature like Anderson localized modes of photons. 
It is the first report of localized mode at THz frequency. 3D rapid prototyping involving 3D printer is 
used to create waveguides with complex layout that can allow for multi-plane signal routing. 
This also is the first demonstration of 3D printing in the development of THz devices. In another 
approach a unique fabrication technique had to be developed to create waveguides mediums with negative 
index of refraction (NIM) as they require feature sizes which cannot easily be attained using conventional 
clean room techniques and 3D printing. This new fabrication approach uses a sacrificial layer technique 
that is used create an effective medium with negative index of refraction and length on the order of tens 
of wavelength. Again, this happens to be the first demonstration of a waveguide with NIM capability at 
THz length scales, as well as a figure of merit (FOM) defined in terms of loss parameter that is an order 
of magnitude better than the previous reported NIM’s in the THz spectral range.
