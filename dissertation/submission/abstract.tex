%\begin{center}{\bf ABSTRACT}\end{center}
Formal verification of hardware designs has become an essential
component of the overall system design flow. The designs are generally
modeled as finite state machines, on which property and equivalence
checking problems are solved for verification. Reachability analysis
forms the core of these techniques. However, increasing
size and complexity of the circuits causes the state explosion
problem. Abstraction is the key to tackling the scalability challenges.

This dissertation presents new techniques for word-level abstraction
with applications in sequential design verification. By bundling
together $k$ bit-level state-variables into one word-level constraint 
expression, the state-space is construed as solutions (variety) to
a set of polynomial constraints (ideal), modeled over the finite
(Galois) field of $2^k$ elements. Subsequently, techniques from
algebraic geometry -- notably, Gr\"obner basis theory and technology
-- are researched to perform reachability analysis and
verification of sequential circuits. This approach adds a ``word-level
dimension'' to state-space abstraction and verification to make the
process more efficient.

While algebraic geometry provides powerful abstraction and reasoning
capabilities, the algorithms exhibit high computational complexity. In
the dissertation, we show that by analyzing the constraints, it is
possible to obtain more insights about the polynomial ideals, which can be
exploited to overcome the complexity. Using our algorithm design and
implementations, we demonstrate how to perform reachability
analysis of finite-state machines purely at the word level. Using this
concept, we perform scalable verification of sequential arithmetic
circuits. As contemporary approaches make use of resolution proofs and
unsatisfiable cores for state-space abstraction, we introduce the
algebraic geometry analog of unsatisfiable cores, and present
algorithms to extract and refine unsatisfiable cores of polynomial
ideals. Experiments are performed to demonstrate the efficacy of our
approaches. 
