\chapter{Optimal Normal Basis}
\label{append:ONB}
The number of non-zero entries in $\lambda$-Matrix or multiplication table (M-table) is known as \emph{Complexity $(C_N)$}.
To define optimal normal basis, it is necessary to find the lower bound of $C_N$.
\begin{Theorem}
If $\mathcal{N}$ is a normal basis over $\F_{p^n}$ with $\lambda$-Matrix $M^{(k)}$, then non-zero entries in matrix $C_N\geq 2n-1$.
\end{Theorem}
\begin{Proof}
Let basis $\mathcal{N} = \{\beta, \beta^p, \beta^{p^2},\dots, \beta^{p^{n-1}}\}$, let us denote $\beta^{p^i}$ by $\beta_i$ for the sake of simplification.
Then $\sum_{i=0}^{n-1} \beta_i = trace\ \beta$. 
Denote $trace\ \beta$ by $b$, consider a $n\times n$ matrix $M^{(0)}$. Then
$$b\beta_0 = \sum_{i=0}^{n-1} \beta \beta_i$$

Therefore, the sum of all rows in $M^{(0)}$ is an n-tuple with $b$ as the first element and zeros elsewhere.
So the first column always includes at least one nonzero element. 
Since the sum of entries in each of other columns should equal to zero (modulo 2), they 
need to include an even number of ``1"s.
Meanwhile, in order to maintain the linear independence among each row, they cannot be all ``0"s. 
Therefore, there are always at least two non-zero elements in each column.

As a result, $C_N \geq 2(n-1)+1  = 2n-1$.
\end{Proof}

If there exists a set of normal basis satisfying $C_N = 2n - 1$, this normal basis is named as 
\emph{Optimal Normal Basis} (ONB).

\begin{Example}
\label{ex:app4}
In $\F_{2^4}$ constructed with $p(\alpha) = \alpha^4 + \alpha + 1$, two normal bases can be found:
$\beta = \alpha^3, \mathcal N_1 = \{ \alpha^3, \alpha^6, \alpha^{12}, \alpha^9\}$ and 
$\beta = \alpha^7, \mathcal N_2 = \{\alpha^7, \alpha^{14}, \alpha^{13}, \alpha^{11}\}$. 
Their multiplication tables are listed below. For basis $\mathcal N_1$:
\begin{equation}
T_1 = \left(
\begin{array}{lccr}
0 & 1 & 0 & 0\\
0 & 0 & 0 & 1\\
1 & 1 & 1 & 1\\
0 & 0 & 1 & 0
\end{array} \right)
\end{equation}
Complexity $C_N$ = 7, so $\mathcal N_1$ is an ONB. For basis $\mathcal N_2$:
\begin{equation}
T_2 = \left(
\begin{array}{lccr}
0 & 1 & 0 & 0\\
1 & 1 & 0 & 1\\
1 & 0 & 1 & 0\\
1 & 0 & 1 & 1
\end{array} \right)
\end{equation}
Complexity $C_N$ = 9, so basis $\mathcal N_2$ is not optimal.
\end{Example}

ONBs are widely used in finite field circuit design, not only because its low complexity, 
but also because of the convenience to construct the $\lambda$-Matrix and M-table for them. 
To explain the convenience and the reason behind it, we introduce two types of ONBs and the method 
to construct them.
The following section refers to \cite{rosingbook}.
\section{Construction of Optimal Normal Basis}
{\bf Type-I ONB} over $F_{2^n}$ satisfies following criteria:
\begin{itemize}
\item $n+1$ must be prime.
\item 2 must be primitive in $\mathbb{Z}_{n+1}$.
\end{itemize}
The second criterion indicates that powers of 2 (exponent from 0 to $n-1$) modulo $n+1$ must cover all integers
from 1 to $n$.

In following, we derive a simple way to construct the $\lambda$-Matrix of type-I ONB 
from the criteria above. 
Assume $\lambda_{ij}^{(k)}$ is the entry with coordinate $(i,j)$ from k-th $\lambda$-Matrix. Then the crossproduct
term can be written as
\begin{equation}
\beta^{2^i}\beta^{2^j} = \sum_{k=0}^{n-1} \lambda_{ij}^{(k)} \beta^{2^k}
\end{equation}
Suppose we only care about k=0. So simplified to following equations:

$$
\begin{cases}
&\beta^{2^i}\beta^{2^j} = \beta\\
&\beta^{2^i}\beta^{2^j} = 1 ~~~(\text{ if }~  2^i = 2^j\pmod{n+1})
\end{cases}$$

Solution $(i,j)$ implies location of entries that equal to ``1" in $\lambda$-Matrix. 
Let $\beta$ be an optimal normal element, 
$\{\beta^{2^i}~|~0\leq i<n\}$ cover all powers of $\beta$ and generates the basis. Thus by solving
$$\begin{cases}
2^i + 2^j = 1 \pmod{n+1}\\
2^i + 2^j = 0 \pmod{n+1}
\end{cases}$$
we obtain $\lambda$-Matrix $M^{(0)}$. This construction method does not require information about 
primitive polynomial nor normal element, so it is very convenient for circuit designers.
Basis $\mathcal N_1$ in Example \ref{ex:app4} is a type-I ONB over $\F_{2^4}$.

{\bf Type-II ONB} over $\F_{2^n}$ satisfies following criteria:
\begin{itemize}
\item $2n+1$ must be prime. And either
\item 2 is primitive in $\mathbb{Z}_{2n+1}$, or
\item $2n+1 = 3 \pmod{4}$ and 2 generates the quadratic residues in $\mathbb{Z}_{2n+1}$
\end{itemize}
The last criterion means that that $2n+1$ should be congruent to $1$ or $3$ modulo 4. 
To generate a Type-II ONB, we first pick an element $\gamma$ with order $2n+1$ in $\F_{2^{2n}}$, 
such that the corresponding normal element $\beta$
from $F_{2^n}$ can be written as $\beta = \gamma + \gamma^{-1}$. Thus the cross-product terms will be:
\begin{equation}
\begin{split}
\beta^{2^i}\beta^{2^j} &= (\gamma^{2^i} + \gamma^{-2^i})(\gamma^{2^j} + \gamma^{-2^j})\\ &= 
(\gamma^{2^i+2^j} + \gamma^{-(2^i+2^j)}) + (\gamma^{2^i-2^j} + \gamma^{-(2^i-2^j)})\\ &= 
\begin{cases}
\beta^{2^k} + \beta^{2^{k\prime}}~~~\text{if }~ 2^i \neq 2^j \pmod{2n+1}\\
\beta^{2^k}~~~~~~~~~~~~~~\text{if }~ 2^i = 2^j \pmod{2n+1}
\end{cases}
\end{split}
\end{equation}
$k$ and $k\prime$ are the 2 possible solutions to multiplication of any 2 basis elements. 
This guarantees the optimum of the basis since it has the minimum number of possible terms. 
In the case of $2^i \neq 2^j \pmod{2n+1}$, at least one of following
equations
\begin{equation}
\label{eqn:app1st}
\begin{cases}
2^i + 2^j = 2^k ~~~\pmod{2n+1}\\
2^i + 2^j = -2^k \pmod{2n+1}
\end{cases}
\end{equation}
has a solution, meanwhile at least one of following equations
\begin{equation}
\label{eqn:app2nd}
\begin{cases}
2^i - 2^j = 2^{k\prime} ~~~\pmod{2n+1}\\
2^i - 2^j = -2^{k\prime} \pmod{2n+1}
\end{cases}
\end{equation}
has a solution as well.

In another case that $2^i = \pm 2^j \pmod{2n+1}$, at least one of the following 4 equations has a solution
\begin{equation}
\label{eqn:app3rd}
\begin{cases}
2^i + 2^j = 2^k ~~~\pmod{2n+1}\\
2^i + 2^j = -2^k \pmod{2n+1}\\
2^i - 2^j = 2^k ~~~\pmod{2n+1}\\
2^i - 2^j = -2^k \pmod{2n+1}
\end{cases}
\end{equation}
In set of Equations \ref{eqn:app1st} and \ref{eqn:app2nd}, there are two possible solutions in total. 
In set of Equations \ref{eqn:app3rd},
there is only one possible solution. Since these equations are all similar, instead of 
working with two different sets we can combine them together and solve a system of 4 equations. 
As a result, to construct 
the $\lambda$-Matrix $M^{(0)}$, we set $k=0$ and find solutions to:
\begin{equation}
\label{eqn:app4th}
\begin{cases}
2^i + 2^j = 1 ~~~\pmod{2n+1}\\
2^i + 2^j = -1 \pmod{2n+1}\\
2^i - 2^j = 1 ~~~\pmod{2n+1}\\
2^i - 2^j = -1 \pmod{2n+1}
\end{cases}
\end{equation}

\begin{Example} 
By solving the system of Equations \ref{eqn:app4th} with $n=5$, we find 9 pairs of
indices $(i,j)$ such that $0\leq i,j <5$. Assign ``1" to corresponding entries in a $5\times5$ matrix,
the result is $M^{(0)}$ for type-II ONB over $\F_{2^5}$:
\begin{equation}
M^{(0)} = \left(
\begin{array}{lcccr}
0 & 1 & 0 & 0 & 0\\
1 & 0 & 0 & 1 & 0\\
0 & 0 & 0 & 1 & 1\\
0 & 1 & 1 & 0 & 0\\
0 & 0 & 1 & 0 & 1
\end{array} \right)
\end{equation}
\end{Example}

\section{Optimal Normal Basis Multiplier Design}
% Customized Galois Field IEEE1363-2000
Designers can easily generate the $\lambda$-Matrix for GF multipliers, which is sufficient to derive the 
structure of the circuit. However, our GB based approach requires specifying the exact normal element $\beta$,
{\it i.e. } it is necessary to obtain $t$ as $\beta = \alpha^t$ before executing our technique.
L\"uneburg's algorithm and Lenstra's algorithm do not guarantee the output normal element is optimal 
normal element, and usually result in high computation cost. Actually if the type of ONB within the design 
is known, instead of looking up the optimal normal element, we can construct a special irreducible $p(\alpha)$
such that optimal normal element equals to the primitive element: $\beta=\alpha$. The concepts and algorithms 
refer to IEEE standard 1363-2000 \cite{IEEE1363}.

For type-I ONB over $\F_{2^k}$, the irreducible polynomial is 
$$p(\alpha) = \alpha^k+\alpha^{k-1}+\cdots+\alpha+1
= \sum_{i=0}^k \alpha^i$$ 

For type-II ONB, the following iterative algorithm is required to generate desired 
irreducible polynomial:

\begin{algorithm}[H]
\SetAlgoNoLine
\LinesNumbered
 \KwIn{Field $\Fkk$ which contains a type-II ONB}
 \KwOut{The irreducible polynomial $p(\alpha)$ for the ONB}

  $f(\alpha)\gets 1$\;
  $p(\alpha)\gets \alpha+1$\;
  \For{$i=1\dots k-1$}
  {
  	$g(\alpha)\gets f(\alpha)$\;
	$f(\alpha)\gets p(\alpha)$\;
	$p(\alpha)\gets \alpha f(\alpha)+g(\alpha)$\;
  }\
\Return{$p(\alpha)$}
\caption {Generating irreducible polynomial for type-II ONB over $\Fkk$}
\label{alg:ieee1363}
\end{algorithm}
\DecMargin{1em}
\vspace{0.2in}