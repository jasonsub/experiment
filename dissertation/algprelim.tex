\chapter{Computer Algebra Fundamentals} \label{ch:ideals}
This chapter reviews fundamental concepts of commutative and 
computer algebra which are used in this work. 
Specifically, this chapter covers monomial ordering, polynomial ideals and 
varieties, and the computation of \Grobner bases.
It also overviews elimination theory as well as Hilbert's Nullstellensatz 
theorems and how
they apply to Galois fields. The results of these theorems are used in polynomial
abstraction and formal verification of Galois field circuits and 
are discussed 
in subsequent chapters. The material of this chapter is
mostly referred from the textbooks \cite{ideals:book} \cite{gb_book} and 
previous work by {\it Lv} \cite{lv:phd}. 

\section{Monomials, Polynomials, and Term Orderings}

\begin{Definition} \label{def:mono}
A {\bf monomial} in variables $x_1,x_2,\cdots,x_d$ is a product of the form:
\begin{equation}
x_1^{{\alpha}_1} \cdot x_2^{{\alpha}_2} \cdot \cdots x_d^{{\alpha}_d},
\end{equation}
where $\alpha_i \ge 0, i\in\{1,\cdots,d\}$. 
The total degree of the monomial is $\alpha_{1}+\cdots+\alpha_{d}$.
\end{Definition} 

Thus, $x^2\cdot y$ is a monomial in variables $x,y$ with total degree $3$.
For simplicity, we will henceforth denote a 
monomial $x_1^{{\alpha}_1} \cdot x_2^
{{\alpha}_2} \cdot \cdots x_d^{{\alpha}_d}$ as $x^{\alpha}$, 
where $\alpha=({\alpha}_1,\cdots,{\alpha}_d)$ is a vector size $d$ of 
integers $\ge 0$, i.e., $\alpha \in \mathbb{Z}_{\ge 0}^{d}$. In the

\begin{Definition}
A {\bf multivariate polynomial} $f$ in variables $x_1, x_2, \ldots, x_d$ 
with coefficients in any given field $\K$ is a finite linear 
combination of monomials with coefficients in $\K$: 
\begin{equation}
	f=\sum_{\alpha}a_{\alpha}\cdot x^{\alpha}, ~~a_{\alpha}\in \K \nonumber
\end{equation}

The set of all polynomials in $x_1, x_2, \ldots, x_d$ with coefficients in 
field $\K$ is denoted by $\K[x_1, x_2, \ldots, x_d]$. 
Thus, $f \in \K[x_1, x_2, \ldots, x_d]$

\begin{enumerate}
\item We refer to the constant $a_{\alpha} \in \K$ as the 
{\bf coefficient} of the monomial $a_{\alpha} x^{\alpha}$.
\item If $a_{\alpha} \neq 0$, we call $a_{\alpha} x^{\alpha}$ a term of $f$.
\end{enumerate}
\end{Definition}

As an example, $2x^2+y$ is a polynomial with two terms $2x^2$ and $y$, with 
$2$ and $1$ as coefficients respectively. 
In contrast, $x+y^{-1}$ is not a polynomial because the exponent of $y$ is 
less than $0$.

Since a polynomial is a sum of its terms,
these terms have to be arranged unambiguously so that they can be 
manipulated in a consistent manner.
Therefore, we need to establish a concept of 
{\bf term ordering} (also called monomial ordering).
A term ordering, represented by $>$, defines how
terms in a polynomial are ordered.
%%%%%%%%%%%%%%%%%%%   monomial    Ordering   %%%%%%%%%%%%%%%%%%%%%%%%%%%%
\begin{Definition}
Let $\mathbb{T}^{d}=\{x^{\alpha}: \alpha\in \mathbb{Z}_{\ge 0}^{d}\}$ be the set of all monomials in $x_{1},\dots,x_{d}$.
A {\bf monomial order} $>$ on $\mathbb{T}^{d}$ is a total well-ordering satisfying:
\begin{itemize}
        \item For all $x^{\alpha}, x^{\beta} \in \mathbb{T}^{d}$, $x^{\alpha}$ and $x^{\beta}$ are comparable
	\item For any $x^{\alpha} \in \mathbb{T}^{d}$, $x^{\alpha}>1$
	\item For all $x^{\alpha}, x^{\beta}, x^{\gamma} \in \mathbb{T}^{d}$, $x^{\alpha}>x^{\beta} \Rightarrow x^{\alpha} \cdot x^{\gamma}> x^{\beta} \cdot x^{\gamma}$
\end{itemize}
\end{Definition}

Term-orderings are totally ordered, i.e. anti-symmetric with constant terms
last in the ordering.
A total-order ensures that there is no ambiguity with respect to where a 
term is found in the term-ordering.  Total orderings for monomials come in 
different forms, notably 
lexicographic orderings (lex), and its variants: degree-lexicographic ordering (deglex) 
and reverse degree-lexicographic ordering (degrevlex).

A {\bf lexicographic ordering} (lex) is a total-ordering $>$ such that 
variables in the terms are lexicographically ordered, i.e. simply based on 
when the variables appear in the ordering.
Higher variable-degrees take 
precedence over lower degrees for equivalent variables (e.g. $a^3 > a^2$ due to $a \cdot a \cdot a > a \cdot a \cdot 1$).
\begin{Definition}
{\bf Lexicographic order:} Let $x_1 > x_2 > \dots > x_d$
lexicographically. Also let $\alpha = (\alpha_1, \dots, \alpha_d);
~\beta = (\beta_1, \dots, \beta_d) \in \mathbb{Z}^d_{\geq 0}$. Then we
have: 
\begin{equation}
x^{\alpha} > x^{\beta} \iff 
\begin{cases}
& \text{Starting  from the  left, the first co-ordinates of $\alpha_i, \beta_i$} \\
& \text{that are different satisfy $\alpha_i > \beta_i$}

\end{cases}
\end{equation}
\end{Definition}

A {\bf degree-lexicographic ordering} (deglex) is a total-ordering $>$ such 
that the total degree of a term takes precedence over the lexicographic 
ordering.  
A {\bf degree-reverse-lexicographic ordering} (degrevlex) is the same as a
deglex ordering, however terms are lexed in reverse.

\begin{Definition}
{\bf Degree Lexicographic order:} Let $x_1 > x_2 > \dots > x_d$
lexicographically. Also let $\alpha = (\alpha_1, \dots, \alpha_d);
~\beta = (\beta_1, \dots, \beta_d) \in \mathbb{Z}^d_{\geq 0}$. Then we
have: 
\begin{equation}
x^{\alpha} > x^{\beta} \iff 
\begin{cases}
\sum_{i=1}^{d}\alpha_i > \sum_{i=1}^{d} \beta_i & \text{ or }\\
\sum_{i=1}^{d}\alpha_i = \sum_{i=1}^{d} \beta_i  \text{ and }
x^{\alpha} > x^{\beta} & \text{w.r.t. lex order}
\end{cases}
\end{equation}
\end{Definition}


\begin{Definition}
{\bf Degree Reverse Lexicographic order:} Let $x_1 > x_2 > \dots > x_d$
lexicographically. Also let $\alpha = (\alpha_1, \dots, \alpha_d);
~\beta = (\beta_1, \dots, \beta_d) \in \mathbb{Z}^d_{\geq 0}$. Then we
have: 
\begin{equation}
x^{\alpha} > x^{\beta} \iff 
\begin{cases}
\sum_{i=1}^{d}\alpha_i > \sum_{i=1}^{d} \beta_i  \text{ or }\\
\sum_{i=1}^{d}\alpha_i = \sum_{i=1}^{d} \beta_i  \text{ and the first co-ordinates}\\
\text{$\alpha_i, \beta_i$ from the right, which are different, satisfy $\alpha_i < \beta_i$}
\end{cases}
\end{equation}

\end{Definition}

Applying these term orderings, we have the following relations, 
where $a > b > c$.

\begin{eqnarray}
    \eqntext{lex:}  
    a^2b > a^2 > abc > ab > ac^2 > ac > b^2c > b^2 > bc^3 > 1 
    \label{ex:ordering:lex}\\
    \eqntext{deglex:} 
    bc^3 > a^2b > abc > ac^2 > b^2c > a^2 > ab >
    ac > b^2 > 1
    \label{ex:ordering:deglex}\\
    \eqntext{degrevlex:}  
    bc^3 > a^2b > abc > b^2c > ac^2 > a^2 > ab >
    b^2 > ac > 1
    \label{ex:ordering:degrevlex}
\end{eqnarray}

The difference between the {\it lex} and two {\it deg-} orderings is
obvious, while the difference between the two degree-based orderings can be
seen by considering from which direction the term is lexed, e.g. $a\cdot c\cdot c > b\cdot b\cdot c$ (deglex, left-to-right)
versus $b\cdot b\cdot c > a \cdot c\cdot c$ (degrevlex, right-to-left). 

\begin{Example}
Let $f = 2x^2yz + 3xy^3 - 2x^3$. The effects of different term orderings on 
$f$ are:
\begin{itemize}
\item lex $x> y> z$: $f = -2x^3 + 2x^2yz + 3xy^3$
\item deglex $x>y>z$:  $f = 2x^2yz + 3xy^3 -2x^3$
\item degrevlex $x>y>z$: $f = 3xy^3 + 2x^2yz - 2x^3$
\end{itemize}
\end{Example}

%Based on the {\it monomial ordering}, we have the following concepts:

\begin{Definition}
The {\bf leading term} is the first term in a term-ordered polynomial.
Likewise, the {\bf leading coefficient} is the coefficient of
the leading term. 
Finally, a {\bf leading monomial} is the leading term 
lacking the coefficient.  We use the following notation:
\begin{eqnarray}
     lt(f)&& \text{--- Leading Term} \\
     lc(f)&& \text{--- Leading Coefficient} \\
     lm(f)&& \text{--- Leading Monomial} \\
     tail(f)&& f - lt(f)
\end{eqnarray}
\end{Definition}

\begin{Example}
\begin{eqnarray}
     f      &=& 3a^2b + 2ab + 4bc \\
     lt(f)  &=& 3a^2b \\
     lc(f)  &=& 3 \\
     lm(f)  &=& a^2b \\
     tail(f) &=& 2ab+4bc
\end{eqnarray}
\end{Example}

{\bf Polynomial division} is an operation over polynomials that is dependent on
the imposed monomial ordering. Dividing a polynomial $f$ by another polynomial
$g$ cancels the leading term of $f$ to derive a new polynomial. 

\begin{Definition}
Let $\K$ be a field and let $f, g \in \K[x_1, x_2, \ldots, x_d]$ be polynomials
over the field. {\bf Polynomial division} of $f$ by $g$ is computes following:
\begin{equation}
f-\frac{lt(f)}{lt(g)}\cdot g
\end{equation}
This polynomial division is denoted
\begin{equation}
f\xrightarrow{g} r
\end{equation}
where $r$ is the resulting polynomial of the division.
If $\frac{lt(f)}{lt(g)}$ is non-zero, then $f$ is considered divisible by $g$, 
i.e. $g \mid f$.
\end{Definition}
Notice that if $g \nmid f$, that is if $f$ is not divisible by $g$,
then the division operation gives $r = f$.
\begin{Example}
Over $\R[x,y,z]$, set the lex term order $x > y > z$.
Let $f = -2x^3 + 2x^2yz + 3xy^3$ and $g = x^2+yz$.
\begin{equation}
\frac{lt(f)}{lt(g)} = \frac{-2x^3}{x^2} = -2x
\end{equation}
Since $\frac{lt(f)}{lt(g)}$ is non-zero $g|f$. The division, $f\xrightarrow{g} r$, 
is computed as:
\begin{eqnarray}
r &=& f-\frac{lt(f)}{lt(g)}\cdot g = -2x^3 + 2x^2yz + 3xy^3 - (-2x \cdot (x^2+yz)) \nonumber \\
&=& -2x^3 + 2x^2yz + 3xy^3 - (-2x^3-2xyz) = 2x^2yz + 3xy^3 + 2xyz
\end{eqnarray}
Notice that the division cancels the leading term of $f$.
\end{Example}

\section{Varieties and Ideals}
%%%%%%%%%%%%%%%%%%%%%%%%%%%%%%%%%%%%%%%%%%%%%%%%%%%%%%%%%%%%%%%%%%%%%%%%%%%%
%%%%%%%%%%%%%%%%%%%%%  variety %%%%%%%%%%%%%%%%%%%%%%%%%%%%%%%%%%%%%%%%%%
In computer-algebra based formal verification,
it is often necessary to analyze the presence 
or absence of solutions to a given system of constraints.
In our applications, these constraints are polynomials and their solutions 
are modeled as {\bf varieties}.

\begin{Definition}
Let $\K$ be a field, and let $f_1, \ldots, f_s 
\in \K[x_1, x_2, \ldots, x_d]$. 
We call $V(f_1, \dots, f_s)$ the {\bf affine variety} 
defined by $f_1, \dots, f_s$ as:
\begin{equation}
V(f_1, \ldots, f_s)= \{(a_1, \ldots, a_{d})\in \K^d:f_i(a_1, \ldots, a_d)=0, \forall{i},1\le i \le s\}
\end{equation}
\end{Definition}

$V(f_1, \dots, f_s)\in \K^d$ is {\bf the set of all solutions} in $\K^d$ 
of the system of equations: 
$f_1(x_1,\ldots,x_d)=\dots=f_s(x_1,\dots,x_d)=0$. 

\begin{Example}
Given $\mathbb{R}\left[x,y\right]$, $V(x^2+y^2)$ is the set of all elements
that satisfy $x^2+y^2=0$ over $\mathbb{R}^2$. So $V(x^2+y^2)=\{(0,0)\}$. 
Similarly, in $\mathbb{R}\left[x,y\right]$, $V(x^2+y^2-1)=\{all\  points\  on\ the\ circle: x^2+y^2-1=0\}$.
Note that varieties depend on which field we are operating on. 
For the same polynomial $x^2+1$, we have:
\begin{itemize}
\item In $\mathbb{R}[x]$, $V(x^2+1)=\emptyset$.
\item In $\mathbb{C}[x]$, $V(x^2+1)=\{(\pm i)\}$.
\end{itemize}
\end{Example}

The above example shows the variety can be infinite, finite (non-empty set) 
or empty. It is interesting to note that since we will be operating over
finite fields $\F_{q}$, and any finite set of points is a variety. Likewise,
any variety over $\F_{q}$ is finite (or empty).
Consider the points $\{(a_1,\dots, a_d): a_1, \dots, a_d \in \F_q\}$
in $\F_q^d$. Any single point is a variety of some polynomial system:
e.g. $(a_1,\dots, a_d)$ is a variety of $x_1-a_1 = x_2 - a_2 = \dots =
x_d-a_d=0$. {\bf Finite unions} and {\bf finite  intersections} of
varieties are also varieties. 

\begin{Example}
Let $U = V(f_1, \dots, f_s)$ and $W =
V(g_1, \dots, g_t)$ in $\F_{q}$. Then:  
\begin{itemize}
\item $U \cap W = V(f_1, \dots, f_s, g_1, \dots, g_t)$
\item $U \cup W = V(f_i g_j: 1 \leq i \leq s, 1 \leq j \leq t)$
\end{itemize}
\end{Example}

One important distinction we need to make about varieties is that a
variety depends not just on the given system of polynomial 
equations, but rather on the {\bf ideal} generated by the polynomials.

%%%%%%%%%%%%%%%%%%%%%  ideal %%%%%%%%%%%%%%%%%%%%%%%%%%%%%%%%%%%%%%%%%%
\begin{Definition} 
A subset $I \subset \K[x_1, x_2, \ldots, x_d]$ is an {
\bf ideal} if it satisfies:
\begin{itemize}
\item $0 \in I$
\item $I$ is closed under addition: $x, y \in I \Rightarrow x+y \in I$
\item If $x \in \K[x_1, x_2, \ldots, x_d]$ and $y \in I$, then $x\cdot y \in I$ and $y\cdot x \in  I$.
\end{itemize}
\end{Definition}

An ideal is generated by its {\it basis} or {\it generators}.

%%%%%%%%%%%%%%%%%%%%%  ideal basis%%%%%%%%%%%%%%%%%%%%%%%%%%%%%%%%%%
\begin{Definition}
Let $f_1, f_2, \ldots, f_s$ be 
polynomials of the ring $\K[x_1, x_2, \ldots, x_d]$. 
Let $I$ be an ideal generated by $f_1, f_2, \ldots, f_s$. Then:
\begin{equation}
I = \langle f_1,\dots,f_s \rangle=\{h_1 f_1 + h_2 f_2 + \ldots + h_s f_s : h_1,\dots,h_s\in\K[x_1, \dots, x_d]\} \nonumber
\end{equation}
then, $f_1, \ldots, f_s$ are called the {\bf basis (or generators)} of the 
ideal $I$ and correspondingly $I$ is denoted as $I = \langle f_1, f_2, 
\ldots, f_s \rangle$. 
\end{Definition}

\begin{Example}
The set of even integers, which is a subset of the ring of integers
$Z$, forms an ideal of $Z$. This can be seen from the following;
\begin{itemize}
\item $0$ belongs to the set of even integers.
\item The sum of two even integers $x$ and $y$ is always an even
  integer.
\item The product of any integer $x$ with an even integer $y$ is
  always an even integer.
\end{itemize}
\end{Example}

\begin{Example}
Given $\mathbb{R}\left[x,y\right]$, $I = \langle x, y \rangle$ is an 
ideal containing all polynomials generated by $x$ and $y$, 
such as $x^2+y$ and $x+x\cdot y$.
$J = \langle x^2, y^2 \rangle$ is an ideal containing all polynomials 
generated by $x^2$ and $y^2$, such as $x^2+y^3$ and $x^{10}+x^2\cdot y^2$. 
Notice that $J\subset I$ because every polynomial generated by $J$ can
be generated by $I$. 
But $I\neq J$ because $x+y$ can only be generated by $I$.
\end{Example}

The same ideal may have many different bases.
For instance, it is possible to have different sets of polynomials
$\{f_1,\dots,f_{s}\}$ and $\{g_{1},\dots,g_{t}\}$ that may generate the same 
ideal, i.e., 
$\langle f_{1},\dots,f_{s}\rangle=\langle g_{1},\dots,g_{t}\rangle$. Since 
variety depends on the ideal, these sets of polynomials have the same 
solutions.

\begin{Proposition}
If $f_1,\dots,f_{s}$ and $g_{1},\dots,g_{t}$ are bases of the same ideal 
in $\mathbb{F}[x_{1},\dots,x_{d}]$,
so that $\langle f_{1},\dots,f_{s}\rangle=\langle g_{1},\dots,g_{t}\rangle$, 
then $V(f_{1},\dots,f_{s})=V(g_{1},\dots,g_{t})$.
\end{Proposition}

\begin{Example}
	Consider the two bases $F_{1}=\{(2x^{2}+3y^{2}-11,x^{2}-y^{2}-3\}$ and $F_{2}=\{x^{2}-4,y^{2}-1\}$.
	These two bases generate the same ideal, i.e., $\langle F_{1}\rangle= \langle F_{2} \rangle$.{}
	Therefore, they represent the same variety, i.e., 
	\begin{equation}
		V(F_{1})= V( F_{2})=\{\pm 2, \pm 1\}.
	\end{equation}
\end{Example}

Ideals and their varieties are a key part of computer-algebra based formal
verification. A given hardware design can be transformed into a set
of polynomials over a field, $f_1, \ldots, f_s \in F$ 
(we showed how this is done for Galois field arithmetic 
circuits in the previous chapter).
This set of polynomials gives the system of equations:
\begin{eqnarray}
f_1 = 0 \nonumber \\
\vdots \nonumber \\
f_s = 0 \nonumber
\end{eqnarray}
Using algebra, it is possible to derive new equations from the original 
system.
The ideal $\langle f_1,\ldots, f_s \rangle$ provides a way of analyzing 
such {\it consequences} of a system of polynomials.

\begin{Example}
Given two equations in $\mathbb{R}[x,y,z]$:
\begin{eqnarray}
x=z+1 \nonumber \\
y=x^2+1 \nonumber 
\end{eqnarray}
we can eliminate $x$ to obtain a new equation:
\begin{equation}
y=(z+1)^2+1=z^2+2z+2 \nonumber 
\end{equation}
Let $f_1, f_2, h \in \mathbb{R}[x,y,z]$ be polynomials based on these 
equations:
\begin{eqnarray}
f_1 = x-z-1 &= 0 \nonumber \\
f_2 = y-x^2-1 &= 0 \nonumber \\
h   = y-z^2-2z-2 &= 0 \nonumber
\end{eqnarray}
If $I$ is the ideal generated by $f_1$ and $f_2$, i.e. 
$I=\langle f_1, f_2 \rangle$, then we find $h \in I$ as follows:
\begin{eqnarray}
g_1 = x+z+1 \nonumber \\
g_2 = 1     \nonumber \\
h = g_1\cdot f_1+g_2\cdot f_2  = y-z^2-2z-2 \nonumber
\end{eqnarray}
where $g_1, g_2 \in \mathbb{R}[x,y,z]$.
Thus, we call $h$ a {\bf member of the ideal} $I$.
\end{Example}

%\subsection{Ideals of Varieties}

Let $\mathbb{K}$ be any field and let $\mathbf{a}=(a_{1},\dots,a_{d}) \in \mathbb{K}^d$ be a point, and $f \in
\mathbb{K}[x_1,\dots, x_d]$ be a polynomial. We say that $f$ {\it vanishes} on $\mathbf{a}$ if $f(\mathbf{a}) = 0$, i.e.,
$\mathbf{a}$ is in the variety of $f$.

\begin{Definition}
For any variety $V$ of $\mathbb{K}^d$, the ideal of polynomials that vanish on $V$,
called the {\it vanishing ideal of $V$}, is defined as $I(V) = \{f\in
\mathbb{F}[x_1,\dots, x_d]: \forall \mathbf{a} \in V, f(\mathbf{a}) =
0\}$. 
\end{Definition}

\begin{Proposition}\label{pro:iofv}
	If a polynomial $f$ vanishes on a variety $V$, then $f \in I(V)$. 
\end{Proposition}

\begin{Example}
	Let ideal $J=\langle x^{2},y^{2}\rangle$. Then $V(J)=\{(0,0)\}$.
	All polynomials in $J$ will obviously agree with the solution and vanish on this variety.
	However, the polynomials $x,y$ are not in $J$ but they also vanish on this variety. 
	Therefore, $I(V(J))$ is the set of all polynomials that vanish on $V(J)$, and the polynomials
	$x,y$ are members of $I(V(J))$.
\end{Example}

\begin{Definition}\label{def:radical}
Let $J \subset \mathbb{K}[x_1,\dots, x_d]$ be an ideal. The {\it radical of $J$} is defined as $\sqrt{J} = \{f \in
\mathbb{K}[x_1,\dots, x_d]: \exists m \in \mathbb{N}, f^m \in J\}$. 
\end{Definition}

\begin{Example}
Let $J=\langle x^2,y^2\rangle \subset \mathbb{K}\left[x,y\right]$.
Note neither $x$ nor $y$ belongs to $J$, but they belong to $\sqrt J$.
Similarly, $x\cdot y \notin J$, but since $(x \cdot y)^{2}=x^{2}\cdot y^{2}\in J$, therefore,
$x\cdot y \in \sqrt J$. 
\end{Example} 

When $J = \sqrt J$, then $J$ is said to be a 
{\it radical ideal}. Moreover, $I(V)$ is a radical ideal.
By analyzing the ideal $J$, generated by a system of polynomials derived 
from a hardware design, its variety $V(J)$, and the ideal of 
polynomials that vanish over this variety, $I(V(J))$, we can reason about the 
existence of certain properties of the design. To check for the existence of 
a property, we formulate the 
property as a polynomial and then perform an {\bf ideal membership test} to 
determine if this polynomial is contained within the ideal $I(V(J))$. 
A {\bf \Grobner basis} provides a decision procedure for performing this 
test, which is described in the following section. 
A future section focuses on {\bf Hilbert's Nullstellensatz}, 
which describes the properties of the ideal of a variety, $I(V(J))$. 

%%%%%%%%%%%%%%%%%%%%	Grobner bases	%%%%%%%%%%%%%%%%%%%%%%%%%%%%%%%%%%
\section{\Grobner Bases}

As mentioned earlier, different polynomial sets may generate the same 
ideal. Some of these generating sets may be a better representation of 
the ideal, and thus provide more information and insight into the properties 
of ideal. One such ideal representation is a {\bf \Grobner basis}, which has
a number of important properties that can solve numerous polynomial 
decision questions:

\begin{itemize}
\item Presence or absence of solutions (varieties)
\item Dimension of the varieties
\item Ideal membership of a polynomial
\end{itemize} 

In essence, a \Grobner basis is a canonical representation of an ideal.
There are many equivalent definitions of \Grobner bases, so we start with 
the definition that best describes their properties:

\begin{Definition}
A set of non-zero polynomials $G=\{g_1,\dots,g_t\}$ which generate the 
ideal $I=\langle g_1,\dots,g_t\rangle$, is called a 
{\bf Gr\"obner basis} for $I$ if and only if 
for all $f \in I$ where $f \neq 0$, there exists a $g_i \in G$ such that $lm(g_i)$ divides $lm(f)$.
\begin{eqnarray}
G = \text{Gr\"obner{Basis}} (I) \iff 
\forall f \in I: f \neq 0, \exists g_i \in G: lm(g_i)\ |\ lm(f)
        \label{eqn:groebnermin}
    \end{eqnarray}
    
\end{Definition}

The foundation for computing the \Grobner basis of an ideal was laid out 
by Buchberger\cite{buchberger_thesis}.
Given a set of polynomials $F=\{f_{1},\dots,f_{s}\}$ that generate ideal $I=
\langle f_{1},\dots,f_{s} \rangle$, 
Buchberger gives an algorithm to compute a Gr\"obner basis $G=\langle g_{1},
\dots,g_{t}\rangle$. This algorithm relies on the notions of $S$-polynomials 
and polynomial reduction.

\begin{Definition}
For $f, g \in \K[x_1,\dots,x_d]$, an {\bf S-polynomial} $Spoly(f,g)$ is 
defined as:
\begin{equation}
    Spoly(f,g)=\frac{L}{lt(f)}\cdot f - \frac{L}{lt(g)}\cdot g
    \label{eqn:spoly}
\end{equation}
\begin{equation}
\text{where }L = lcm\left(lt(f), lt(g)\right) \nonumber
\end{equation}
Note $lcm$ denotes least common multiple.
\end{Definition}

\begin{Definition}
    The {\bf reduction} of a polynomial $f$, by another polynomial $g$, to
    a reduced polynomial $r$ is denoted:
    \begin{equation*}
        f\stackrel{g}{\textstyle\longrightarrow}r
    \end{equation*}
    Reduction is carried out using multivariate, polynomial long division. 
    % The long division is performed according to a term-ordering on polynomials, and the division algorithm 
    %terminates when the leading term of the divisor does not divide any 
    %other term in the dividend.
  
    For sets of polynomials, the notation 
    \begin{equation*}
    f\stackrel{F}{\textstyle\longrightarrow}_+r    
    \end{equation*}
    represents the reduced polynomial $r$ resulting from $f$ as reduced by a 
    set of non-zero polynomials $F = \{f_1,\dots,f_s\}$.  The polynomial $r$ is considered {\bf reduced} if 
    $r = 0$  or no term in $r$ is divisible  by a $lm(f_i), \forall f_i \in F$.
\end{Definition}

The reduction process $f\stackrel{F}{\textstyle\longrightarrow}_+r$, of 
dividing a polynomial $f$ by a set of polynomials of $F$, can be modeled as
repeated long-division of $f$ by each of the polynomials in $F$ until no
further reductions can be made. The result of this process is then $r$.
This reduction process is shown in Algorithm \ref{alg:polydiv}.

\begin{algorithm}[H]
\SetAlgoNoLine

 \KwIn{$f,f_{1},\dots,f_{s}$}
 \KwOut{$r,a_{1},\dots,a_{s}$, such that $f=a_{1}\cdot f_{1}+\dots+a_{s}\cdot f_{s}+r$.}
 
 $a_{1}=a_{2}=\dots=a_{s}=0$; $r=0$\;
 $p:=f$\;
 
 \While { $p \neq 0$ }
 {
	i=1\;
	divisionmark = false\;
	\While { $i\le s  $ \&\& divisionmark = false }
	{
		\eIf {$f_{i}$ can divide $p$}
		{
			$a_{i}=a_{i}+lt(p)/lt(f_{i})$\;
			$p=p-lt(p)/lt(f_{i}) \cdot f_{i}$\;
			divisionmark = true\;
		}
		{
			i=i+1\;
		}
	}
	
	\If {divisionmark = false}
	{
		$r=r+lt(p)$\;
		$p=p-lt(p)$\;
	}

 }
\caption{Polynomial Reduction}\label{alg:polydiv}
\end{algorithm}

The reduction algorithm keeps canceling the leading terms of polynomials 
until no more leading terms can be further canceled.
So the key step is $p=p-lt(p)/lt(f_{i}) \cdot f_{i}$, as the following 
example shows.
\begin{Example}
Given $f = y^{2}-x$ and $f_{1} = y - x$ in $\mathbb{Q}[x,y]$ with $deglex$: 
$y>x$, perform $f\stackrel{f_1}{\textstyle\longrightarrow}_+r$:

\begin{enumerate}
\item $f=y^{2}-x$, $f/f_{1}=f-lt(f)/lt(f_{1}) \cdot f_{1}=y^{2}-x-(y^{2} /y) \cdot (y-x)=y\cdot x-x$
\item $f=y\cdot x-x$, $f/f_{1}=f-lt(f)/lt(f_{1}) \cdot f_{1}=(y\cdot x-x)/f_{1}=x^{2}-x$
\item $f=x^{2}-x$, no more operations possible, so $r=x^{2}-x$
\end{enumerate}
\end{Example}

With the notions of $S$-polynomials and polynomial reduction in place,
we can now present Buchberger's Algorithm 
for computing Gr\"obner bases \cite{buchberger_thesis}. Note that a fixed 
monomial (term) ordering is required for a \Grobner basis 
computation to ensure that polynomials are manipulated in a consistent 
manner.

\begin{algorithm}[H]
\SetAlgoNoLine
 \KwIn{$F = \{f_1, \dots, f_s\}$, such that $I=\langle f_1, \dots, f_s\rangle$}, and term order $>$
 \KwOut{$G = \{g_1,\dots ,g_t\}$, a Gr\"{o}bner basis of $I$ }
  $G:= F$\;
  \Repeat{$G = G'$}
  {
  	$G' := G$\;
  	\For{ each pair $\{f_{i}, f_{j}\}, i \neq j$ in $G'$} 
	{
		$Spoly(f_{i}, f_{j}) \stackrel{G'}{\textstyle\longrightarrow}_+r$ \;
		\If{$r \neq 0$}
		{
			$G:= G \cup \{r\}$ \;
		}
	}
   }
\caption {Buchberger's Algorithm}\label{alg:gb}
\end{algorithm}

Buchberger's algorithm takes pairs of polynomials ($f_{i}, f_{j}$) in 
the basis $G$ and combines them into ``$S$-polynomials'' 
($Spoly(f_{i}, f_{j})$) to cancel leading terms. The $S$-polynomial is then 
reduced (divided) by all elements of $G$ to a remainder $r$, denoted as  
$Spoly(f_{i}, f_{j}) \stackrel{G}{\textstyle\longrightarrow}_+r$. This
process is repeated for all unique pairs of polynomials, including
those created by newly added elements, until no new polynomials are
generated; ultimately constructing the \Grobner basis.
\begin{Example}\label{exp:gbsimple}
Consider the ideal $I \subset \mathbb{Q}[x, y]$, $I = \langle f_1, f_2 
\rangle$, where $f_1 = yx - y, ~f_2 = y^2 - x$. 
Assume a degree-lexicographic term ordering with $y > x$ is imposed. 

First, we need to compute $Spoly(f_{1},f_{2})=x\cdot f_{2}-y\cdot f_{1}=y^{2}-x^{2}$.
Then we conduct a polynomial reduction 
$y^{2}-x^{2}\stackrel{f_{2}}{\textstyle\longrightarrow}x^{2}-x \stackrel{f_{1}}{\textstyle\longrightarrow}x^{2}-x$.
Let $f_{3}=x^{2}-x$. Then $G$ is updated as $\{f_{1},f_{2},f_{3}\}$. Next we compute $Spoly(f_{1},f_{3})=0$. So there
is no new polynomial generated. Similarly, we compute $Spoly(f_{2},f_{3})=x\cdot y^{2}-x^{3}$, followed by 
$x\cdot y^{2}-x^{3}\stackrel{f_{1}}{\textstyle\longrightarrow}y^{2}-x^{3} \stackrel{f_{2}}{\textstyle\longrightarrow}x-x^{3}
\stackrel{f_{2}}{\textstyle\longrightarrow}0$. Again, no polynomial is generated. Finally, $G=\{f_{1,}f_{2},f_{3}\}$.

\end{Example}

When computing a \Grobner basis, it's important to note that if $lt(f_i)$ 
and $lt(f_j)$ have no common variables, the S-poly reduction step in 
Buchberger's algorithm,
$Spoly(f_{i}, f_{j}) \stackrel{G'}{\textstyle\longrightarrow}_+r$,
will produce $r=0$.

\begin{Proof}
If $lt(f)$ and $lt(g)$ have no common variables,  
$L=lcm(lt(f),lt(g))=lt(f)\cdot lt(g)$. Then: 
\begin{equation}
    Spoly(f,g)=\frac{L}{lt(f)}\cdot f - \frac{L}{lt(g)}\cdot g=
\frac{lt(f)\cdot lt(g)}{lt(f)}\cdot f - \frac{lt(f)\cdot lt(g)}{lt(g)}\cdot g
= lt(g)\cdot f - lt(f)\cdot g \nonumber
\end{equation}
Thus, every monomial in $Spoly(f, g)$ is divisible by either $lt(f)$ 
or $lt(g)$, so computing 
$Spoly(f, g) \stackrel{f,g}{\textstyle\longrightarrow}_+r$ will give $r=0$.
\end{Proof}

As mentioned previously, a \Grobner basis gives a decision procedure to test 
for polynomial membership in an ideal. This is explained in the following 
Theorem.
    
\begin{Theorem}\label{the:membership}
	{\bf Ideal Membership Test}
 Let $G = \{g_1,\cdots,g_t \}$ be a Gr\"obner basis for an ideal $I \subset \mathbb{K}[x_1,\cdots,x_d ]$
	and let $f \in \mathbb{K}[x_{1},\dots, x_{d}]$. Then $f \in I$ if and only if the remainder on division of $f$ by
	$G$ is zero.
\end{Theorem}
In other words, 
\begin{equation}
f \in I \iff f \stackrel{G}{\textstyle\longrightarrow}_+0
\end{equation}

\begin{Example}
Consider Example \ref{exp:gbsimple}. Let $f = y^2x - x$ be another
polynomial. Note that $f = yf_1 + f_2$, so $f \in I$. If we divide $f$
by $f_1$ first and then by $f_2$, we will obtain a zero
remainder. However, since the set $\{f_1, f_2\}$ is not a Gr\"{o}bner
basis, we find that the reduction $f
\stackrel{f_2}{\textstyle\longrightarrow} x^2 - x
\stackrel{f_1}{\textstyle\longrightarrow} x^2 - x  \neq 0$;
i.e. dividing $f$ by $f_2$ first and then by $f_1$ does not lead to a
zero remainder. However,  if we compute the Gr\"{o}bner basis $G$ of
$I$, $G = \{x^2 - x, yx - y, y^2 - x\}$, dividing $f$ by polynomials
in $G$ in any order will always lead to the zero remainder. Therefore,
one can decide ideal membership unequivocally using the Gr\"{o}bner
basis. 
\end{Example}

A \Grobner basis is not a canonical representation of an ideal, but a
{\bf reduced \Grobner basis} is. To compute a reduced \Grobner basis, we
first must compute a minimal \Grobner basis.

\begin{Definition}\label{def:minigb}
A {\bf minimal Gr\"obner basis} for a polynomial ideal $I$ is a \Grobner basis $G$ for $I$ such that
	\begin{itemize}
		\item $lc(g_{i})=1,\forall g_{i}\in G$
		\item $\forall g_{i} \in G$,  $lt(g_{i}) \notin \langle lt(G-\{g_{i}\})\rangle$
	\end{itemize}
\end{Definition}
A {\bf minimal} \Grobner basis is a \Grobner basis such that all polynomials
have a coefficient of $1$ and no leading term of any element in $G$ divides 
another in $G$.
Given a \Grobner basis $G$, a minimal \Grobner basis can be
computed as follows:
\begin{enumerate}
\item Minimize every $g_i \in G$, i.e $g_i=g_i/lc(g_i)$
\item For $g_i, g_j \in G$ where $i\neq j$, remove $g_i$ from $G$ if $lt(g_i)\mid lt(g_j)$, i.e. remove every polynomial in $G$ whose leading term is divisible by the leading term of some other polynomial in $G$.
\end{enumerate}

A minimal Gr\"obner basis can then be further reduced.
\begin{Definition}
	A {\bf reduced Gr\"obner basis} for a polynomial ideal $I$ is a Gr\"obner basis $G=\{g_{1},\dots,g_{t}\}$ such that:
	\begin{itemize}
		\item $lc(g_{i})=1,\forall g_{i}\in G$
		\item $\forall g_{i} \in G$, no monomial of $g_{i}$ lies in $\langle lt(G-\{g_{i}\})\rangle$
	\end{itemize}
\end{Definition}
$G$ is a reduced Gr\"obner basis when no monomial of any element in $G$ 
divides the leading term of another element. 
This reduction is achieved as follows:

\begin{Definition}
Let $H = \{h_1, \ldots, h_t\}$ be a minimal Gr\"obner basis.  Apply
the following reduction process: 
\begin{itemize}
\item $h_1 \stackrel{G_1}{\textstyle\longrightarrow}_+ g_1$, where
  $g_1$ is reduced w.r.t. $G_1 = \{h_2, \ldots, h_t\}$

\item $h_2 \stackrel{G_2}{\textstyle\longrightarrow}_+ g_2$, where
  $g_2$ is reduced w.r.t. $G_2 = \{g_1, h_3, \ldots, h_t\}$
\item $h_3 \stackrel{G_3}{\textstyle\longrightarrow}_+ g_3$, where
  $g_3$ is reduced w.r.t. $G_3 = \{g_1, g_2, h_4, \ldots, h_t\}$

\hspace{0.25in} $\vdots$
\vspace{0.1in}
\item $h_t \stackrel{G_t}{\textstyle\longrightarrow}_+ g_t$, where
  $g_t$ is reduced w.r.t. $G_t = \{g_1, g_2, g_3, \ldots, g_{t-1}\}$
\end{itemize}
Then $G = \{g_1, \ldots, g_t\}$ is a {\bf reduced Gr\"obner basis.}
\end{Definition}


Subject to the given term order $>$, such a reduced Gr\"obner
  basis $G = \{g_1, \dots, g_t\}$ is a {\bf unique canonical
    representation of the ideal}, as 
given by Proposition \ref{pro:unique} below.


\begin{Proposition}\label{pro:unique} \cite{gb_book} 
Let $I \neq \{0\}$ be a polynomial ideal. Then, for a given monomial ordering, $I$ has a unique reduced Gr\"obner basis.
\end{Proposition}

\Grobner basis computation depends on the $Spoly$ computation, which in turn 
depends on the leading terms of polynomials. Thus, different monomial 
orderings can result in different \Grobner basis computations for the 
same ideal. Computation using a degrevlex ordering tends to be least 
difficult, while lex ordering tends to be computationally complex. However, 
lex ordering used in the computation of \Grobner basis is an {\bf elimination
ordering}; that is, the polynomials contained in the resulting \Grobner basis
have continuously eliminated variables in the ordering. This is the topic of 
elimination theory, which is described in the following section.

\section{Elimination Theory}

Elimination theory uses {\bf elimination ordering} 
to systematically eliminate variables from a system of polynomial
equations.

\begin{Definition}
Let $I$ be an ideal in  $\K[x_1,\dots,x_k]$. The $i$-th 
{\bf elimination ideal} $I_i$ is the ideal of $\K[x_{i+1},\dots,x_k]$ defined
by
\begin{equation}
I_k = I \cap \K[x_{i+1},\dots,x_k]
\end{equation}
\end{Definition}

The elimination ideal $I_i$ has eliminated all the variables 
$x_1,\dots,x_i$, i.e. it only contains polynomials with variables in
$x_{i+1},\dots,x_k$. 
We can generate elimination ideals by computing
\Grobner bases using elimination orderings. 

\begin{Theorem}
$\left[\bf{Elimination\  Theorem}\right]$
Let $I$ be an ideal in $\K[x_1,\dots,x_k]$ and let $G$ be the \Grobner 
basis of $I$ with respect to the lex order (elimination order) 
$x_1>x_2>\dots>x_k$. Then, for every $0\leq i\leq k$,
\begin{equation}
G_k=G\cap \K[x_{i+1},\dots,x_k]
\end{equation}
is a \Grobner basis of the $i$-th elimination ideal $I_i$.
\label{thm:elimth}
\end{Theorem}

This can be better visualized using the following example.

\begin{Example}
Given the following equations in $\R[x,y,z]$
\begin{eqnarray}
x^2+y+z&=1 \nonumber \\
x+y^2+z&=1 \nonumber \\
x+y+z^2&=1 \nonumber
\end{eqnarray}
let $I$ be the ideal generated by these equations:
\begin{equation}
I=\langle x^2+y+z-1, x+y^2+z-1, x+y+z^2-1\rangle \nonumber
\end{equation}
The \Grobner basis for $I$ with respect to lex order $x>y>z$ is 
found to be $G=\{g_1,g_2,g_3,g_4\}$ where
\begin{eqnarray}
g_1&=&x+y+z^2-1 \nonumber \\
g_2&=&y^2-y-z^2+z \nonumber \\
g_3&=&2yz^2+z^4-z^2 \nonumber \\
g_4&=&z^6-4z^4+4z^3-z^2 \nonumber
\end{eqnarray}

Notice that while $g_1$ has variables in $\R[x,y,z]$, $g_2$ and $g_3$ only 
have variables in $\R[y,z]$ and $g_4$ only has variables in $\R[z]$. Thus, 
$G_1=G\cap \R[y,z]=\{g_2,g_3,g_4\}$ and $G_2=G\cap \R[z]=\{g_4\}$

Also notice that since $g_4$ only contains variable $z$, and since $g_4=0$, 
a solution for $z$ can be obtained. This solution can then be applied to 
$g_2$ and $g_3$ to obtain solutions for $y$, and so on.
\end{Example}

Elimination theory provides the basis for our abstraction approach.


%%%%%%%%%%%%%%%%%%%%%%%%%%%%%%%%%%%%%%%%%%%%%%%%%%%%%%%%

\section{Hilbert's Nullstellensatz}

In this section, we further describe some correspondence between ideals and 
varieties in the context of algebraic geometry. The celebrated results of 
Hilbert's Nullstellensatz establish these correspondences.

%%%%%%%%%%%%%%%%%algebraically closed field%%%%%%%%%%%%%
\begin{Definition}\label{def:acf}
A field $\overline {\K}$ is an {\bf algebraically closed} field if every  
polynomial in one variable with degree at least $1$, with coefficients 
in $\overline {\K}$, has a root in $\overline {\K}$. 
\end{Definition}
In other words, any non-constant polynomial equation over 
$\overline {\K}\left[x\right]$ always has at least one root 
in $\overline {\K}$. Every field $\K$ is contained in an algebraically 
closed one $\overline {\K}$. 
For example, the field of real numbers $\mathbb{R}$ is not an algebraically closed 
field, because $x^2+1=0$ has no root in $\mathbb{R}$. 
However, $x^2+1=0$ has roots in the field of 
complex numbers $\mathbb{C}$, which is an algebraically closed field. 
In fact, $\mathbb{C}$ is the algebra closure of $\mathbb{R}$. 
Every algebraically closed field is an infinite field. 

%%%%%%%%%%%%%%%%weak nullstellensatz%%%%%%%%%%%%%
%\begin{Theorem}
%$\left[\bf{Weak\  Nullstellensatz}\right]$ Let 
%$I \subset \overline {\mathbb{K}}[x_1, x_2, \cdots, x_d]$ 
%be an ideal satisfying $V(I)=\emptyset$. 
%Then $I=\overline {\mathbb{K}}[x_1, x_2, \cdots, x_d]$, Or equivalently, 
%\begin{equation}
%V(I)=\emptyset\ \iff\ I=\overline {\mathbb{K}}[x_1, x_2, \cdots, x_d]=\langle 1 \rangle 
%\end{equation}
%\end{Theorem}
%
%\begin{Corollary}
%	Let $I=\langle f_{1},\dots,f_{s} \rangle \subset \overline {\mathbb{K}}[x_1, x_2, \cdots, x_d]$. 
%	Let $G$ be the reduced Gr\"obner basis of $I$. Then $V(I)=0 \iff G=\{1\}$.
%\end{Corollary}

%{\bf Weak Nullstellensatz} offers a way to evaluate whether or not the 
%system of multivariate polynomial equations (ideal $I$) has common solutions 
%in ${\overline {\mathbb{K}}}^d$. For this purpose, we only need to check if 
%the ideal is generated by the unit element, i.e., $1\in I$. 
%This approach can be used to evaluate the feasibility of constraints in 
%verification problems.
%%%%%%%%%%%%%%%%%%%%%%%%%strong Nullstellensatz%%%%%%%%%%%%%%%%%%%%%%%%%%%%%% 
An interesting result is one of {\bf Strong 
Nullstellensatz}.
The strong
Nullstellensatz establishes the correspondence between radical ideals
and varieties. 

\begin{Theorem}\label{thm:sns}
({\it The Strong Nullstellensatz} \cite{gb_book}) 
Let $\overline{\mathbb{K}}$ be an algebraically closed field, and let $J$
be an ideal in $\overline{\mathbb{K}}[x_1,\dots, x_d]$. 
Then we have $I(V_{\overline{\mathbb{K}}}(J)) =\sqrt{J}$. 
\end{Theorem}

%\subsection{Nullstellensatz over Galois fields}

Strong Nullstellensatz holds a special form over Galois fields $\Fq$.
Recall the notion of vanishing polynomials over Galois fields from the 
previous chapter: for every element $A \in \Fq$, $A-A^q = 0$; then the 
polynomial $x^q-x$ in $\Fq[x]$ vanishes over $\Fq$. Thus, if 
$J_0=\langle x^q-x \rangle$ is the ideal generated by the vanishing 
polynomial, $V(J_0)=\Fq$. Similarly, over $\Fq[x_1,\dots,x_d]$, $J_0$ is 
$\langle x_1^q-x_1,\dots,x_d^q-x_d \rangle$ and $V(J_0)=(F_q)^d$.


\begin{Definition}
Given two ideals, $I_1=\langle f_1, \dots,f_s \rangle$ and 
$I_2=\langle g_1,\dots g_t\rangle$, then the {\bf sum of ideals} 
$I_1+I_2=\langle f_1,\dots,f_s,g_1,\dots g_t\rangle$
\end{Definition}

\begin{Theorem}
({\it Strong Nullstellensatz over $\Fq$})
For any Galois field $\Fq$, let $J \subset \Fq[x_1,\dots,x_d]$ be any ideal 
and let $J_0 = \langle x_1^q-x_1, x_d^q-x_d \rangle$ be the ideal of all
vanishing polynomials. Let $V_{\Fq}(J)$ denote the variety of $J$ over $\Fq$.
Then, $I(V_{\Fq}(J))=J+J_0$.
\end{Theorem}

The proof is given in \cite{gao:gf-gb-ms}. Here, we provide a proof outline.

\begin{Proof}
\begin{enumerate}
\item $\sqrt{J+J_0} = J+J_0$. That is, $J+J_0$ is a radical ideal.
\item $V_{\Fq}(J)=V_{\overline{\Fq}}(J+J_0)$.
\item Due to (2), $I(V_{\Fq}(J)) = I(V_{\overline{\Fq}}(J+J_0))$. 
By Strong Nullstellensatz, this is equivalent to $\sqrt{J+J_0}$.
Finally, due to (1), this is equivalent to $J+J_0$.
\end{enumerate}
\end{Proof}

%% 
%% Using this result, Weak Nullstellensatz can be 
%% modified to be applicable over finite fields $\Fq$.
%% %%%%%%%%%%%%%%%%weak nullstellensatz in finite field%%%%%%%%%%%%%
%% \begin{Theorem}\label{wnull:ff}
%% $[\bf{Weak~Nullstellensatz~in~\Fq}]$\cite{null:1890}\\
%% Given $f_1,f_2,\cdots,f_s \in \Fq[x_1,x_2,\cdots,x_d]$. 
%% Let $J=\langle f_1,f_2,\cdots,f_s\rangle \subset \Fq[x_1,
%% x_2, \cdots, x_d]$ be an ideal. Let $J_0 = \langle 
%% x_1^{2^k}-x_1,x_2^{2^k}-x_2,\cdots,x_d^{2^k}-x_d \rangle$ be the ideal
%% of vanishing polynomials in $\Fq$. Then
%% $V_{\Fq}(J) = V_{\overline {\Fq}}(J +
%% J_0)=\emptyset$,  if and only if the reduced
%% Gr\"obnerBasis$(J+J_{0})=\{1\}$. 
%% \end{Theorem}

%% The proof is given in \cite{null:1890}. Here, we provide a proof outline.

%% \begin{Proof}
%% The variety of $J$ over $\Fq[x_1,x_2,\cdots,x_d]$ 
%% is equivalent to the variety over the algebraic closure of $\Fq$ 
%% intersected by the entire field $\Fq$. That is, $V_{\Fq}(J)=V_{\overline 
%% {\Fq}}(J) \cap \Fq$. 

%% Let $J_0 = \langle 
%% x_1^{2^k}-x_1,x_2^{2^k}-x_2,\cdots,x_d^{2^k}-x_d \rangle$ be the ideal
%% generated by all vanishing polynomials in $\Fq[x_1,x_2,\cdots,x_d]$.
%% Then $V_{\overline{\Fq}}(J_0)=\Fq$. 

%% Thus, $V_{\Fq}(J) = V_{\overline{\Fq}}(J)\cap V_{\overline{\Fq}}(J_0)
%% = V_{\overline{\Fq}}(J+J_0)$.
%% \end{Proof}


%%%%%%%%%%%%%%%%%%%%%%%%%%%%%%%%%%%%%%%%%%%%%
%%%%%%%%%%%%%%%%%%%%%%%%%%%%%%%%%%%%%%%%%%%%%
\section{Concluding Remarks}

Our approach to word-level abstraction of Galois field arithmetic 
circuits applies concepts of polynomial ideals, varieties, \Grobner basis, 
and elimination theory to abstract a word-level representation of the 
circuit. This approach is described in the next chapter. However, a \Grobner
basis computation is prohibitively expensive; thus we propose improvements 
to our original approach in a subsequent chapter.

%Once we have a word-level representation, we can apply these representations
%to perform equivalence checking of Galois field arithmetic circuits.
%The verification problem is formulated using Weak
%Nullstellensatz, as it applies over Galois fields, and subsequently solved 
%using a \Grobner basis approach. This approach is described in a future 
%chapter.
