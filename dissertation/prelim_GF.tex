\chapter{Galois Field and Sequential Arithmetic Circuits}
This chapter provides a mathematical background for understanding 
finite fields (Galois fields) and explains how to design Galois field arithmetic circuits.
We first introduce the mathematical concepts of groups, rings, fields, and 
polynomials. 
We then apply these concepts to create Galois field arithmetic functions and 
explain how to map them to a Boolean circuit implementation.
Additonally, we introduce a special type of sequential arithmetic hardware based on normal basis, as well
as the normal basis theory behind the designing such hardware.

The material is referred from \cite{galois_field:mceliece, ftheory:2006, ff:1997} for Galois field concepts, 
\cite{mastro:1989, PT:1985, acar:1998, wu:2002, Knezevic:2008} for hardware design over Galois fields 
and previous work by {\it Lv} \cite{lv:phd}.
Normal basis theory in this section refers to \cite{normal_book, gao:phd_normal_basis} and sequential
normal basis arithmetic hardware designs come from \cite{mullinONB,MasseyOmura,agnew1991implementation, RHmulti}.

\section{Commutative Algebra}
\label{sec:algebra}
\subsection{Group, ring and field}
\begin{Definition}
An {\bf abelian group} is a set $\mathbb{S}$ with a binary operation $'+'$
which satisfies the following properties: 
\begin{itemize}
\item {\it Closure Law:} For every $a, b \in \mathbb{S}, a + b \in \mathbb{S}$  
\item {\it Associative Law:} For every $a, b, c \in \mathbb{S}, (a + b) + c = a + (b + c)$
\item {\it Commutativity:} For every $a, b \in \mathbb{S}, a + b = b + a$. 
\item {\it Additive Identity:} There is an identity element $0 \in \mathbb{S}$
such that for all $a \in \mathbb{S};$ $a + 0 = a$.
\item {\it Additive Inverse:} If $a \in \mathbb{S}$, then there is an
element $a^{-1} \in \mathbb{S}$ such that $ a + a^{-1} = 0$.
\end{itemize}
\end{Definition}

The set of integers $\mathbb{Z}$ forms an abelian group under the addition operation. 

\begin{Definition}
Given a set $\mathbb{R}$ with two binary operations, $'+'$ and $'\cdot'$, 
and element $0 \in \mathbb{R}$, the system $\mathbb{R}$ is called a {\bf commutative ring with unity} if the following properties hold:
\begin{itemize}
\item $\mathbb{R}$ forms an abelian group under the '+' operation with additive identity element $0$.
\item {\it Multiplicative Distributive Law}: For all $a, b, c \in$ $\mathbb{R}$, $a\cdot (b + c) = a\cdot b + a\cdot c$.
\item {\it Multiplicative Associative Law}: For every $a, b, c\in \mathbb{R}$, $a\cdot (b\cdot c) = (a\cdot b)\cdot c$. 
\item {\it Multiplicative Commutative Law}: For every $a,b \in \mathbb{R}$, $a\cdot b = b\cdot a$
\item {\it Identity Element}: There exists an element $1 \in$ $\mathbb{R}$ 
such that for all $a \in \mathbb{R}$, $a\cdot 1 = a =1\cdot a$
\end{itemize}
\end{Definition}

{\bf Ring} is a broad algebraic concept. In this dissertation, this word is used to refer a special 
sort of ring -- {\bf commutative ring with unity}. Two common 
examples of such rings are the set of integers, $\mathbb{Z}$, and the set of 
rational numbers, $\mathbb{Q}$. Note that while both of these examples are
rings with an infinite number of elements, the number of elements in a ring 
can also be finite.

\begin{Definition}
A {\bf field} $\mathbb{F}$ is a commutative ring with unity, where every
non-zero element in $\mathbb{F}$ has a multiplicative inverse; i.e. $\forall
a \in \mathbb{F} - \{0\}$, $\exists \hat{a} \in \mathbb{F}$ such that $ a \cdot
\hat{a} = 1$.
\end{Definition}

A field is defined as a ring with one extra condition: the presence of a 
multiplicative inverse for all non-zero elements.
Therefore, a field must be a ring while a ring is not necessarily a field.
For example, the set $\mathbb{Z}_{2^k} = \{0,1,\cdots, 2^k-1\}$ forms a finite ring.
However, $\mathbb{Z}_{2^k}$ is not a field because not every element in
$\mathbb{Z}_{2^k}$ has a multiplicative inverse. 
In the ring $\mathbb{Z}_{2^3}$, for 
instance, the element $5$ has an inverse ($5\cdot5\pmod{8}=1$) but the element $4$
does not.

The main concept of field theory is {\bf Field Extensions}. The idea behind a
field extension is to take a base field and construct a larger field which 
contains the base field as well as satisfies additional properties. For example,
the set of real numbers $\mathbb{R}$ forms a field; one common extension of 
$\mathbb{R}$ is the set of complex numbers $\mathbb{C}=\mathbb{R}(i)$. Every
element of $\mathbb{C}$ can be represented as $a+b\cdot i$ where $a,b \in \mathbb{R}$,
hence $\mathbb{C}$ is a two-dimensional extension of $\mathbb{R}$.

Like rings, fields can also contain either an infinite or a finite number of 
elements. 
In this dissertation we focus on finite fields, also known as Galois fields, and 
the construction of their field extensions.

\subsection{Galois Field}
Galois fields, also known as finite fields, find widespread applications in 
many areas of electrical engineering and computer science such as error-
correcting codes, elliptic curve cryptography, digital signal processing, 
testing of VLSI circuits, among others.
In this dissertation, we specifically focus on their application to 
Elliptic Curve Cryptography as Galois field arithmetic circuits.
This section describes the relevant Galois field concepts
\cite{galois_field:mceliece} \cite{ftheory:2006} \cite{ff:1997}
and hardware arithmetic designs over such fields \cite{mastro:1989} \cite{PT:1985} 
\cite{acar:1998} \cite{wu:2002} \cite{Knezevic:2008}. 

%%%%%%%%%%%%%%%%%%%%%%%%%%%%%%%%%%%%%%%%%%%%%%%%%%%%%%%%%%%%%%%%%%%%%%%%%%%%

\begin{Definition} 
A {\bf Galois field}, denote $\Fq$, is a field with a finite
number of elements, $q$. The number of elements $q$ of the Galois field is
a power of a prime integer, i.e. $q = p^k$, where $p$ is a prime
integer, and $k \geq 1$. Thus a Galois field can also be denoted as 
$\F_{p^{k}}$.
\end{Definition}

Fields in the form $\F_{p^{k}}$ are called Galois extension fields.
We are specifically interested in extension fields of type 
$\Fkk$, where $k > 1$. These are extensions of the binary
field $\F_2$.
\begin{Example}
Addition and multiplication operations over $\F_2$:
\begin{table}[!h]
	\centering
	\begin{tabular}{m{1cm}|l|ll|m{1cm}}
	\hhline{~---~}
	\multirow{3}{*}{} & $+$ & $0$ & $1$ & \multirow{3}{*}{} \\
	\hhline{~---~}
	& $0$ & $0$ & $1$ & \\
	& $1$ & $1$ & $0$ & \\
	\hhline{~---~}
	\multicolumn{5}{c}{}\\
	\multicolumn{5}{c}{Addition over $\F_2$}\\
	\end{tabular}
	\quad
	\begin{tabular}{m{1cm}|l|ll|m{1cm}}
	\hhline{~---~}
	\multirow{3}{*}{} & $\cdot$ & $0$ & $1$ & \multirow{3}{*}{} \\
	\hhline{~---~}
	& $0$ & $0$ & $0$ & \\
	& $1$ & $0$ & $1$ & \\
	\hhline{~---~}
	\multicolumn{5}{c}{}\\
	\multicolumn{5}{c}{Multiplication over $\F_2$}\\
	\end{tabular}
\end{table}

Notice that addition over $\F_2$ is a Boolean {\sc XOR} operation, 
because it is performed modulo $2$.
Similarly, multiplication over $\F_2$ performs a Boolean {\sc AND} operation.
\end{Example}

Algebraic extensions of the binary field $\F_{2}$  
are generally termed as {\it binary extension fields} $\Fkk$.
Where elements in $\F_2$ can only represent $1$ bit, elements in $\Fkk$ 
represent a $k$-bit vector.
This allows them to be widely used in digital hardware applications.
In order to construct a Galois field of the form $\Fkk$, 
an {\bf irreducible polynomial} is required:
\begin{Definition}
A polynomial $P(x) \in \mathbb{F}_{2}\left[x\right]$ is {\bf irreducible} 
if $P(x)$ is non-constant with degree $k$ and cannot be 
factored into a product of polynomials of lower degree in $\mathbb{F}_2[x]$.
\end{Definition}

Therefore, the polynomial $P(x)$ with degree $k$ is irreducible over 
$\mathbb{F}_{2}$ if and only if it has no roots in $\mathbb{F}_{2}$,
i.e if $\forall a \in \mathbb{F}_{2}$, $P(a)\neq 0$.
For example, $x^2+x+1$ is an irreducible polynomial over $\mathbb{F}_{2}$
because it has no solutions in $\mathbb{F}_{2}$, i.e. $(0)^2+(0)+1=1\neq0$ 
and $(1)^2+(1)+1=1\neq0$ over $\F_2$.
Irreducible polynomials exist for any degree $\geq 2$ in $\mathbb{F}_2[x]$.

Given an irreducible polynomial $P(x)$ of degree $k$ in the polynomial ring 
$\mathbb{F}_2[x]$, we can construct a binary extension field 
$\mathbb{F}_{2^k} \equiv \mathbb{F}_2[x] \pmod{P(x)}$.
Let $\alpha$ be a root of $P(x)$, i.e., $P(\alpha)=0$.
Since $P(x)$ is irreducible over
$\mathbb{F}_2[x]$, $\alpha \notin \mathbb{F}_2$. 
Instead, $\alpha$ is an element in $\mathbb{F}_{2^k}$. 
Any element $A \in \mathbb{F}_{2^k}$ is then represented as: 
\begin{equation}\label{rep:poly}
A= \sum_{i=0}^{k-1} (a_i \cdot \alpha^i) = a_0 + a_1\cdot\alpha + \cdots + a_{k-1}\cdot \alpha^{k-1}\nonumber
\end{equation}
where $a_i \in \mathbb{F}_2$ are the coefficients and $P(\alpha)=0$.

To better understand this field extension, compare its similarities to another
common-place
field extension $\C$, the set of complex numbers. $\C$ is an extension of the field 
of real numbers $\R$ with an additional element $i=\sqrt{-1}$, which is an imaginary
root in $\R$.
Thus $i \notin \R$, rather $i \in \C$.
Every element $A \in \mathbb{C}$ can be represented as:
\begin{equation}\label{rep:polyC}
A=\sum_{j=0}^{1} (a_j \cdot i^j)=a_0+a_1\cdot i
\end{equation}
where $a_j \in \R$ are coefficients. Similarly, $\Fkk$ is an extension of $\F_2$ with 
an additional element $\alpha$, which is the ``imaginary root'' of an irreducible 
polynomial $P$ in $\F_2[x]$.

Every element $A \in \Fkk$ has a degree less than $k$ because 
$A$ is always computed modulo $P(x)$, which has degree $k$. 
Thus, $A\pmod {P(x)}$ can be of degree at most $k-1$ and at least $0$.
For this reason, the field $\mathbb{F}_{2^k}$ can be viewed as a $k$
dimensional vector space over $\mathbb{F}_{2}$. 
The equivalent bit vector representation for element $A$ is:
\begin{equation}
A=(a_{k-1} a_{k-2} \cdots a_{0})
\end{equation}

\begin{Example}
A 4-bit Boolean vector, $(a_{3} a_{2} a_{1} a_{0})$
can be presented over $\F_{2^4}$ as: 
\begin{equation}
a_3 \cdot \alpha^3+a_2 \cdot \alpha^2+a_1 \cdot \alpha+a_0
\end{equation}
For instance, the Boolean vector $1011$ is represented as the element 
$\alpha^3+\alpha+1$.
\end{Example}

\begin{Example}\label{exp:1}
Let us construct $\mathbb{F}_{2^4}$ as $\mathbb{F}_2[x] \pmod{ P(x)}$, where
$P(x)=x^4+x^3+1 \in \mathbb{F}_2[x]$ is an irreducible polynomial of degree $k=4$. 
Let $\alpha$ be the root of $P(x)$, i.e. $P(\alpha)=0$. 

Any element $A \in \mathbb{F}_2[x] \pmod{ x^4 + x^3 + 1}$
has a representation of the type: $A = a_3 x^3 + a_2 x^2 +
a_1 x + a_0$ (degree $< 4$) where the coefficients $a_3, \dots, a_0$ are in $\F_2 =
\{0, 1\}$. Since there are only $16$ such polynomials, we obtain
$16$ elements in the field $\mathbb{F}_{2^4}$. Each element in
$\mathbb{F}_{2^4}$ can then be viewed as a $4$-bit vector over $\mathbb{F}_{2}$. 
Each element also has an exponential $\alpha$
representation. All three representations are shown in Table
\ref{tab:gfelement}.

\begin{table}[h]
\begin{center}
\caption{Bit-vector, Exponential and Polynomial representation of
elements in  $\mathbb{F}_{2^4} = \mathbb{F}_2[x] \pmod{x^4+x^3+1}$}\label{tab:gfelement} 
\begin{tabular}{|c|c|c||c|c|c|} 
\hline
$a_3a_2a_1a_0$ & Exponential & Polynomial     &$a_3a_2a_1a_0$ & Exponential & Polynomial  \\
\hline
$0000$        & $0$         & $0$            & $1000$ & $\alpha^3$ &  $\alpha^3$\\
\hline
$0001$        & $1$         & $1$            & $1001$ & $\alpha^4$ & $\alpha^3 + 1$\\
\hline
$0010$        & $\alpha$    & $\alpha$       & $1010$ & $\alpha^{10}$&$\alpha^3 + \alpha$  \\
\hline
$0011$        & $\alpha^{12}$& $\alpha + 1$   & $1011$ & $\alpha^5$ & $\alpha^3+\alpha+1$\\
\hline
$0100$        & $\alpha^2$  & $\alpha^2$     &  $1100$ & $\alpha^{14}$ & $\alpha^3 + \alpha^2$\\
\hline
$0101$        & $\alpha^9$   &$\alpha^2 + 1$ & $1101$  &$\alpha^{11}$  & $\alpha^3+\alpha^2+1$\\
\hline
$0110$        & $\alpha^{13}$& $\alpha^2 + \alpha$ & $1110$ & $\alpha^8$& $\alpha^3+\alpha^2+\alpha$\\
\hline
$0111$        &$\alpha^7 $ & $\alpha^2+\alpha+1$ & $1111$ &$\alpha^6$ & $\alpha^3+\alpha^2+\alpha+1$\\
\hline
\end{tabular}
\end{center}
\end{table}

We can compute the polynomial representation from the exponential representation.
Since every element is computed $\pmod{P(\alpha)} = \pmod{\alpha^4+\alpha^3+1}$, 
we compute the element $\alpha^{4}$ as 
\begin{equation}
\alpha^{4} \pmod{ \alpha^4+\alpha^3+1} = -\alpha^3 - 1 = \alpha^3+1
\end{equation}
Recall that all coefficients of $\F_{2^4}$ 
are in $\F_{2}$ where $-1 = +1$ modulo 2.
The next element $\alpha^{5}$ can be computed as 
\begin{equation}
\alpha^{5} = \alpha^{4}\cdot \alpha = (\alpha^3+1)\cdot \alpha = \alpha^4+\alpha = \alpha^3+\alpha+1 
\end{equation}
Then $\alpha^6$ can be computed as $\alpha^{5}*\alpha$ and so on.
\end{Example}

An irreducible polynomial can also be a primitive polynomial.

\begin{Definition}
A {\bf primitive polynomial} $P(x)$ is a polynomial with coefficients in $\mathbb{F}_2$ 
which has a root $\alpha$ $\in$ $\mathbb{F}_{2^k}$
such that \{$0$, $1(=\alpha^{{2^k}-1})$, $\alpha$, $\alpha^2$, $\cdots$, $\alpha^{2^k-2}$\} is the set of 
all elements in $\mathbb{F}_{2^k}$, 
where $\alpha$ is a {\bf primitive element} of $\mathbb{F}_{2^k}$. 
\end{Definition}

A primitive polynomial is guaranteed to generate all distinct elements 
of a finite field $\mathbb{F}_{2^k}$ while an irreducible polynomial
has no such guarantee.
Often, there exists more than one irreducible polynomial of degree $k$.
In such cases, any degree $k$ irreducible polynomial can be 
used for field construction. For example, both $x^3+x+1$ and $x^3+x^2+1$ 
are irreducible in $\mathbb{F}_2$ and either one can be used
to construct $\mathbb{F}_{2^3}$. This is due to the following:

\begin{Theorem}\label{the:unique}
There exist a {\bf unique} field $\mathbb{F}_{p^k}$, for any prime $p$ and any positive integer $k$.
\end{Theorem}

Theorem \ref{the:unique} implies that Galois fields with the same number of elements are 
{\bf isomorphic} to each other up to the labeling of the elements. 

Theorem \ref{the:fer} provides an important property for investigating solutions to
polynomial equations in $\Fq$.

\begin{Theorem}\label{the:fer}
 $\left[Generalized\  Fermat's\  Little\  Theorem \right]$ Given a
 Galois field $\mathbb{F}_{q}$, each element $A \in \mathbb{F}_{q}$ satisfies: 
\begin{eqnarray}\label{fe}
 A^{q} & \equiv & A  \nonumber \\
 A^{q} - A & \equiv& 0  
\end{eqnarray}
\end{Theorem} 

We can extend Theorem \ref{the:fer} to polynomials in $\mathbb{F}_{q}[x]$ as 
follows: 
\begin{Definition}
Let $x^q-x$ be a polynomial in $\mathbb{F}_{q}[x]$.
Every element $A \in \mathbb{F}_{q}$ is a solution to  $x^q-x=0$. 
Therefore, $x^{q} - x$ always {\it vanishes} in $\mathbb{F}_{q}$. Such 
polynomials are called {\bf vanishing polynomials} of the field $\mathbb{F}_{q}$.
\end{Definition}

\begin{Example}
Given $\mathbb{F}_{2^2} =\{0,1,\alpha,\alpha+1\}$ with $P(x)=x^2+x+1$, where $P(\alpha)=0$. 
 \begin{eqnarray}
 0^{2^2}&=&0 \nonumber \\
 1^{2^2}&=&1 \nonumber \\
 \alpha^{2^2}&=&\alpha \pmod {\alpha^2+\alpha+1}\nonumber \\
 (\alpha+1)^{2^2}&=&\alpha+1 \pmod {\alpha^2+\alpha+1} \nonumber 
 \end{eqnarray}
\end{Example}

A Galois field $\F_q$ can be fully contained within a larger field $\F_{q^k}$.
That is, $\F_q \subset \F_{q^k}$.
For example, Fig \ref{fig:contain2_4_16} shows the containment of the fields 
$\F_2 \subset \F_4 \subset \F_{16}$. It's easy to see that since $\F_4=\F_{2^2}$, it
contains $\F_2$. Likewise $\F_{16}=\F_{4^2}=\F_{2^4}$ contains $\F_4$ and $\F_2$.
The elements $\{0,1,\alpha,\dots,\alpha^{14}\}$
designate $\F_{16}$. Of these, $\{0,1,\alpha^5,\alpha^{10}\}$ create $\F_4$.
From these, only $\{0,1\}$ exist in $\F_2$.

\begin{Theorem}
$\F_{2^n}\subset\F_{2^m}$ iff $n \mid m$, i.e. if $n$ divides $m$.
\end{Theorem}

Therefore:
\begin{itemize}
\item $\F_2 \subset \F_{2^2} \subset \F_{2^4} \subset \F_{2^8} \subset \dots$
\item $\F_2 \subset \F_{2^3} \subset \F_{2^9} \subset \F_{2^{27}} \subset \dots$
\item $\F_2 \subset \F_{2^5} \subset \F_{2^{25}} \subset \F_{2^{125}} \subset \dots,$ and so on
\end{itemize}

\begin{Definition}
The {\bf algebraic closure} of the Galois field $\F_{2^k}$, denoted $\overline{\Fkk}$, is the 
union of all fields $\F_{2^n}$ such that $k \mid n$.
\end{Definition}