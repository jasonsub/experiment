% LaTeX file for a 1 page document
\documentclass[12pt]{article}
%\usepackage{e-jc}
\usepackage{amsfonts}

\title{Theory about Normal Basis on $F(2^k)$}

\author{Xiaojun Sun \ \ \ \ Prof. Priyank Kalla\\
\small Department of Electrical \& Computer Engineering\\[-0.8ex]
\small University of Utah, Salt Lake City\\
\small \texttt{\{xiaojun.sun,kalla\}@ece.utah.edu}
}

\date{Last Update: May 14, 2013}

\begin{document}
\maketitle

\begin{abstract}
N/A
\end{abstract}

\section{Introduction}

a) Why use NB?\\
b) Relations between NB \& StB?\\
c) What's ONB?\\
{\bf Simple stuff we've discussed.}

\section{Characterization of Normal Basis}
\textbf{Conceptions from Linear Algebra}

%\hspace{8mm}\\

\textit{\underline{Frobenius Map:} \\
$\sigma : x \rightarrow x^q, x \in F_{q^n}$ \\
Linear transformation of $F_{q^n}$ over $F_q$. \\
}

\textit{\underline{T-invariant subspace / cyclic vector:} \\
A subspace $W \subset V$ is called T-invariant when $Tu \in W \forall vector u \in W$\\
Subspace $Z(u,T) = <u,Tu,T^2u,...>$ is called T-cylic subspace of V. \\
If $Z(u,T) = V$, then u is called a cyclic vector of V for T.\\
}

\textit{\underline{Nullspace of polynomial:} \\
For any polynomial $g(x) \in F[x]$, the null space of g(T) consists of all vectors u such that
$g(T)u = 0$.\\
}

\textit{\underline{T-Order / minimal polynomial:} \\
For any vector $u \in V$, the monic polynomial $g(x) \in F[x]$ with smallest degree such that
$g(T)u = 0$ is called the T-Order of u or minimal polynomial of u.\\
That is, for an arbitrary element $\theta$ in $F_{q^n}$, find least positive integer k such that
$\sigma^k\theta = \sum_{i=0}^{k-1} c_i\sigma^i\theta$, then the $\sigma$-Order of $\theta$ can be 
written by $Ord_{\theta,\sigma}(x) = x^k - \sum_{i=0}^{k-1} c_ix^i$. \\
}

\textbf{Lemmas \& Theorems from linear algebra}
\begin{itemize}
\item[-] \textbf{Lemma 1}\ \ \ \ $g(x) \in F[x]$ and W is its null space. Let $d(x) = gcd(f(x),g(x)), e(x) = f(x)/d(x)$. Then $dim(W) = deg(d(x))$ and $W = \{e(T)u | u \in V\}$.
\item[-] \textbf{Lemma 2}\ \ \ \ Factorize f(x): $f(x) = \prod_{i=1}^{r} f_{i}^{d_i} (x)$, each $f_i(x)$ is prime to others. Assume $V_i$ be null space of $f_{i}^{d_i} (x)$, then $V = V_1\oplus V_2 \oplus ...\oplus V_r$.
Furthermore, define $\Psi_i(x) = f(x)/f_{i}^{d_i} (x)$. $\forall u_j \in V_j, u_j \neq 0, \Psi_i(T)u_j \neq 0$, only if $i = j$.
\item[-] \textbf{Lemma 3}\ \ \ \ Minimal and characteristic polynomial for $\sigma$ are both $x^n - 1$.
\item[-] \textbf{Corollary 1}\ \ \ \ An element $\alpha \in F_{q^n}$ is normal element if and only if $Ord_{\alpha,\sigma}(x) = x^n - 1$.
\item[-] \textbf{Theorem 1}\ \ \ \ Consider we are dealing with $F_{2^n}$, then field characteristic $p = 2$.
				Define $t = p^e$ where $n = kp^e, gcd(k,p) = 1$, so here t=1 if n is odd. Then 
			$x^n - 1$ can be factorized as $(\varphi_1(x)\varphi_2(x)\cdot\cdot\cdot\varphi_r(x))^t$.
			Additionally define $\Phi_i(x) = (x^n - 1)/\varphi_i(x)$. We get: \\ 
			An element $\alpha \in 	F_{q^n}$ is normal element if and only if $\Phi_i(\sigma)\alpha \neq 0, i = 1,2,...,r$.
\item[-] \textbf{Theorem 2}\ \ \ \ Let $W_i$ be the null space of $\varphi_{i}^{t} (x)$ and $\widetilde{W_i}$ the
				 null space of $\varphi_{i}^{t-1} (x)$. Let $\overline{W_i}$ be any subspace of
			 $W_i$ such that $W_i = \overline{W_i}\oplus \widetilde{W_i}$. Then\\
			$F_{q^n} = \displaystyle\sum_{i=1}^{r} \overline{W_i}\oplus \widetilde{W_i}$ \\
			is a direct sum where $dim(\overline{W_i}) = d_i$ and $dim(\widetilde{W_i}) = (t-1)d_i$.\\
			Furthermore, an element $\alpha \in F_{q^n}$ with $\alpha = \sum_{i=1}^{r} (\overline{\alpha_i} + \widetilde{\alpha_i}), \overline{\alpha_i} \in \overline{W_i}, \widetilde{\alpha_i} \in \widetilde{W_i}$,
			is a normal element if and only if $\overline{\alpha_i} \neq 0 \forall i = 1,2,...,r$.
\item[-] \textbf{Normal Basis Theorem for Finite Fields}\ \ \ \ There always exists a normal basis of $F_{q^n}$ over $F_q$.
\end{itemize}

\section{Algorithms for Normal Basis Construction}
\textbf{L\"uneburg's Algorithm}\\
Step 1: For each i = 0,1,...,n-1, compute $\sigma$-Order $f_i = Ord_{\alpha^i}(x)$. Here $x^n - 1 = lcm(f_0,f_1,...,f_{n-1})$.\\
Step 2: Apply factor refinement to $\{f_i\}$ and get $f_i = \prod_{1\leq j\leq r} g_{j}^{e_{ij}}, i = 0,1,...,n-1$.\\
Step 3: For each j, find an index $i_j$ (denote as i(j)) so that $e_{ij}$ is max in this j-th column.\\
Step 4: Let $h_j = f_{i(j)}/g_{j}^{e_{i(j)j}}$, take $\beta_j = h_j(\sigma)\alpha^{i(j)}$. Then\\
\indent\indent\indent\indent\indent\indent\indent\indent\indent$\beta = \displaystyle\sum_{j=1}^{r} \beta_j$\\
is the normal element.\\

\textbf{Preliminary to Lenstra's Algorithm}
\begin{itemize}
\item[-] \textbf{Lemma 4}\ \ \ \ For an arbitrary element $\theta \in F_{q^n}$ that $Ord_\theta(x) \neq x^n - 1$,
let $g(x) = (x^n - 1)/Ord_\theta(x)$. There exists another element $\beta$ such that $g(\sigma)\beta = \theta$.\
\item[-] \textbf{Lemma 5}\ \ \ \ Same $\theta$ and $g(x)$ defined as last lemma. Assume there exists a solution 
$\beta$ such that $deg(Ord_\beta(x)) \leq deg(Ord_\theta(x))$. Then there exists a non-zero element $\eta$ such that\\
$g(\sigma)\eta = 0$, and\\
$deg(Ord_{\theta+\eta}(x)) > deg(Ord_\theta(x))$.\\

\end{itemize} 

\textbf{Lenstra's Algorithm}\\
Step 1: Take an arbitrary element $\theta \in F_{q^n}$, determine $Ord_\theta(x)$.\\
Step 2: If $Ord_\theta(x) = x^n - 1$ then algorithm ends.\\
Step 3: Calculate $g(x) = (x^n - 1)/Ord_\theta(x)$, and solve $\beta$ from $g(\sigma)\beta = \theta$.\\
Step 4: Determine $Ord_\beta(x)$. If $deg(Ord_\beta(x)) > deg(Ord_\theta(x))$ then replace $\theta$ by $\beta$ and go to step 2;
	otherwise if $deg(Ord_\beta(x)) \leq deg(Ord_\theta(x))$ then find a non-zero element $\eta$ such that $g(\sigma)\eta = 0$,
	replace $\theta$ by $\theta + \eta$ and determine the order of new $\theta$, then go to step 2.\\

\section{Optimal Normal Basis Characterization}
\textbf{Definition of Optimal Normal Basis}\\
To define optimal normal basis, it is necessary to find out the minimum number
of terms in multiplication table(defined below).
\textit{Theorem 3}\ \ If N is a normal basis for $F_{p^n}$ with multiplication table $M_T$, then non-zero terms in 
matrix $C_N\leq 2n-1$.\\
\textit{Proof}\ \ Let $N = \{\beta, \beta^p, \beta^{p^2},\dots, \beta^{p^{n-1}}$, denote $\beta^{p^i}$ by $\beta_i$. 
Then $\sum_{i=0}^{n-1} \beta_i = trace \beta$. \\
Denote $trace \beta$ with $b$, consider $n\times n$ matrix (multiplication table for lowest digit) $M_0$. Then\\
$b\beta_0 = \sum_{i=0}^{n-1} \beta \beta_i$.\\
Therefore, the sum of all rows in $M_0$ is an n-tuple with $b$ as the first element and zeros elsewhere.
So the first column must contain at least 1 nonzero element, and other columns must get a sum of zero 
while keeping every row linearly independent so there should be at least 2 non-zero elements each column.\\
For $F_{2^n}$, easy to figure out the first row must contain a "1" because $\beta \beta^{2^0} = \beta^{2^1}$.
for the rest rows $\beta \beta^{2^i} \neq \beta^{2^{i+1}}$,
so there must be at least 2 "1"s to fulfil the equation.\\
\textbf{Multiplication table}\\
$\beta \left( \begin{array} {lcr}
 \beta \\ 
\beta^2 \\ 
\cdot \\ 
\cdot \\ 
\cdot \\ 
\beta^{2^{n-1}} 
\end{array} \right) = M_T \left( \begin{array} {lcr}
 \beta \\ 
\beta^2 \\ 
\cdot \\ 
\cdot \\ 
\cdot \\ 
\beta^{2^{n-1}} 
\end{array} \right)$. \\

Complexity $C_N \geq 2n - 1$.\\

\textbf{ONB Existance Theory}\\
Key words: k-th primitive root of unity, Euler's criterion, quadratic residues\\
\hspace{8mm}\par
Type I ONB: n+1 is prime and q is primitive in $\mathbb{Z}_{n+1}$, then the n nonunit (n+1)th roots of unity form ONB.\\
\hspace{8mm}\par
Type II ONB: 2n+1 is prime, if (1) 2 is primitive in $\mathbb{Z}_{2n+1}$, OR \\
(2) $2n + 1 \equiv 3(mod 4)$ and 2 generates the quadratic residues in $\mathbb{Z}_{2n+1}$.\\
Then $\alpha = \gamma + \gamma^{-1}$ generates ONB, where $\gamma$ is a primitive (2n+1)th root if unity.\\

\textbf{Low-complexity Normal Basis design}\\
N/A.\\

\section{Other NB Properties \& Problems}
\textbf{N-poly}\\
N-poly is irreducible polynomial with linearly independent roots. Normal basis is a set of roots of N-poly.
\begin{itemize}
\item[-] \textbf{Corollary 2}\ \ \ \ For irreducible polynomial $f(x) = x^n + a_1x^{n-1} + ... + a_n \in F_{2^n}[x]$, 
		that is an N-poly if and only if $a_1 \neq 0$.
\end{itemize}

\textbf{Problems and Discussion}\\
\hspace{8mm}\par
Prob1: how to prove $\lambda$-Matrix is multiplication table?\\
\indent I think this is a good way to understand the essence of normal basis 
theory. This problem can be rewrite like this:\\
Why there exist a rotating symmetry in $\lambda$-Matrix that
 $\lambda_{0j}^{(k)} = \lambda_{jk}^{(0)}$? Andrew guess it's a Frobenius
 element matrix rotating symmetry involves with field automorphism. Need 
further consultation on this part. How to get this property? Check out the
 last part of my note.\\
\hspace{8mm}\par
Prob2: Proof of Lemma 5.\\
(Already proved. Please check out the note.)\\
\hspace{8mm}\par
Prob3: Fast algorithm for determining $\sigma$-Order.\\
(Do not need FAST algorithms. Just do a $(O(n^2))$ scan.)\\


\end{document}
