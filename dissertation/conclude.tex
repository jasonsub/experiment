\chapter{Conclusions and Future Work}
\label{ch:concl}

A combinational circuit with $k$-inputs and $k$-outputs implements
Boolean functions  $f: \B ^k \rightarrow \B ^k$, where $\B = \{0,
1\}$. The function can also be construed as a mapping  $f: \Fkk
\rightarrow \Fkk$, where $\Fkk$ denotes the Galois field of
$2^k$ elements. A circuit with differing input and output sizes 
computes $f: \B ^m \rightarrow \B ^n$, which can be represented
as a function over Galois fields $f: \F_{2^m} \rightarrow \F_{2^n}$.
This circuit can also be analyzed as the function 
$f: \Fkk \rightarrow \Fkk$, where $\Fkk \supset \F_{2^m}$ 
and $\Fkk \supset \F_{2^m}$.

Every function $f$ over $\Fkk$ is a polynomial
function --- i.e., there exists a unique,  minimal, canonical
polynomial $\Func$ that describes $f$. This dissertation presented novel
techniques based on computer-algebra and algebraic-geometry to
derive the canonical (word-level) polynomial representation from the
circuit as  $Z = \Func(A)$ over $\Fkk$, where $A$ and $Z$ denote, 
respectively, the input and output bit-vectors of the circuit.

A theory for word-level polynomial abstraction of bit-level circuits 
over Galois fields is first developed.
This theory is derived using techniques from computer-algebra, notably the theory of
\Grobner basis. However, due to the computational complexity of computing a \Grobner
basis, the solution is not scalable to large designs.
In order to overcome these limitations, new symbolic computational
algorithms are developed and refined. The algorithms employ techniques
from the binomial expansion over $\Fkk$ and $\F_4$-style reduction and can exploit
hierarchy in a given circuit. Finally, an efficient implementation of the algorithmic
approach is presented.

Experiments show that the proposed approach works exceptionally well
for abstracting word-level Galois field arithmetic circuits. It has been
shown that the approach can abstract and verify these types of circuits with up to 
$1024$-bit datapaths. Other contemporary techniques
cannot to verify these types for circuits beyond $163$-bits and fail to
abstract them beyond $32$-bits.

However, in cases of random logic, the abstraction approach
can generate high-degree polynomials:
\begin{equation}
X^{q-1}+X^{q-2}\dots
\end{equation}
In these cases, the polynomials derived during the computation are dense, 
and the computational complexity of manipulating such polynomials
makes abstraction infeasible.

\section{Future Work}

Due to the modular nature of the proposed solution, there are many
potential future research directions that can be explored.

\subsection{Hardware Acceleration}
The first reduction step, $f_Z\xrightarrow{F-f_{z_i},F_0}_+ r$
is the most computationally complex part of the proposed abstraction approach. This
reduction could be implemented using a hardware accelerator.
Significant speed-ups have 
been observed in GPU implementations of circuit simulation algorithms 
\cite{PengLi:GPU}. 
Furthermore, this work has shown cases where multiple, independent abstractions 
need to be computed at the same time, such as when abstracting a word-level 
representation of a composite field multiplier.

These abstractions can be computed in parallel with one another, and this
parallelism could then be exploited using a GPU. Furthermore, our 
approach to compute the abstractions uses an $F4$-style reduction procedure, 
which performs many complex computations over a large matrix. 
Operations over matrices can be suitably implemented using a GPU.
Lastly, the substitution by $a_i=\Func(A)$ is trivially parallelized. 
Thus, further study is proposed to implement word-level abstraction 
on a general purpose GPU.

\subsection{Integration with EDA Tools} 
The proposed canonical word-level abstraction approach is a full, 
self-contained solution.
It can thus be integrated into other EDA tools. There are direct
applications of word-level abstractions to design synthesis. For instance, 
the approach can compute a functional decomposition of a logic or it could 
be used in high-level RTL synthesis. Since the derived abstraction 
is canonical, it can also be used in verification engines such as SMT solvers.
The abstraction approach most efficiently handles AND/XOR logic, so it
could be used to complement approaches in the mentioned tools that are
efficient over AND/OR logic.

\subsection{Polynomial Reductions using Data-Structures}
The abstraction approach poorly handles chains of OR gates
due to their representation as polynomials of Galois fields.
Other polynomial-based tools \cite{polybori:2009} have shown that it 
can be beneficial to represent polynomials internally as decision diagrams.
Thus, it is worthwhile to explore whether it is possible to
implement the algorithmic approach in a different data-structure that
is better-suited for handling this type of logic.
One candidate data-structure is the And-Invert-Graph, as this structure efficiently
handles OR gates. The widely-used tool ABC\cite{abc} provides a very efficient, 
flexible, and open-source implementation of the AIG data structure.

Recall that a one-step reduction  of the polynomial $f$ by polynomial $g$, 
$f\xrightarrow{g} r$, is computed as:
\begin{equation}
r = f-\frac{LT(f)}{LT(g)}\cdot g
\end{equation}
Over Boolean circuits, $\B \equiv \F_2$. Since the leading term of
any monomial in $\B$ is $1$, and $-1\equiv +1$, then
\begin{equation}
f-(\frac{LT(f)}{LT(g)}\cdot g)\equiv f+(\frac{LM(f)}{LM(g)}\cdot g)
\end{equation}
which computes an XOR operation
\begin{equation}
f\oplus(\frac{LM(f)}{LM(g)}\cdot g)
\end{equation}
while the $\cdot$ operator acts as an AND operation $\wedge$.
So one step of the reduction procedure can be computed as the following 
AND/XOR operation:
\begin{equation}
r = f \oplus \frac{LM(f)}{LM(g)}\wedge g
\end{equation}

Thus, we propose an investigation into implementing the algorithms presented
in this dissertation over AIGs.

\subsection{Application to Sequential Circuit Verification}
Sequential Galois field arithmetic circuits over $\Fkk$
take k-bit inputs and produce a k-bit result after k-clock cycles 
of operation. Formal verification of sequential arithmetic
circuits with large datapath sizes is beyond the capabilities of
contemporary verification techniques. To address this problem,
we described a verification method in \cite{sun:date15} 
which uses the presented abstraction approach to implicitly 
unroll the sequential arithmetic circuit over multiple (k) clock-cycles.
The resulting function computed by the state-registers of the circuit 
is represented canonically as a multi-variate word-level polynomial over
$\Fkk$. While directly applicable to sequential Galois field arithmetic circuits, 
this work needs to be further generalized in order to make applicable to
any sequential state machine. 

\subsection{Application to Formal Software Verification}
Computer algebra techniques based on \Grobner basis theory have been used 
in formal software verification \cite{manna:program}.
In this work, a \Grobner basis computation is used to derive {\it loop invariants}.
However, the derived invariants are not bit-precise, so not every invariant that is
computed can be applied to the verification. As our approach maintains
the input-output relationship in the abstraction, it could be applied to
find bit-precise invariants.


\subsection{Application to Integer Arithmetic Circuits}
The abstraction approach derives a
word-level representation of circuits over Galois fields, $\Fkk$. 
In order to expand its usability, we conjecture whether it is possible to 
apply concepts from this approach to {\bf abstract word-level representations 
of circuits over integer rings}, $\Zkk$. 
As any function over a Galois 
field $\Fkk$ is a polynomial function, there exists a polynomial which 
describes the word-level function of a given circuit over $\Fkk$. 
%If the word-level 
%input and output of the circuit is $A$ and $Z$, respectively, this 
%polynomial exists in the form $Z=\Func(A)$.
However, not every function over an integer ring $\Zkk$ is a polynomial 
function. 
Thus, a single polynomial which describes the function of a circuit over 
$\Zkk$ is not guaranteed to exist. 
%Instead, there could exist a set of
%polynomials $\{f_1,\dots,f_n\}$ which describes this function.
%These polynomials are in following form:
%\begin{eqnarray}
%f_1:Z=\Func_1(A) \nonumber \\
%\vdots \nonumber \\
%f_n:Z=\Func_n(A) \nonumber
%\end{eqnarray}
Even though there may not exist a single polynomial which describes the
entire function over $\Zkk$, elimination theory and \Grobner basis still
applies over this ring. Thus, it may be possible to modify the theory and 
implementation of our word-level 
abstraction approach in order to abstract a set of word-level 
polynomials over $\Zkk$.

