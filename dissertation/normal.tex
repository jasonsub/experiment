\chapter{Functional Verification of Sequential Normal Basis Multiplier}
\label{ch:normal}
In order to utilize our traversal algorithm, it is necessary to find out
a sort of suitable circuit benchmarks which is easy to compute its
Gr\"bner basis (GB). From the work of Lv et al. \cite{lv_dissertation},
we learn that arithmetic circuits in Galois field (GF) is
convertible to an ideal of circuit polynomials, and the 
ideal generators form a GB itself with reverse topological
term order. Furthermore, according to the work of Pruss et al.
\cite{tim_dissertation}, with a limited computation complexity,
we can abstract the word-level signature of an arithmetic 
component working in GF. Thus, we consider the possibility 
of applying our traversal algorithm on sequential Galois
field circuits. In each frame, we can use the techniques 
from \cite{tim_dissertation} to abstract the word-level
signature of the combinational logic, which corresponds
to the transition function in our traversal algorithm.
As a result, we manage to find a type of sequential GF multiplier
which we can apply our traversal algorithm to actually 
verify its functional correctness.

\section{Motivation}
\label{sec:normal_motiv}
