\numberofappendices=1
\appendix
\chapter{}
\fixchapterheading
\section{Representations of Base Field Elements over Extension Fields}\label{append:Fpk}
Consider a field $\F_q$ and a $k$-bit extension of this field, $\F_{q^k}$.
This extension is created using a primitive, irreducible polynomial $P(x)$ of
degree $k$ over $\F_q[x]$.
Any element $A\in \F_{q^k}$ can be represented as
\begin{equation}
A=a_0+a_1\alpha+\dots+a_{k-1}\alpha^{k-1} \label{eqn:aToAFq}
\end{equation}
where $\{a_0,\dots,a_{k-1}\}\in\F_q$ and $\alpha$ is the primitive element of $\F_{q^k}$, i.e. 
$P(\alpha)=0$. The goal is to derive the polynomial functions $a_i=\Func(A)$ for all
$0\leq i < k$. Note that $\F_q$ could itself be an extension field of a base field $\F_p$ 
where $q=p^l$ for some $l\geq 1$.

Each $a_i$ has the following property:
\begin{equation}
a_i=a_i^{q} \label{eqn:appendixaiq}
\end{equation}

Element $A$ has a similar property:
\begin{equation}
A=A^{q^k} \label{eqn:appendixAqk}
\end{equation}

Examine what happens when Equation \ref{eqn:aToAFq} is raised by the power $q$.
The following lemma is applied.

\begin{Lemma}\label{lemma:raiseToPq}\cite{galois_field:mceliece}
{\it Let $\alpha_1, \dots, \alpha_t$ be any elements in $\F_{p^k}$. Then
\begin{equation}
(\alpha_1+\alpha_2+\cdots+\alpha_t)^{p^i}=\alpha_1^{p^i}+\alpha_2^{p^i}+\dots+\alpha_t^{p^i}
\end{equation}
for all integers $i \geq 0$. }
\end{Lemma}

\begin{Corollary}\label{corollary:raiseqk}
Since $q=p^l$ for some $l \geq 1$, Lemma \ref{lemma:raiseToPq} is applicable to
$\F_{q^k}$.
Let $\alpha_1, \dots, \alpha_t$ be any elements in $\F_{q^k}$. Then
\begin{equation}
(\alpha_1+\alpha_2+\cdots+\alpha_t)^{q^i}=\alpha_1^{q^i}+\alpha_2^{q^i}+\dots+\alpha_t^{q^i}
\end{equation}
for all integers $i \geq 0$.
\end{Corollary}
Applying Corollary \ref{corollary:raiseqk} to Equation \label{eqn:aToFq} gives:
\begin{equation}
A^{q^i}=a_0^{q^i}+a_1^{q^i}\alpha^{q^i}+\dots+a_{k-1}^{q^i}\alpha^{(k-1)q^i} \label{eqn:appendixAqi}
\end{equation}
Then, applying 
Equation \ref{eqn:appendixaiq} to this result gives:
\begin{equation}
A^{q^i}=a_0+a_1\alpha^{q^i}+\dots+a_{k-1}\alpha^{(k-1)q^i} \label{eqn:appendixAqiRewrite}
\end{equation}

In this way, $k$ unique equations are derived, $\{A,A^q,\dots,A^{q^{k-1}}\}$,
along with $k$ unknowns, $\{a_0,\dots,a_{k-1}\}$.
These equations can be represented in matrix form, $\mathbf{A = M a}$, where 
$\mathbf{A}=\{A,A^q,\dots,A^{q^{k-1}}\}^T$, 
$\mathbf{M}$ is
a $k$ by $k$ matrix of coefficients $\in \F_{q^k}$, and $\mathbf{a}=\{a_0,\dots,a_{k-1}\}^T$:
\begin{equation}
\begin{bmatrix}
1      &   \alpha           & \alpha^2           & \dots & \alpha^{k-1}\\
1      &   \alpha^q         & \alpha^{2q}        & \dots & \alpha^{(k-1)q}\\
\vdots & \vdots             & \vdots             & ~     & \vdots \\
1      &   \alpha^{q^{k-1}} & \alpha^{2q^{k-1}} & \dots & \alpha^{(k-1)q^{k-1}}
\end{bmatrix}
\begin{bmatrix}
a_0 \\ a_1 \\ \vdots \\ a_{k-1}
\end{bmatrix}
=
\begin{bmatrix}
A \\ A^q \\ \vdots \\ A^{q^{k-1}}
\end{bmatrix}
\end{equation}

Treat $\mathbf{a}$ as a vector of unknowns, $\mathbf{M}$ and $\mathbf{A}$ as constants. 
This system of equations can be solved using Gaussian 
elimination. However, this system also has a special structure which can
be exploited. $\mathbf{M}$ is a $k$ by $k$ Vandermonde matrix of the form 
$V(\alpha,\alpha^q,\dots,\alpha^{q^{k-1}})$.
Due to the property of Vandermonde matrices:
\begin{equation}
|\mathbf{M}| = \prod\limits_{0\leq i < j < k}(\alpha^{q^j} - \alpha^{q^i})
\end{equation}
Since the elements $\{\alpha,\alpha^q,\dots,\alpha^{q^{k-1}}\}$ are distinct:
\begin{equation}
|\mathbf{M}| \neq 0
\end{equation}
Thus, Cramer's rule can be applied to derive 
an equation for every $a_i$:
\begin{eqnarray}
a_i = \frac{|\mathbf{M_i}|}{|\mathbf{M}|} \label{eqn:cramerformFq}%\\
\end{eqnarray}
where $\mathbf{M_i}$ is $\mathbf{M}$ with the column $\{\alpha^i,\alpha^{iq},\dots,\alpha^{iq^{k-1}}\}^T$ 
replaced by $\mathbf{A}$.
\begin{equation}
\mathbf{M_i} = \begin{bmatrix}
1      &   \alpha            & \dots  & \alpha^{i-1}          & A           & \alpha^{i+1}              & \dots & \alpha^{k-1} \\
1      &   \alpha^q         & \dots  & \alpha^{(i-1)q}       & A^q         & \alpha^{(i+1)q}       & \dots & \alpha^{(k-1)q} \\
\vdots & \vdots             & \cdot  & \vdots                & \vdots      & \vdots                & \cdot & \vdots \\
1      &   \alpha^{q^{k-1}} & \dots  & \alpha^{(i-1)q^{k-1}} & A^{q^(k-1)} & \alpha^{(i+1)q^(k-1)} & \dots & \alpha^{(k-1)q^(k-1)}
\end{bmatrix}
\end{equation}

Computing $|\mathbf{M_i}|$ by interpolating along the $A'$ column gives
\begin{equation}
|\mathbf{M_i}|=\sum\limits_{j=0}^{k-1}(-1)^{(i+j)}A^{q^j}
|V_{i+1}(\alpha,\dots,\alpha^{q^(j-1)},\alpha^{q^(j+1)},\dots,\alpha^{q^(k-1)})|
\end{equation}
where $V_i(x_1,\dots,x_n)$ is the Vandermonde matrix $V(x_1,\dots,x_n)$ with the
$i$-th column skipped and an extra column added to the end.
\begin{equation}
V_i(x_1,\dots,x_n) 
=
\begin{bmatrix}
1 & x_1 & x_1^2  & \dots  & x_1^{i-1} & x_1^{i+1} & \dots & x_1^n \\
1 & x_2 & x_2^2  & \dots  & x_2^{i-1} & x_2^{i+1} & \dots & x_2^n \\
\vdots & \vdots  & \vdots & \cdot  & \vdots    & \vdots    & \cdot & \vdots    \\
1 & x_n & x_n^2  & \dots  & x_n^{i-1} & x_n^{i+1} & \dots & x_n^n
\end{bmatrix} 
\end{equation}
The computation of determinant of $V_i$ is known to the linear algebra community to
be
\begin{equation}
|V_i(x_1,\dots,x_n)| = |V(x_1\dots,x_n)|\cdot S_{n-i}(x_1,\dots,x_n)
\end{equation}
where $S_i(x_1,\dots,x_n)$ is the $i$-th fundamental symmetric polynomial in $\{x_1,\dots,x_n\}$.
Thus, $|\mathbf{M_i}|$ is computed as.
\begin{eqnarray}
|\mathbf{M_i}| &=& \sum\limits_{j=0}^{k-1}[(-1)^{j}A^{q^j} \nonumber \\
& & \cdot |V(\alpha,\dots,\alpha^{q^(j-1)},\alpha^{q^(j+1)},\dots,\alpha^{q^(k-1)})| \nonumber \\
& & \cdot S_{n-1-i}(\alpha,\dots,\alpha^{q^(j-1)},\alpha^{q^(j+1)},\dots,\alpha^{q^(k-1)})] \label{eqn:MireducedFq}
\end{eqnarray}

Lastly, this work has discovered the following proposition for the determinant $|\mathbf{M}|$.
\begin{Proposition}
The determinant $|\mathbf{M}|$ has the property
\begin{equation}
|\mathbf{M}|^q=(-1)^{k-1}|\mathbf{M}|
\end{equation}
\end{Proposition}
\begin{Proof}
\it Since $\mathbf{M}$ is a Vandermonde matrix of the form $V(\alpha,\alpha^q,\dots,\alpha^q{k-1})$, 
then 
\begin{equation}
|\mathbf{M}| = \prod\limits_{0\leq i < j < k}(\alpha^{q^j}-\alpha^{q^i}) \label{proofeqn:detMFq}
\end{equation}
Computing $|\mathbf{M}|^q$ gives
\begin{equation} \label{proof:MsquareFq}
|\mathbf{M}|^q = [\prod\limits_{0\leq i < j < k}(\alpha^{q^j}-\alpha^{q^i})]^q
\end{equation}
Applying Corollary \ref{corollary:raiseqk} to Equation \ref{proof:MsquareFq} gives
\begin{equation}
|\mathbf{M}|^q = \prod\limits_{0\leq i < j < k}(\alpha^{q^{j+1}}-\alpha^{q^{i+1}})
\end{equation}
When $j=k-1$, the product term is in the form $(\alpha^{q^k}-\alpha^{q^{i+1}})$. 
Since $\alpha^{q^k}=\alpha$ over $\F_{q^k}$, this term equivalent to  $-(\alpha^{q^{i+1}}-\alpha)$.
This gives the property:
\begin{equation}
|\mathbf{M}|^q=(-1)^{k-1}|\mathbf{M}|
\end{equation}
\end{Proof}
%Appendix \ref{append:Fpk}
