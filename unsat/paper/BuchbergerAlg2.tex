%%This is a very basic article template.
%%There is just one section and two subsections.
\documentclass{article}
\usepackage[english]{babel}
\usepackage{amsmath}
\usepackage{amsthm}
\usepackage{amssymb}


\theoremstyle{plain}
\newtheorem{defn}{Definition}[section]
%\usepackage{datetime}
\newtheorem{thm}[defn]{Theorem} % definition numbers are dependent on
% theorem
% %
\newtheorem{prop}[defn]{Proposition} % definition numbers are dependent on
\newtheorem{dem}[defn]{Proof} % definition numbers are dependent on
              % numbers
\newtheorem{exemp}{Example}[defn]
\newtheorem{lema}{\bf Lemma}[defn]


\def\opn#1#2{\def#1{\operatorname{#2}}} % to make operators

\opn\Ker{Ker} \opn\Coker{Coker}  \opn\Hom{Hom} \opn\Im{Im}
\opn\End{End} \opn\Aut{Aut} \opn\defect{def} \opn\ord{ord}
\opn\id{id} \opn\dim{dim} \opn\det{det} \opn\tr{tr} \opn\grad{grad} \opn\lcm{lcm}
\opn\min{min} \opn\max{max} \opn\det{det}
%\def\Frob{{\mathcal F}}
\opn\Span{Span}   \opn\rang{rang}  \opn\id{id} \opn\dim{dim} \opn\ad{ad} \opn\tr{tr} \opn\ker{ker}
\opn\GL{GL} \opn\SL{SL} \opn\mod{mod} \opn\diag{diag}
\opn\min{min} \opn\sgn{sgn} \opn\ini{in_<}  \opn\Mon{Mon} \opn\LC{LC_<}
\opn\LT{LT_<}
\opn\s{supp}  


\title{On Cores of Ideals with Empty Variety}


\begin{document}

\maketitle


Let $F=\{f_1,f_2,\ldots,f_m\}$ be a set of $m$ polynomials in $\mathbb{F}_2[x_1,\ldots,x_n]$ and $I=(F)$ the ideal generated by the set $F$. Assume that our term order is LEX with $x_1>x_2>...>x_n$. Suppose that it is known that $V(I)=\emptyset$.
 
We can now apply the Buchberger's algorithm \cite{HH} to the system of generators $F$ of $I$.
 
Next, we will keep track of the S-polynomials that give non-zero remainder when divided by the system of generators of $I$ at that moment.
\begin{center}
$g_{ij}:= S(f_i,f_j)-\displaystyle\sum_{k=1}^l c_kf_k$,
\end{center}
where $0\neq c_k\in\mathbb{F}_2[x_1,\ldots,x_n]$ and $\{f_1,\ldots,f_l\}$ is the current system of generators of $I$.

For each non-zero $g_{ij}$, we will record the following data:
$$((g_{ij})(f_i,f_j)(h_{ij},h_{ji})| (f_1,\ldots,f_l),(c_1,\ldots,c_l))$$


 Note that here $g_{ij}$ denotes the remainder of the $S$-polynomial $S(f_i,f_j)$ modulo the current system of generators $f_1,\ldots,f_l$ of the ideal $I$ and we denote by 
$$h_{ij}:=\displaystyle\frac{\lcm(\ini(f_i),\ini(f_j))}{\LT(f_i)}, h_{ji}:=\displaystyle\frac{\lcm(\ini(f_i),\ini(f_j))}{\LT(f_j)}$$
the coefficients of $f_i$, respectively $f_j$ in the $S$-polynomial $S(f_i,f_j)$.


I recall that we have $$S(f_i,f_j)=\displaystyle\frac{\lcm(\ini(f_i),\ini(f_j))}{\LT(f_i)}f_i-\frac{\lcm(\ini(f_i),\ini(f_j))}{\LT(f_j)}f_j$$.

The last part contains the $f_i$'s that appear in the division process and the second parentheses has the corresponding coefficients.

 When we obtain $1$ as a remainder of a $S$-polynomial to the current system of generators, we stop.

As an output of the Buckberger's Algorithm, we can obtain not only the Gr\"obner bases, let's call it $\{g_1,\ldots,g_t\}$, but also a matrix $M$ such that
\begin{center}
\begin{align}
   \begin{bmatrix}
           g_{1} \\
           g_{2} \\
           \vdots \\
           g_{t}
         \end{bmatrix}
    &= M \begin{bmatrix}
           f_{1} \\
           f_{2} \\
           \vdots \\
           f_{m}
         \end{bmatrix}
  \end{align}

\end{center}
\begin{thm}
With the notations above, we have that a core for the system of generators $F$ of the ideal $I$ is given by the union of those $f_i$'s from $F$ that appear in the data recorded above and correspond to the nonzero entries in the matrix $M$. 
\end{thm}

\begin{proof}
In our case, $g_1=1$ and $t=1$, since the variety is empty, and hence the ideal is the unit ideal. Say $M= [a_1, \ldots, a_m]$. Then the output of the algorithm gives

$$1 = a_1f_1+\cdots + a_m f_m.$$ Clearly, if $a_i=0$ for some $i$ then $f_i$ does not appear in this equation and should not be included in the core.

\end{proof}


%
%

 Our next goal will be to describe an algorithm that allows us to compute a core of the ideal and the matrix $M$.

INITIAL DATA: $S_0=\{f_1,\ldots,f_m\}$ system of generators of the ideal $I$, monomial order $<$ on $\mathbb{F}_2[x_1,\ldots,x_n]$.

STEP 1: We start computing the S-polynomials of the system of generators $\{f_1,\ldots,f_m\}$ as in the Buchberger's algorithm \cite{HH}. After computing the first one, we divide it by the system of generators $S_0$. In this way, we will obtain the following expression
$$g_{i_1i_2}:= h_{i_1i_2}f_{i_1}-h_{i_2i_1}f_{i_2}-\displaystyle\sum_{k=1}^m c_kf_k$$

If the remainder $g_{i_1i_2}$ is non-zero, we will denote it by $f_{m+1}$ and we will add it to our initial system of generators. We will also record the data as follows
$$((f_{m+1}:=g_{i_1i_2})(f_{i_1},f_{i_2})(h_{i_1i_2},h_{i_2i_1})| (f_1,\ldots,f_m),(c_1,\ldots,c_m))$$ 
Then we will consider $S_1:=\{f_1,\ldots,f_m,f_{m+1}\}$ to be the new system of generators of $I$.

STEP s:  If $S_s:=\{f_1,\ldots,f_{s}\}$ is the current system of generators of $I$ and we have the following relation for $f_s$
$$f_s=g_{ij}=h_{ij}f_i-h_{ji}f_j-\displaystyle\sum_{k=1}^{s-1} a_kf_k$$

and the recorded data 
$$((f_{s}:=g_{ij})(f_{i},f_{j})(h_{ij},h_{ji})| (f_1,\ldots,f_{s-1}),(a_1,\ldots,a_{s-1}))$$

We will compute the next S-polynomial which gives a non-zero remainder when divided by our current system of generators $S_s$
$$f_{s+1}=g_{pq}=h_{pq}f_p-h_{qp}f_q-\displaystyle\sum_{k=1}^s b_kf_k$$
  
By substituting $f_s$ in the expression of $f_{s+1}$ we get 
$$f_{s+1}= h_{pq}f_p-h_{qp}f_q-\displaystyle\sum_{k=1}^{s-1} b_kf_k-b_s(h_{ij}f_i-h_{ji}f_j-\displaystyle\sum_{k=1}^{s-1} a_kf_k)$$
$$= h_{pq}f_p-h_{qp}f_q-\displaystyle\sum_{k=1}^{s-1} b_kf_k-b_sh_{ij}f_i+b_sh_{ji}f_j+\displaystyle\sum_{k=1}^{s-1} b_sa_kf_k$$
$$= h_{pq}f_p-h_{qp}f_q-b_sh_{ij}f_i+b_sh_{ji}f_j+\displaystyle\sum_{k=1}^{s-1} (b_sa_k-b_k)f_k$$

 Next, we will record the data for $f_{s+1}$
$$((f_{s+1}:=g_{pq})(f_{p},f_{q})(h_{pq},h_{qp})| (f_1,\ldots,f_{s-1},f_i,f_j),(b_sa_1-b_1,\ldots,b_sa_{s-1}-b_{s-1},b_sh_{ij},b_sh_{ji}))$$


The algorithm will stop when we will obtain the last non-zero remainder $f_l=1$. After using the previous relations, we will have the following
$$1=f_l=\displaystyle\sum_{k=1}^{m} d_kf_k$$
OUTPUT:
the core of the system of generators $S_0$ is $\{f_k: k\in\overline{1,m}, d_k\neq 0\}$
        
the matrix with the coefficients  $$M=(d_1,\ldots,d_k)$$.
	\begin{exemp}
Consider the following set of polynomials:
\begin{center}
$f_1: abc + ab + ac + bc + a + b + c + 1$
\end{center}
\begin{center}
$f_2: b$
\end{center}
\begin{center}
$f_3: ac$
\end{center}
\begin{center}
$f_4: ac + a$
\end{center}
\begin{center}
$f_5: bc + c$
\end{center}
\begin{center}
$f_6: abd + ad + bd + d$
\end{center}
\newpage
\begin{center}
$f_7: cd$
\end{center}
\begin{center}
$f_8: abd + ab + ad + bd + a + b + d + 1$
\end{center}
\begin{center}
$f_9: abd + ab +bd + b$
\end{center}
 
Let $S_0=\{f_1,\ldots,f_9\}$ and $I=(S_0)$ ideal in $\mathbb{F}_2[a,b,c,d]$. Then $V(I)=\emptyset$ as $GB(I)=\{1\}$.

INITIAL DATA: $S_0=\{f_1,\ldots,f_9\}$ system of generators of the ideal $I$, monomial order $<=<_{LEX}$ on $\mathbb{F}_2[a,b,c,d]$

STEP 1:
 
 We start computing the S-polynomials and their corresponding remaiders modulo the current system of generators.
 We keep track of the first non-zero remaider
$$f_{10}=g_{12}=f_1+acf_2+af_2+f_3+cf_2+f_2$$
recording the data as follows

\begin{center}
$((f_{10}=g_{12}), (f_1,f_2)(1,ac)|(f_2,f_3,f_2,f_2)(a,1,c,1))$
\end{center}
STEP 2:
 The second non-zero remainder is given by
$$f_{11}=g_{18}=df_1+cf_8+f_1+adf_2+af_2+df_{10}+df_2+f_7+f_{10}+f_{2}$$
and the corresponding recorded data 
$$((f_{11}=g_{18}), (f_1,f_8)(d,c)|(f_1,f_2,f_2,f_{10},f_2,f_7,f_{10},f_{2})(1,ad,a,d,d,1,1,1))$$

  Now by substituting $f_{10}$ in the expression of $f_{11}$, we obtain

$$f_{11}=(acd+ac+cd+c)f_2+(d+1)f_3+f_7+cf_8$$
so we have the following data
$$((f_{11}=g_{18}), (f_1,f_8)(0,c)|(f_2,f_3,f_7)(acd+ac+cd+c,d+1,1)$$ 

STEP 3:
The last non-zero remaider comes from the $S$-polynomial of $f_3$ and $f_4$
$$1=f_{12}=g_{34}=1=f_3+f_4+f_{10}+f_{11}$$
so we record the data 
$$((f_{12}=g_{34}=1), (f_3,f_4)(1,1)|(f_{10},f_{11})(1,1))$$
and after making the substitutions we obtain
$$1= f_1 + (acd + cd + a + 1)f_2 + (d+1)f_3 + f_4 + f_7 + c f_8$$

OUTPUT:

the core of the $S_0$ is given by  $\{f_1,f_2,f_3,f_4,f_7,f_8\}$ 
 
the matrix $M$ is given by $$M=(1,acd + cd + a + 1,d+1,1,1,c)$$

 
\end{exemp}				
\begin{thebibliography}{99}

\bibitem{HH}  J\"urgen Herzog, Viviana Ene, \emph{Gr\"obner Bases in Commutative Algebra}, American Mathematical Society, 2012.
\end{thebibliography}

\end{document}