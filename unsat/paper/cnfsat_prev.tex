\section{UNSAT core extraction through CNF-SAT solvers}
There are 2 major kinds of minimal UNSAT core extraction algorithms,
one is based on deletion,  the other one is based on insertion. Both
algorithms require multiple calls of SAT solver to verify if a clause
is necessary,  so the major efforts to enhance efficiency are to
reduce the number of calls. Currently a large portion of SAT solvers
work on DPLL algorithm and its variations,  which use resolution --
SAT solver can produce an UNSAT core by reading resolution refutation
-- and are usually incremental,  thus it is possible to re-use
previously learnt information from SAT solving. 

One of the state-of-art MUS extractors based on resolution is
Haifa-MUC \cite{Haifa2014}, it gives a proof of UNSAT to incremental
SAT solver rather than the original formula. Moreover,  several
optimizations are implemented within this tool to boost its
performance: maintaining a partial resolution proof instead of whole
proof can reduce processing time; applying Eager model rotation can
mark multiple clauses as necessary without calling  SAT solver. AMUSE
\cite{AMUSE} is a typical MUS extractor using insertion-based
algorithm. It adds selector variables to original formula,  and
constructs an unsatisfiable subformula by making assumptions. 

Recently A. Belov et al. \cite{Joao2012} introduced hybrid MUS
extraction algorithm combining methods from  both deletion-based and
insertion-based extractors. It not only grows UNSAT core from small
subsets but also call SAT solvers for its clause removing
strategy. Recursive model rotation (RMR) technique is also used to
learn transition clauses to save the cost of calling SAT solvers. 
