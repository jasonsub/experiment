\section{Reducing the Size of the Infeasible Core $F_c$}

The core $F_c$ obtained from the aforementioned procedure is not
guaranteed to be minimal. For example, let us revisit the the core
$F_c=\{f_1,\dots, f_5\}$ generated in the previous section. Note that
while $F_c$ is a smaller infeasible core of $F$, it is not minimal. In
fact, Example 1 shows that $F_{c4} =\{f_1,f_2,f_4,f_5\}$ is the
minimal core where $F_{c4} \subset F_{c}$. Clearly, the polynomial
$f_3 \in F_c$ is a redundant element of the core. It could be
discarded. We will now describe how the size of this core could be
reduced further.  For this purpose, we will have to perform a more
systematic book-keeping of the data generated during the execution of
Buchberger's algorithm and the refutation tree. 

\subsection{Identifying redundant polynomials from the refutation tree}

We will keep track of the S-polynomials that give a non-zero remainder
when divided by the system of polynomials $F$ at that moment:
\begin{equation}
\label{eqn1}
g_{ij}= S(f_i,f_j)+\displaystyle\sum_{k=1}^m c_kf_k,
\end{equation}
where $0\neq c_k\in\mathbb{F}[x_1,\ldots,x_n]$ and
$\{f_1,\ldots,f_l\}$ is the ``current'' system of polynomials
$F$. For each non-zero $g_{ij}$, we will record the following data: 

\begin{equation}
\label{data1}
((g_{ij})(f_{i},h_{ij})(f_{j},h_{ji})| (c_{k1},f_{k1}),(c_{k2},f_{k2}),\dots,(c_{kl},f_{kl}))
\end{equation}

Note that in (\ref{eqn1}) and (\ref{data1}), $g_{ij}$ denotes the
remainder of the $S$-polynomial $S(f_i,f_j)$ modulo the current system
of polynomials $f_1,\ldots,f_m$, and we denote by  

$$h_{ij}:=\displaystyle\frac{\lcm(lm(f_i),lm(f_j))}{lt(f_i)},
h_{ji}:=\displaystyle\frac{\lcm(lm(f_i),lm(f_j))}{lt(f_j)}$$ 
the coefficients of $f_i$, respectively $f_j$, in the $S$-polynomial
$S(f_i,f_j)$. Furthermore, in (\ref{data1}), $(c_{k1}, \dots c_{kl})$ are the
polynomial coefficients of $(f_{k1}, \dots,f_{kl})$ that appear in the
division process. 

\begin{example}
Revisiting Example \ref{ex:1},  the data corresponding to
$Spoly(f_1,f_2)\xrightarrow{F}_+  g_{12} = f_{10}$ reduction is
obtained as the following sequence of computations:
$$f_{10}=g_{12}=f_1-acf_2-af_2-f_3-cf_2-f_2.$$ As the coefficient
field is $\mathbb{F}_2$ in this example, $-1 = +1$, so:
$$f_{10}=g_{12}=f_1+acf_2+af_2+f_3+cf_2+f_2$$ is obtained.
The data is recorded according to (\ref{data1}):

\begin{center}
$((f_{10}=g_{12}), (f_1,1)(f_2,ac)|(a,f_2),(1,f_3),(c,f_2),(1,f_2))$
\end{center}

\end{example}

Of course, our approach and the book-keeping stops when we obtain
1 as the remainder of the S-polynomial modulo the current system of
generators. As an output of the Buchberger's algorithm, we can obtain
not only the Gr\"obner bases,  $G = \{g_1,\ldots,g_t\}$, but also a
matrix $M$ such that: 
\begin{center}
\begin{align}
   \begin{bmatrix}
           g_{1} \\
           g_{2} \\
           \vdots \\
           g_{t}
         \end{bmatrix}
    &= M \begin{bmatrix}
           f_{1} \\
           f_{2} \\
           \vdots \\
           f_{s}
         \end{bmatrix}
  \end{align}

\end{center}

Each element $g_i$ of $G$ is a polynomial combination of $f_1, \dots,
f_s\}$. Moreover, this matrix $M$ is constructed precisely using the
data that is recorded in the form of (\ref{data1}). We now a condition
when the matrix $M$ may identify some redundant elements. 


\begin{theorem}
\label{thm}
With the notations above, we have that a core for the system of
generators $F = \{f_1,\dots,f_s\}$ of the ideal $J$ is given by the
union of those $f_i$'s from $F$ that appear in the data recorded above
and correspond to the nonzero entries in the matrix $M$.  
\end{theorem}

\begin{proof}
In our case, since the variety is empty, and hence the ideal is the
unit ideal, we have that $G = \{g_1=1\}$ and $t=1$. Therefore $M=
[a_1, \ldots, a_s]$ is a vector. Then the output of the algorithm
gives:

$$1 = a_1f_1+\cdots + a_s f_s.$$ Clearly, if $a_i=0$ for some $i$ then
$f_i$ does not appear in this equation and should not be included in
the infeasible core of $F$. 

\end{proof}


\begin{example}

Corresponding to Example \ref{ex:1} and the refutation tree shown in
Fig. \ref{fig:refute}, we discover that the polynomial $f_3$ is used
only twice in the division process. In both occasions, the quotient of
the division is 1. Therefore, from Fig. \ref{fig:refute}, it follows that:
\begin{equation}
1 = (f_2 + f_5) + \dots + \mathbf{1\cdot f_3} + \dots + \mathbf{1\cdot f_3}+ \dots (f_1
+ f_2)
\end{equation}

Since $1 + 1 = 0$ over $\F_2$, we have that the entry in $M$
corresponding to $f_3$ is 0, and so $f_3$ can be discarded from the
core. 
\end{example}

\subsection{The GB-Core Algorithm Outline}

The following steps describe an algorithm (GB-Core) that allows us to compute a
refutation tree of the polynomial set and corresponding matrix $M$. 

{\bf Inputs:} Given a system of polynomials $F=\{f_1,\ldots,f_s\}$
, a monomial order $>$ on $\mathbb{F}[x_1,\ldots,x_n]$. 

{\bf S-polynomial reduction:} We start computing the S-polynomials of the system of
generators $\{f_1,\ldots,f_s\}$, then divide each of them by the system of
generators $G=\{f_1,\dots,f_s,\dots,f_m\}$ which is the intermediate result of Buchberger's algorithm. 
In this way, we obtain the following expression:
\begin{equation}
\label{eqn:red}
g_{ij}= h_{ij}f_{i}+h_{ji}f_{j}+\displaystyle\sum_{k=1}^m c_kf_k
\end{equation}
If the remainder $g_{ij}$ is non-zero, we will denote it by
$f_{m+1}$ and we will add it to intermediate system of generators $G$. We
will also record the data as (\ref{data1}): 
\begin{displaymath}
((f_{m+1}=g_{ij})(f_{i},h_{ij})(f_{j},h_{ji})| (c_{k1},f_{k1}),(c_{k2},f_{k2}),\dots,(c_{kl},f_{kl}))
\end{displaymath}
It forms a tree with root $f_{m+1}$.

{\bf Coefficients recording:} In (\ref{eqn:red}) we get a coefficient vector including coefficients
$c_k$ where $k>s$. These coefficients are associated with new generators in the Gr\"obner basis,
which cannot benefit the UNSAT core extraction. Therefore we need to rewrite the vector to
one with only $c_k, k=1\dots s$, which can be achieved by substituting $f_k,k>s$ by $f_1,\dots,f_s$:
initially we have $f_{s+1}= c_1f_1 + c_2f_2 + \cdots + (h_{ij}+c_i)f_{i}+\cdots+(h_{ji}+c_j)f_{j}+\cdots+c_sf_s$,
inductively if we can present $f_{s+1}$ to $f_{s+k-1}$ by $f_1,\dots,f_s$, we can also rewrite
$f_{s+k} = d_1f_1+\cdots+d_sf_s$. Then we add a new row $ (d_1,\dots,d_s ) $ to coefficient matrix $M$.

{\bf Termination and refutation tree construction:} By repeating S-polynomial reduction and coefficients 
recording until we get remainder $f_m = 1$, the algorithm
will generate a set of new generators $G$ and coefficient matrix $M$.
Meanwhile we can construct the refutation tree with the recorded data.
We start with 
\begin{displaymath}
((f_{m}=1)(f_{i},h_{ij})(f_{j},h_{ji})| (c_{k1},f_{k1}),(c_{k2},f_{k2}),\dots,(c_{kl},f_{kl}))
\end{displaymath}
It represents a tree start from leaves $f_{i}, f_j$ then internal nodes $f_{k1},\dots,f_{kl}$
and ``1" as the root. Next we substitute nodes $f_{m-1}, f_{m-2}, \dots, f_{s+1}$ in order
with corresponding subtrees. The algorithm returns the complete refutation tree as well as
matrix $M$.

The complete algorithm is concluded as pseudo code below:

\begin{algorithm}[H]
 \caption{GB-core algorithm (based on Buchberger's algorithm)}
 \label{algo:gbcore}
 \begin{algorithmic}[1]

 \REQUIRE $F = \{f_1, \dots, f_s\} \in \F[x_1, \dots, x_n], f_i\neq 0$
 \ENSURE Refutation tree $T$ and coefficients matrix $M$
 \STATE{ {Initialize: list $G \gets F$; Dataset $D\gets \emptyset$; $M\gets s\times s$ unit matrix} }
 \FOR {{ each pair $(f_i,f_j)\in G$  }}
 	\STATE  $f_{sp},(f_{i},h_{ij})(f_{j},h_{ji}) \gets$ spoly-pair($f_i,f_j$) \COMMENT{$f_{sp}$ is the S-polynomial}
 	\STATE{{ $g_{ij}|(c_{k1},f_{k1}),\dots,(c_{kl},f_{kl}) \gets (sp\xrightarrow{G}_+  g_{ij})$}}
 	\IF{{ $g_{ij} \neq 0$ }}
 		\STATE {{$G \gets G \cup g_{ij}$}}
 		\STATE {{Update $D$ and $M$}}
 	\ENDIF
 	\IF {{$g_{ij} = 1$}}
 		\STATE {{Construct $T$ from $D$}}
 		\STATE {{ Return($T,M$) }}
 	\ENDIF
 \ENDFOR
 \end{algorithmic}
 \end{algorithm}
