\section{Experiment results}
We have implemented our core extraction approach (the GB-Core
algorithm) using the \textsc{Singular} symbolic
algebra computation system [v. 3-1-6] \cite{DGPS}. With our
tool implementation, we have performed experiments to extract a
minimal unsat core from a given set of  
polynomials. Our experiments run on a desktop with
3.5GHz Intel $\text{Core}^\text{TM}$ i7-4770K Quad-core CPU, 16 GB RAM and
64-bit Ubuntu Linux OS.

%Nowadays most SAT benchmarks are huge, which choke our GB engine. 
Gr\"obner basis is not an efficient engine for solving contemporary
CNF-SAT benchmarks that are of large size. In order to validate our
approach, we use a customized  benchmark library: "aim-100" is revised
from ramdon 3-SAT benchmark "aim-50/100", "subset" series are generated for
random subset sum problems, "cocktail" is revised from a mixed
factorization/random 3-SAT benchmark, and "phole4/5" are generated by
adding redundandant clauses to pigeon hole benchmarks. Moreover, we
created SMPO/RH benchmarks that correspond to
equivalence checking instances of sequential Galois field normal basis
modulo multiplier implementations (also known as Agnew's multiplier \cite{SMPO}
and Reyhani-Hassan multiplier \cite{RHmulti} respectively) against the golden model spec. 
Similarly, the "MasVMon" benchmarks are the miter circuits of
 Mastrovito multipliers compared against  Montgomery
multipliers. The CNF formulae are translated as polynomial constraints
over $\mathbb{F}_2$ (as shown in \cite{condrat-tacas07}), and the
GB-Core algorithm is applied. 
%These benchmarks are in moderate size for our GB engine but not too trivial. 

\vspace{-0.1in}
\begin{table}[htb]
\centering
\caption{\centering Results of running benchmarks using our tool\\
\small{Asterisk($^*$) denotes a benchmark which is not converted from CNF benchmark}}
\begin{tabular}{|c||c|c|c|c|c|c|}
\hline
\multirow{4}{2.2cm}{\centering Benchmark} 
& \multirow{4}{0.9cm}{\centering \# Polys} 
& \multirow{4}{0.8cm}{\centering \# MUS} 
& \multirow{4}{1.9cm}{\centering Size of core w/ iterative refinement heuristic}
 & \multirow{4}{1.6cm}{\centering \# GB-core iterations}
 & \multirow{4}{2cm}{\centering \# Size of core after using syzygy heuristic}
 & \multirow{4}{1.3cm}{\centering Runtime (sec)} \\
  & & & & & & \\
    & & & & & & \\
      & & & & & & \\
\hline
\hline
$5\times 5$ SMPO & 240  & 137  & 137  & 8  &  & 3160\\
$4\times 4$ SMPO$^*$ & 84  & 21  & 21  & 1  &  & 125\\
$3\times 3$ SMPO$^*$ & 45  & 15  & 15  & 1  &  & 6.6\\
$3 \times 3$ SMPO & 17 & 2 & 2 & 1 &  & 0.07  \\
$4 \times 4$ MasVMont$^*$ & 148 & 83 & 83 & 1 &  & 23 \\
$3 \times 3$ MasVMont$^*$ & 84 & 53 & 53 & 1 &  & 4.3 \\
$2 \times 2$ MasVMont & 27 & 23 & 23 & 2 &  & 2.3 \\
$5\times 5$ RH$^*$ & 142  & 34  & 35  & 4  &  & 1001\\
$4\times 4$ RH$^*$ & 104  & 35  & 36  & 3  &  & 98\\
$3\times 3$ RH$^*$ & 50  & 20  & 20  & 1  &  & 2.9\\
aim-100 & 79 & 22 & 22 & 1  &  & 43\\
cocktail & 135 & 4 & 4 & 2 &  & 5 \\
subset-1 & 100 & 6 & 6 & 1 &  & 2.4 \\
subset-2 & 141 & 19 & 23 & 2 & 21 & 12 \\
subset-3 & 118 & 16 & 12 & 2 & 11 & 8.7 \\
phole4 & 104 & 10 & 16 & 1 & 10 & 4.4 \\
phole5 & 169 & 19 & 25 & 3 & 19 & 14 \\
\hline
\end{tabular}
\label{tab:result}  
\end{table} 

In Table \ref{tab:result}, we list the details of our experimental
results. In the table, \#Polys denotes the given number of polynomials
from which a core is to be extracted. \#MUS is the minimal core
extracted by PicoMUS. \#GB-core iterations corresponds to the
number of calling GB-core engine to arrive at an unsatisfiable core. The
second last column is showing the improvement of minimal core
size applying syzygy heuristic on cases which are not able to be
refined further with the iterative heuristic.
The data shows that in most of these
cases, our tool can produce a minimal core when our iterations
converge. From the experimental results we can also make the
observation that the GB-core algorithm, particularly Theorem
\ref{thm}, offers quite a lot of scope for identifying redundant
polynomials that can be eliminated from the core --- without resorting
to a brute-force membership check of every polynomial $f_i \in F -
\{f_i\}$. 




\section{Conclusions}
This paper addressed the problem of identifying an infeasible core of
a set of multivariate polynomials, with coefficients from a field,
that have no common zeros. The problem is posed in the context of
computational algebraic geometry and solved using the Gr\"obner basis
algorithm. We show that by recording the data produced by the
Buchberger's algorithm -- the $Spoly(f_i,f_j)$ pairs, as well as the
polynomials of $F$ used in the division process
$Spoly(f_i,f_j)\xrightarrow{F}_+ 1$ -- we can identify certain
conditions under which a polynomial can be discarded from a core. An
algorithm was implemented within the Singular computer algebra tool
and some experiments were conducted to validate the approach. While
the use of GB engines for SAT solving has a rich history, the problem
of unsat core identification hasn't been fully addressed by the SAT
community. We hope that this paper will kindle some interest in this
topic which is worthy of attention from the SAT community. 
