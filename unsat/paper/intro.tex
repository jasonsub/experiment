\section{Introduction}

The Boolean Satisfiability Problem (SAT) is formulated as finding
solutions satisfying a set of Boolean equations, or to show that no
such solutions exist (unsat). Such problems are often represented in
Conjunctive Normal Form (CNF), whereby sets of literal-disjunctions
(clauses) must be simultaneously satisfied through some variable
assignment. When no solutions exist, it is possible to identify a
subset $C_c$ of the original set of clauses $C$ such that $C_c$ is
unsat too. This set $C_c$ is called the unsat core of the
problem. Unsat cores have various uses in system design and
verification, where they can identify the causes of unsat in the
problem and restore satisfiability. Various unsat core extractors are
available \cite{Haifa2014} \cite{Joao2012} that can identify smaller
cores -- and {\it minimal} unsat cores -- from very large unsat
problems.  

The problem has analogous manifestations in the polynomial algebra
domain. Suppose that we are given a set of polynomials $F =
\{f_1,\dots,f_s\}$ with coefficients from any field. Suppose, further,
that the system of equations $f_1 = f_2 = \dots = f_s = 0$ has no
common zero -- i.e. the system of polynomial constraints is infeasible
(or unsat). Can we identify a subset $F_c \subseteq F$ such that the
system of polynomials of $F_c$ also has no common
zeros? In other words, we wish to devise algorithmic techniques to
identify infeasible or unsat cores $F_c$ of $F$. 

Clearly, computational algebraic geometry -- particularly the theory
and technology of Gr\"obner bases -- provides a mechanism to answer
the polynomial unsat core question. The given set of polynomials $F$
generates an ideal, and the celebrated Hilbert's Nullstellensatz
\cite{ideals:book} provides all the tools to characterize the zero-set
(varieties) of polynomial ideals. Most Nullstellensatz-related
polynomial decision questions can be answered by the Gr\"obner basis
algorithm \cite{gb_book}. This motivates our investigation into
extraction of infeasible cores of the set of polynomials $F$ using
Gr\"obner basis techniques.



\subsubsection{Previous Work:} 


Moreover, as Gr\"obner basis techniques
find more applications in SAT solving \cite{CEI:stoc-96} \cite{condrat-tacas07}
\cite{zengler:GB-SAT}, unsat cores of polynomial systems will also find


in hardware and software verification
\cite{lv:tcad2013}, as well as 

