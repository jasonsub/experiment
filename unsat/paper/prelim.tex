\section{Preliminaries}

We denote by $C$ the set of clauses representing the CNF-SAT
problem. The problem is assumed to be unsat, and $C_c$ denotes
the set of clauses that constitute the unsat core. 

Let $\F$ be any field, $\Fc$ its algebraic closure, and $R = \R$ 
the polynomial ring in variables $x_1, \dots, x_d$ with coefficients
from $\F$. A monomial in
variables $x_1, \dots, x_d$  is a power product of the form  $X =
x_1^{e_{1}}\cdot x_2^{e_{2}}\cdots x_d^{e_{d}}$, where 
$e_i \in \Z_{\geq 0}, i\in \{1, \dots, d\}$. A {\it polynomial} 
$f \in R$ is written as a finite sum of terms 
$f = c_1 X_1 + c_2 X_2 + \dots + c_t X_t$.  Here $c_1, \dots, c_t$ are
coefficients and $X_1, \dots, X_t$ are monomials. To systematically
manipulate the polynomials, a monomial order $>$ (or a term order) is
imposed on the ring --- i.e. a total order on the monomials
s.t. multiplication with another monomial preserves the ordering.
The monomials of $f = c_1 X_1 + c_2 X_2 + \dots + c_t X_t$  are
ordered w.r.t. to $>$, such that  $X_1 > X_2 > \dots >  X_t$.  Subject
to such a term order, $lt(f) = c_1 X_1, ~lm(f) = X_1, ~lc(f) = c_1$,
are the {\it leading term}, {\it leading monomial} and {\it   leading
  coefficient} of $f$, respectively. 
%We also denote tail($f$) = $f - lt(f) = c_2X_2 + \dots + c_t X_t$. 
While our approach works for any permissible term order, in the paper,
we consider terms ordered {\it degree-lexicographically}, where the
monomials are ordered according to their descending total degree, and
ties are broken lexicographically.  

{\bf Polynomial Reduction:} Let $f, g$ be polynomials. If a non-zero
term $cX$ of $f$ is divisible by the leading term of $g$, then we say
that $f$ {\it is reducible to} $r$ modulo $g$, denoted $f
\stackrel{g}{\textstyle\longrightarrow} r$, where $r = f - {cX   \over
  lt(g)} \cdot g$. Similarly, $f$ can be {\it reduced (divided) 
w.r.t. a set of polynomials}  $F = \{f_1, \dots, f_s\}$ to obtain a
remainder $r$. This reduction is denoted $f \stackrel{F} {\textstyle
  \longrightarrow}_+ r$, and the remainder $r$ has the property that
no term in $r$ is divisible by the leading term of any polynomial
$f_i$ in $F$. The division algorithm ({\it e.g.} Alg. ??? \ref{alg:div})
records the {\it data} related to both the remainders and the quotients in
the division process. We will utilize this data to identify the core. 

\subsubsection{Ideals, Varieties and Nullstellensatz:} 
Let $F = \{f_1, \dots, f_s\}$ denote the given set of polynomials. An
{\it ideal} $J \subseteq R$ generated by polynomials $f_1, \dots, f_s \in R$
is: \vspace{-0.1in} 
$$J = \langle f_1, \dots, f_s \rangle = \{\sum_{i=1}^{s} h_i\cdot f_i:
~h_i \in \F[x_1,\dots,   x_d]\}.$$  %\vspace{-0.1in}
The polynomials $f_1, \dots, f_s$ form the {\it basis} or the {\it
  generators} of $J$.  

We have to consider the set of solutions to the system of polynomials
equations $f_1 = \dots = f_s = 0$. The set of all solutions to a
given system of polynomial equations $f_1 = \dots = f_s = 0$ is called
the {\it variety}, which is uniquely defined by the ideal 
$J = \langle f_1,\dots,f_s\rangle$ generated by
the polynomials. The variety is denoted by $V(J)=V(f_1,\dots,f_s)$ and
defined as: 
$$V(J) = V(f_1, \dots, f_s) = \{\mathbf{a} \in \F^d: \forall
f \in J, f(\mathbf{a}) = 0\},$$ 
where $\mathbf{a} = (a_1, \dots, a_d) \in \F ^d$ denotes a point in
the affine space.

\begin{theorem} \label{thm:weak-ns}
{\it The Weak Nullstellensatz} \cite{ideals:book}: 
%Let $\F$ be a field and $\Fc$ be its algebraic closure. 
Let $J = \langle f_1,\dots, f_s \rangle \subseteq \F[x_1, \dots, x_d]$ be
an ideal and $V_{\Fc}(J)$ be its variety over $\Fc$. Then
$V_{\Fc}(J) = \emptyset \iff J = \F[x_1,\dots,x_d] \iff 1 \in J$. 
\end{theorem}

The Weak Nullstellensatz provides a mechanism to ascertain whether or not
a given system of polynomials has a feasible solution. This is deduced
by testing whether the unit element is a member of the ideal $J$. This
{\it ideal membership test} requires the computation of a Gr\"obner
basis. 

\subsubsection{Gr\"obner bases:} An ideal $J$ may have many bases
(generators). A {\it Gr\"obner basis} (abbreviated as GB) is a generating set for $J$
with special properties that allow to solve many polynomial decision
problems, including ideal membership testing. From among the many
{\it equivalent} definitions of a Gr\"obner basis, we will consider the
following two:

\begin{definition}
\label{def:gb}
$\bf{\left[Gr\ddot{o}bner\ Basis\right]}$ \cite{gb_book}:\\
(i) For a monomial ordering $>$, a set  of non-zero polynomials $G =
\{g_1,g_2,\cdots,g_t\}$ contained in an ideal $J$, is called a
Gr\"{o}bner basis for $J$ iff $\forall f \in J$, $f\neq 0$ there
exists $i \in \{1,\cdots, t\}$ such that $lm(g_i)$ divides $lm(f)$;
i.e., $G = GB(J) \Leftrightarrow\  \forall f \in J : f \neq 0, \exists
g_i \in G : lm(g_i)\mid lm(f)$. Or equivalently, \\
(ii) $G = \{g_1,g_2,\cdots,g_t\}$ is a Gr\"{o}bner basis for $J$ iff
division by $G$ of any polynomial in $J$ gives the remainder 0;
i.e. $G = GB(J) \iff \forall f \in J, f\xrightarrow{g_1,\dots,g_t}_+ 0$.
\end{definition}

Buchberger's algorithm 
\cite{buchberger_thesis},
shown in Algorithm \ref{alg:gb},  computes a Gr\"obner
basis over a field. Given polynomials $F = \{f_1, \dots, f_s\}$, 
the algorithm computes the Gr\"obner basis $G = \{g_1, \dots, 
g_t\}$. 
% such that ideal $I = \langle f_1, \dots, f_s\rangle = \langle g_1,
% \dots, g_t\rangle$. 
The algorithm takes pairs of polynomials $(f, g)$, and computes their
{\it S-polynomial} ($Spoly(f, g)$): 

\begin{equation}
       Spoly(f,g)=\frac{L}{lt(f)}\cdot f - \frac{L}{lt(g)}\cdot g \nonumber
      \label{eqn:spoly}
\end{equation}
where $L = \text{LCM}(lm(f), lm(g))$. 
%, where $lm(f)$ is the leading monomial of $f$, and $lt(f)$ is the
%leading term of $f$. 
$Spoly(f,g)$ cancels the leading terms of $f$ and $g$. Therefore, the 
computation $Spoly(f,g) \xrightarrow{G'}_+r$ results in a remainder $r$,
which if non-zero, provides an element with new leading term in the
generating set.  The Gr\"obner basis algorithm terminates when for all pairs
$(f, g)$, $Spoly(f,g)\xrightarrow{G'}_+0$. 

\begin{algorithm}[hbt]
\SetAlgoNoLine
 \KwIn{$F = \{f_1, \dots, f_s\}$}
 \KwOut{$G = \{g_1,\dots ,g_t\}$\\} %, a Gr\"{o}bner basis
  $G:= F$\;
  \Repeat{$G = G'$}
  {
        $G' := G$\;
        \For{ each pair $\{f, g\}, f \neq g$ in $G'$} 
        {
                $Spoly(f, g) \stackrel{G'}{\textstyle\longrightarrow}_+r$ \;
                \If{$r \neq 0$}
                {
                        $G:= G \cup \{r\}$ \;
                }
        }
   }
\caption {Buchberger's Algorithm}\label{alg:gb}
\end{algorithm}

A Gr\"obner basis $G$ may contain redundant elements. To remove
these redundant elements, $G$ is first made {\it minimal} and
subsequently {\it reduced.} To obtain a minimal GB, all polynomials
$g_j$ are removed from $G$ if there exists a $g_i$ such that 
$lm(g_i) ~|~ lm(g_j)$. Then the remaining elements ($g_i$'s) are made
monic by dividing each $g_i$ by $lc(g_i)$. This minimal basis is
further reduced by ensuring that no term in $g_j$ is divisible by the
leading term $lt(g_i)$ for all $i \neq j$. Subject to %the given term order 
$>$, the reduced Gr\"obner   basis $G_r = \{g_1, \dots, g_t\}$ is a
unique, canonical representation of ideal $J$. 

In the context of Nullstellensatz, $V_{\Fc}(J) = \emptyset \iff G_r =
\{1\}$. This implies that for ideals with empty variety, there exists
a sequence of $Spoly(f, g)$ reductions in the reduced GB computation
that leads to the unit element. We now show that by analyzing this
data, the unsat core $F_c$ of the given generating set $F =
\{f_1,\dots,f_s\}$ can be identified. 


