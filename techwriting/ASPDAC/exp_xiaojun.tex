\section{Experimental Results}
\label{sec:result}
We have implemented our approach within the \textsc{Singular} symbolic
algebra computation system [v. 3-1-6] \cite{DGPS}. Using our
implementation, we have performed experiments to verify two SMPO
architectures \cite{agnew1991implementation} \cite{RHmulti} over
$\Fkk$, for various datapath/field sizes. Bugs are also introduced
into the SMPO designs by modifying a few gates in the combinational
logic block. Experiments using SAT-, BDD-, and AIG-based solvers are
also conducted and results are compared against our approach. 
Our experiments run on a desktop with 3.5GHz Intel
$\text{Core}^\text{TM}$ i7 Quad-core CPU, 16 GB RAM and 64-bit Linux.  

{\it Evaluation of SAT/ABC/BDD based methods:} 
For Angew's SMPO
design, there is an equation in following form (this is a 5-bit
example): 
\begin{align}
R_i &= b_ia_{i+1} + b_{i+1}(a_i + a_{i+3}) + b_{i+2}(a_{i+3} + a_{i+4}) \nonumber\\
&+ b_{i+3}(a_{i+1} + a_{i+2}) + b_{i+4}
(a_{i+2} + a_{i+4}),\ 0\leq i\leq 4 \nonumber
\end{align}
This equation can be used as specification of any unrolled SMPO circuit. For SAT/ABC/BDD based methods,
we first unroll our RH-SMPO design, then create a \emph{miter} with the unrolled design and
circuit specification, which we use unrolled Angew's SMPO in this paper, to check if it is unsatisfiable.

The SAT solver we use is \emph{Lingeling} \cite{biere2013lingeling}. Single-output miter circuit is written
in DIMACS CNF \cite{DIMACSformat} format and fed to Lingeling. In ABC, we use the Combinational Equivalence Checking (CEC) engine 
 to read and compare 2 networks from unrolled RH-SMPO and specification (both written in BLIF format). 
 For BDD based approach, we use the BDD module integrated within VIS tool \cite{brayton1996vis}. The miter is described by a BLIF 
 file and read by VIS. After BDD is built, if there is no path to leaf "1", then the miter is unsatisfiable.

The runtime results given in seconds include experiments for 11, 18, 23 and 33 bits unrolled SMPO
and their specifications. They show that all SAT, ABC and BDD based approach 
cannot verify SMPO beyond 33 bits.

\begin{table}[tb]
\centering
\caption{Runtime for verification of bug-free SMPO circuits over $\Fkk$ for SAT, ABC and BDD based methods. \emph{TO} = timeout of 14 hrs}
\begin{tabular}{|c||c|c|c|c|} 
\hline
& \multicolumn{4}{|c|}{Word size of the operands $k$-bits}  \\
\hline
Solver & 11 & 18 & 23 & 33 \\
\hline
\hline
Lingeling & 593  & \emph{TO}  & \emph{TO}  & \emph{TO}\\
\hline
\hline
ABC & 6.24 & \emph{TO} & \emph{TO} & \emph{TO}\\
\hline
\hline
BDD & 0.1 & 11.7 & 1002.4 & \emph{TO}  \\
\hline
\end{tabular}
\label{table:satbdd}  
\end{table} 

The CEC engine embedded within ABC tool \cite{brayton2010abc} combines multiple techniques such as AIG rewriting via structural hashing, simulation and
SAT and has high efficiency when verifying structurally similar netlists, esp. comparing original design with its synthesized and optimized implementation.
However, the RH-SMPO we verify has distinct structure from our specification (Angew's SMPO). We can approximately give the similarity between these 2 designs
using information from networks before and after structural hashing. First, we read in the miter design created from unrolled RH-SMPO and specification,
record the original number of nodes in this network as $N_1$; then we use command \emph{fraig\_sweep} in ABC to reduce this network, equivalent nodes in AIG
are merged, record the number of nodes after reduction as $N_2$; $\frac{N_1-N_2}{N_1}$ reflects the portion of equivalent nodes in original miter, which is
the \emph{similarity} between two designs.

\begin{table}[tb]
\centering
\caption{Similarity between RH-SMPO and Angew's SMPO}
\begin{tabular}{|c||c|c|c|c|} 
\hline
Size $k$ & 11 & 18 & 23 & 33 \\
\hline
$N_1$ & 734  & 2011  & 3285  & 6723\\
\hline
$N_2$ & 529 & 1450 & 2347 & 4852\\
\hline
Similarity & $27.9\%$ & $27.9\%$ & $28.6\%$ & $27.8\%$  \\
\hline
\end{tabular}\label{table:similarity}  
\end{table} 

\subsection{Evaluation of Our Approach}
Our approach requires only one time reduction for each clock cycle using RATO. If we use built-in function in \textsc{Singular} to directly
compute Gr\"obner basis, the high-degree word definition polynomial will exceed \textsc{Singular}'s capacity, even for 11 bits SMPO.

We also introduce bugs to our SMPO designs; these bugs are simply swapping an arbitrary pair of gate output signals. Since our approach does not
rely on arithmetic complexity of the design, the runtime for verifying buggy circuits has almost no difference from bug-free ones.

\begin{table}[tb]
\centering
\caption{Runtime (given in seconds) for verification of bug-free and buggy RH-SMPO circuits over $\Fkk$ using our approach}
\begin{minipage}{8cm}
\def\arraystretch{1.5}\tabcolsep 2pt
\begin{tabular}{|c||c|c|c|c|c|} 
\hline
Operand size $k$ & 33 & 51 & 65 & 81 & 89 \\
\hline
\#variables & 4785 & 11424 & 18265 & 28512 & 34354\\
\hline
\#polynomials & 3630 & 8721 & 13910 & 21789 & 26255\\
\hline
\#terms & 13629 & 32793 & 52845 & 82539 & 99591 \\
\hline
\hline
Runtime(bug-free) & 112.6 & 1129 & 5243 & 20724 & 36096 \\
\hline
Runtime(buggy) & 112.7 & 1129 & 5256 & 20684 & 36120\\
\hline
\end{tabular}
\end{minipage}
\label{table:SMPO}  
\end{table}

%\begin{table}[h]
%\centering
%\begin{tabular}{|c||c|c|c|c|c|} 
%\hline
%Operand size $k$ & 36 & 66 & 82 & 89 & 100 \\
%\hline
%\#variables & 183 & 333 & 413 & 448 & 503\\
%\hline
%\#polynomials & 2700 & 8910 & 13694 & 16109 & 20300\\
%\hline
%\#terms & 12996 & 43626 & 67322 & 79299 & 100100 \\
%\hline
%\hline
%Runtime(bug-free) & 113 & 3673 & 15117 & 28986 & 50692 \\
%\hline
%Runtime(buggy) & 118 & 4320 & 15226 & 31571 & 58861\\
%\hline
%\end{tabular}
%\caption{Runtime (given in seconds) for verification of bug-free and buggy Angew's SMPO circuits over $\Fkk$ using our approach}\label{table:SMPO}  
%\end{table}
