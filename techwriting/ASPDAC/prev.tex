\section{Review of Previous Work}
\label{sec:prev}

Verification of a combinational GF arithmetic circuit $C$ against a
polynomial specification $\F$ has been previously addressed \cite{ibm:blueveri}
\cite{lv:tcad2013} \cite{pruss:dac14}. Verification problems in
\cite{ibm:blueveri} \cite{lv:tcad2013} are formulated using
Nullstellensatz and decided using the \Grobner basis algorithm.
%% Moreover, in \cite{lv:tcad2013} the authors show that by analyzing the
%% given circuit $C$, a term order ($>_1$) can be derived that renders
%% the set of polynomials itself a \Grobner basis. Consequently, a
%% \Grobner basis computation is not required, and the verification test
%% can be performed simply by dividing the specification polynomial $\F$
%% by polynomials of $C$ (when represented using $>_1$). 

The paper 
\cite{pruss:dac14} performs verification by deriving a canonical
word-level polynomial representation $\F$ from the circuit $C$. Their
approach views any arbitrary Boolean function (circuit) $f: \B^k
\rightarrow \B^k$ as a polynomial function $f: \Fkk \rightarrow \Fkk$,
and derives a canonical polynomial representation $\F$ over
$\Fkk$. They show that this can be achieved by computing a reduced 
\Grobner basis w.r.t. an {\it abstraction term order} derived from the
circuit. Subsequently, they propose a \underline{r}efinement of this
\underline{a}bstraction \underline{t}erm \underline{o}rder (called
RATO), that enables to compute the \Grobner basis of a smaller subset
of polynomials. The authors show that their approach can prove
correctness of up to 571-bit combinational GF multipliers. 

Since we are also interested in deriving a polynomial representation
of the computation performed by the sequential circuit $S$, we draw
inspirations from \cite{pruss:dac14} -- particularly the use of RATO
-- for sequential verification. However, the approach of
\cite{pruss:dac14} suffers from a few limitations: i) When the GF
polynomial implemented by the circuit is dense (say,  due to the
presence of a bug in the design), their approach is computationally
infeasible; ii) The use of RATO still requires a \Grobner basis
computation (even though on a subset of polynomials) to derive the
polynomial $\F$, which can lead to a memory explosion. Experiments in
\cite{pruss:dac14} are only successful for hierarchically designed and
bug-free GF circuits. {\it While we do employ RATO as the term order
  for sequential verification, we further present an efficient
  symbolic computation approach that does not suffer from these
  limitations}. 

The problem addressed in this paper is not suitable to be
solved by conventional bit-level sequential equivalence
\cite{coudert:iccad90} or (bounded) model checking frameworks
based on interpolation \cite{mcmillan:interpolation_cav03} or property
directed reachability \cite{IC3}. This is mostly due to the word-level
and GF polynomial nature of the specification (property) $\F$, which 
is also only valid in one state of the machine. The use of algebraic
geometry has been proposed for model checking 
\cite{Avrunin:CAV} \cite{Vardi:gb_bool_ring} \cite{polybori:2009};
however, these approaches are a straight-forward application of {\it
  bit-level} Boolean \Grobner basis engines in lieu of binary decision diagrams (BDDs) or 
  satisfiability (SAT)
solvers.  
%computations to the model checking algorithm. 

