\section{Improving our Approach}
\label{sec:improve}

Computing \Grobner bases with elimination orders is infeasible for
large circuits; the complexity of Buchberger's algorithm to compute
$GB(J + J_0)$ in $\Fq$ is $q^{O(d)}$ \cite{lv:tcad2013}. 
To overcome this complexity, \cite{pruss:dac14} proposed a
\underline{r}efinement of ATO (called RATO), and simplified the
\Grobner basis computation. They exploited Buchberger's product
criteria \cite{productc:1979}, which states that: {\it If the leading  
monomials of $f_i, f_j$ are relatively prime, then $Spoly(f_i, f_j)$
reduces to zero modulo the generating set, i.e. $Spoly(f_i, f_j)
\xrightarrow{F} _+ 0$.} This concept was exploited in RATO as follows: 

Perform a reverse topological sorting of the nodes in the
combinational logic, and define a {\it lex} term order by the
following relation $>_{r}$: {\it bit-level circuit variables ordered
  reverse topologically} $>$  {\it word-level output variables} $>$
{\it word-level input   variables}. Representing the polynomial ideal
$J$ in RATO has the effect that there exists {\it one and only one
  pair of polynomials} in $J$ that do not have relatively prime
leading terms (see  Section 5 in \cite{pruss:dac14} for details). All
other polynomial pairs will have leading terms that are 
relatively prime, so these polynomial pairs are not considered in
Buchberger's algorithm.  The authors of \cite{pruss:dac14} exploited
this concept and showed how the \Grobner basis of $(J + J_0)$ can be
computed by a {\it subset} of polynomials, which  improves the
scalability of their approach. Their approach, however, cannot
circumvent the \Grobner basis computation altogether. Consequently, 
their approach fails to derive a canonical polynomial abstraction when
the representation is dense, and contains monomials of high-degrees
(e.g. in case of buggy designs). 

It turns out that RATO can be applied to sequential circuits in much
the same way: {\it \{bit-level circuit variables ordered
  reverse topologically} $x_1 > \dots x_d\}$ $>$  {\it \{word-level NS
  variables} $R'> A'> B'\}$ $>$ {\it \{word-level PS variables}
$R>A>B\}$. Importantly, {\it we show that using RATO, the polynomial
abstraction can be derived without resorting to a \Grobner basis
computation.} Perform the following operations: 

\begin{enumerate}
\item Represent the polynomials of the sequential circuit $S$ using RATO.
\item Due to RATO, only one pair of polynomials ($f_i, f_j$) will have
  leading terms that will not be relatively prime.
\item Reduce $Spoly(f_i, f_j) \xrightarrow{F}_+ h$.
\end{enumerate}
As described in \cite{pruss:dac14}, remainder $h$ will contain:
  i) the word-level variables, and ii) bit-level inputs to the
  combinational logic, i.e. bit-level present-state variables. All
  other internal circuit variables will not appear in $h$, as they
  will be canceled by division due to RATO. 

\begin{Example}
\label{ex:newRATO}
Let us re-visit Example \ref{ex:RHSMPO} and the RH-SMPO circuit of
Fig. \ref{fig:RHmulti}. For this circuit, the term order under
RATO is lex with: $\{r_0',r_1',r_2',r_3',r_4'\} > \{r_0, r_1, r_2,r_3,r_4\}
> \{e_0,e_3,e_4\}, \{d_0,d_1,d_2\}, \{c_1,c_2,c_3,c_4\} 
\{ a_0, a_1, a_2, a_3, a_4, \\
b_0, b_1, b_2,  b_3, b_4 \} > R'>R>\{A,B\}$. 


Among all the generators of the ideal $J$ from Ex.\ref{ex:RHSMPO},
using RATO \textbf{we find only one pair of polynomials} whose leading  
monomials are not relatively prime:
$(f_i, f_j), f_i = r_0'+r_4+e_0, f_j
=r_0'\alpha^5+r_1'\alpha^{10}+r_2'\alpha^{20}+r_3'\alpha^9+r_4'\alpha^{18}
+ R'$. Then: %Buchberger's algorithm will compute $Spoly(f_w,
\vspace{-0.1in}
{\small
\begin{align}
&Spoly(f_i,f_j) \xrightarrow{J+J_0}_{+}\nonumber\\
&(\alpha^3+\alpha^2+\alpha) r_1+(\alpha^4+\alpha^3+\alpha^2) r_2+(\alpha^2+\alpha) r_3+(\alpha) r_4\nonumber\\
&+(\alpha^3+\alpha^2) a_1 b_1+(\alpha^4+\alpha^3+\alpha^2+\alpha) a_1 b_2+(\alpha^2+\alpha) a_1 b_3\nonumber\\
&+(\alpha^2+1) a_1 b_4+(\alpha^4+1) a_1 B+(\alpha^4+\alpha) a_2 b_1+(\alpha^4+\alpha^3+\alpha) a_2 b_2\nonumber\\
&+(\alpha^3+1) a_2 b_3+(\alpha^3+\alpha^2+1) a_2 b_4+(\alpha^3+\alpha^2) a_2 B+(\alpha^2+\alpha) a_3 b_1\nonumber\\
&+(\alpha^3+1) a_3 b_2+(\alpha+1) a_3 b_3+(\alpha^4+\alpha^2+\alpha) a_3 b_4\nonumber\\
&+(\alpha^4+\alpha^3+\alpha) a_3 B+(\alpha^3+1) a_4 b_1+a_4 b_2+(\alpha^4+\alpha^2+\alpha) a_4 b_3\nonumber\\
&+(\alpha^4+\alpha^3+1) a_4 b_4+(\alpha^2+\alpha) a_4 B+(\alpha^4+1) b_1 A+(\alpha^3+\alpha^2) b_2 A\nonumber\\
&+(\alpha^4+\alpha^3+\alpha) b_3 A+(\alpha^2+\alpha) b_4 A+(\alpha^4+\alpha^2+\alpha+1) R'+A B\nonumber
\end{align}
}
\end{Example}
\vspace{-0.1in}
As shown above, the remainder $h$ contains both bit-level variables and
word-level {\it state variables} $a_i, b_i, r_i, R, A, B$. We desire
a polynomial in only word 
level variables (e.g. $R'+\mathcal{F}(A,B)$), without computing a
\Grobner basis. This can be achieved if we  represent the
bit-level state variables $a_i, b_i, r_i$ in terms of their word-level
variables $A, B, R$, respectively. This can be achieved due to 
the following property of finite fields:

\begin{Lemma}
\label{lemma:square}
For $\alpha_1, \dots, \alpha_t \in \Fkk$, $(\alpha_1 + \alpha_2 +
\dots + \alpha_t)^{2^i} = \alpha_1^{2^i} + \alpha_2^{2^i} + \dots +
\alpha_t^{2^i}$, for $i = 1, 2, \ldots$. 
\end{Lemma}

Consider the polynomials that define the relationship between the
word-level and bit-level variables. Let 
$ A = a_0 \beta + a_1\beta^2 + \dots + a_{k-1}\beta^{2^{k-1}}$.
Due to Lemma \ref{lemma:square}, we have $A^2 = a_0 \beta^{2}+ a_1
\beta^{4} + \dots + a_{k-1} \beta^{2\cdot 2^{k-1}}$ (as
$a_i^2 = a_i$). By repeated squaring:
\begin{eqnarray}
A & = & a_0 \beta + a_1\beta^2 + \dots + a_{k-1}\beta^{2^{k-1}}\nonumber\\
A^2 & = & a_0 \beta^{2}+ a_1 \beta^{4} + \dots + a_{k-1}
\beta^{2\cdot 2^{k-1}} \nonumber\\
\vdots & \vdots& \vdots \nonumber\\
A^{2^{k-1}} & = & a_0 \beta^{2^{k-1}}  +  a_1 \beta^{2^{k-1}\cdot 2} +
\dots + a_{k-1}\beta^{2^{2(k-1)}} \nonumber
\end{eqnarray}
Consider the above as $k$ {\it linear equations} with $a_0, \dots, a_{k-1}$
as $k$-unknowns, with $\beta$ and its powers as coefficients, and 
$A, A^2, ... \dots, A^{2^{k-1}}$ as $k$ constants. Then, we can solve
for the unknowns $a_0, \dots, a_{k-1}$ and obtain expressions for them
in terms of $A$ and $\beta$. The problem can be setup in
matrix form:

\begin{displaymath}
\begin{pmatrix}
\beta & \beta^2 & \cdots & \beta^{2^{k-1}} \\
\beta^2 & \beta^4 & \cdots & \beta^{2^{k-1}\cdot 2} \\
\vdots & \vdots & \ddots & \vdots \\
\beta^{2^{k-1}} & \beta^{2^{k-1}\cdot 2} & \cdots & \beta^{2^{2(k-1)}}
\end{pmatrix}
\begin{pmatrix}
a_0\\
a_1\\
\vdots\\
a_{k-1}
\end{pmatrix}
=
\begin{pmatrix}
A\\
A^2\\
\vdots\\
A^{2^{k-1}}
\end{pmatrix}
\end{displaymath}

Gaussian elimination on this system of equations can be 
applied, and each bit-level variable $a_i$ can be represented as a
function of the word-level variables: $a_i = g_i(A)$. These bit-level
variables can be easily substituted in the remainder $h$ obtained by
reduction due to RATO. What we will obtain is a word-level polynomial
of the form $h: R' + \F(A, B)$ which canonically represents the
$k$-cycle function of the circuit. It is also easy to show that this
polynomial is a unique, canonical representation because it is reduced
modulo $J + J_0$. 
% = \langle x_1^2 - x_1, \dots, x_d^2 - x_d, R^q - R, A^q -
%A, B^q - B\rangle$. 

%% By replacing all bit-level variables by corresponding word-level
%% variable polynomials, we transform the remainder of $Spoly$ reduction
%% to the form of $R'+R+\mathcal F'(A,B)$. Note $R$ is present state
%% notion of output, which equals to initial value $R=0$ in first clock
%% cycle, or value of $R'$ from last clock cycle. By substituting $R$
%% with its corresponding value ($0$ or a polynomial only about $A$ and
%% $B$), we get the desired polynomial function $R'+\mathcal F(A,B)$. 
 
