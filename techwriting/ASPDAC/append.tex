\appendix

\subsection{Abstract Algebra Concepts}
Let $q = 2^k$ and $\mathbb F_{q}[x_1,\dots, x_d]$ be the polynomial
ring with indeterminates $x_1,\dots,x_d$. A polynomial $f = c_1 X_1 +
c_2 X_2 + \dots + c_t X_t$ is a finite sum of terms, where $c_1,
\dots, c_t$ are coefficients and $X_1, \dots, X_t$ are monomials. A
monomial ordering $X_1 > X_2 \dots > X_t$ is imposed on the
polynomials to process them systematically. Then, $LT(f) = c_1 X_1,
LM(f) = X_1$ denote the leading term and the leading monomial of $f$,
respectively. Given polynomials $f, g$, if $cX$ is a term in $f$ that
is divisible by $LT(g)$, 
then $f \xrightarrow{g} r$ denotes a one-step reduction (division) of
$f$ by $g$, resulting in remainder $r = f - {{cX} \over {LT(g)}} \cdot
g$. Similarly, $f$ reduces to $r$ modulo the set of polynomials $F =
\{f_1, \dots, f_s\}$, denoted $f \stackrel{F} {\textstyle
  \longrightarrow}_+ r$, such that no term in $r$ is divisible by the
$LT(f_i)$ of any polynomial in $ f_i \in F$.   

Let $f_1,\dots, f_s$ be polynomials in $\mathbb F_q[x_1,\dots,x_d]$,
then we set $J = \langle f_1,\dots, f_s\rangle = \left\{\sum_{i=1}^s
h_if_i:h_1,\dots,h_s \in \mathbb F_q[x_1,\dots,x_d]\right\}$.  $J =
\langle f_1,\dots, f_s\rangle$ forms an ideal in $\mathbb
F_{q}[x_1,\dots, x_d]$, and $f_1,\dots, f_s$ are its generators.  
Let $\mathbf{a} = (a_1, \dots, a_d) \in \Fq^d$ be a point. For any
ideal $J = \langle f_1, \dots, f_s \rangle \subseteq 
\Fq[x_1,\dots, x_d]$, the {\it affine variety} of $J$ over
$\Fq$ is: 
$V(J) = \{\mathbf{a} \in \Fq^d: \forall f \in
J, f(\mathbf{a}) = 0\}.$ The variety $V(J)$ corresponds to
the set of all solutions to $f_1 = \dots = f_s = 0$. Any finite set of
points can be construed as the variety of an ideal --- a concept we
exploit to model the  set of (finite) states of the machine. 

An ideal may have many generating sets. The set $G = \{g_1, \dots,
g_t\}$ is called a \Grobner basis of $J$ if and only if the
leading term of all polynomials in $J$ is divisible be the leading
term of some polynomial $g_i$ in $G$: i.e. $\forall f \in J, \exists
g_i \in G \ s.t. \ LT(g_i) ~|~ LT(f)$. The famous Buchberger's
algorithm, given in textbook \cite{ideals:book}, is used to compute a
\Grobner basis (GB). Operating on  input $F = \{f_1, \dots, f_s\}$,
and subject to the imposed term order $>$, it derives $G = GB(J) = \{
g_1, \dots, g_t \}$. Buchberger's algorithm repeatedly computes
$S$-polynomials. For pairs $(f_i, f_j) \in F$, $Spoly(f_i, f_j) =
\frac{L}{lt(f_i)}\cdot f_i - \frac{L}{lt(f_j)}\cdot f_j$, where $L =
LCM(LM(f_i), LM(f_j))$. Reducing $Spoly(f_i, f_j) \xrightarrow{F}_+ r$
cancels the leading terms of $f_i, f_j$ and gives a polynomial $r$
with a new leading term. This remainder $r$ is added to the current
basis and $Spoly(f_i, f_j)$ computations are repeated for all pairs of
polynomials until all $S$-polynomials reduce to 0.  

\subsection{Normal Basis Theory}
Let $\beta$ be a element in the Galois field $F_{2^n}$ constructed by primitive element $\alpha$ and minimal polynomial 
$f(\alpha)$. Then a basis in the form $\{\beta, \beta^2, \beta^4, \beta^8, ... ,\beta^{2^{n-1}}\}$ is a
\emph{Normal Basis}; here $\beta$ is called \emph{Normal Element}.\\

For arithmetic in Galois fields, multiplication (with modulus) can be greatly simplified if Normal Basis representation 
is adopted to represent operands, especially in element square and multiplication:\\

\textit{Example B.1:}\ \ Element square: In $F_{2^n}$, we have relation $(a+b)^2 = a^2 + b^2$. Then 
$(b_0\beta + b_1\beta^2 + b_2\beta^4 + \dots + b_{n-1}\beta^{2^{n-1}})^2 = 
b_0^2\beta^2 + b_1^2\beta^4 + b_2^2\beta^8 + \dots + b_{n-1}^2\beta = 
b_{n-1}^2\beta + b_0\beta^2 + b_1\beta^4 + \dots + b_{n-2}\beta^{2^{n-1}}$, which means a simple right-cyclic rotation.
One the other hand, standard basis representation do not have this benefit:\\
In $F_{2^3}$ constructed by $\alpha^3 + \alpha + 1$, standard basis is $\{ 1, \alpha, \alpha^2\}$.
Let $\beta = \alpha^3$, $N = \{ \beta, \beta^2, \beta^4\}$ is a normal basis. Write down an element with both representations:
$E = a_0 + a_1\alpha + a_2\alpha^2 = b_0\beta + b_1\beta^2 + b_2\beta^4$, and its square in standard basis:
$E^2 = a_0 + a_1\alpha^2 + a_2\alpha^4 = a_0 + a_2\alpha + (a_1 + a_2)\alpha^2$. When it is written in normal basis:
$E^2 = b_2\beta + b_0\beta^2 + b_1\beta^4$.

\textit{Example B.2:}\ \ Normal basis multiplication: Assume there are 2 binary vectors representing 2 operands in normal
basis: $A = (a_0, a_1, \dots, a_{n-1}), B = (b_0, b_1, \dots, b_{n-1})$; similarly the product can also be written
as: $C = A*B = (c_0, c_1, \dots, c_{n-1}).$\\
Then the highest digit of product can be represented by a function: $c_{n-1} = f(a_0, a_1, \dots, a_{n-1}; b_0, b_1, 
\dots, b_{n-1})$. Square both side: $C^2 = A^2*B^2$, i.e. the second highest digit $c_{n-2} = f(a_{n-1}, a_0, a_1, 
\dots, a_{n-2}; b_{n-1}, b_0, b_1, \dots, b_{n-2}).$ By this method it is easy to get all digits of product $C$.\\

\textit{Example B.3:}\ \ $\lambda$-Matrix: We can employ a binary $n\times n$ matrix $M$ to describe the "function"
 mentioned above: $c_{n-1} = f(A, B) = A \cdot M \cdot B^T$, $B^T$ denotes vector transposition. 
More specifically, we denote the matrix by \emph{$k$-th $\lambda$-Matrix}: $c_k = A \cdot M^{(k)} \cdot B^T$.
Then $c_{k-1} = A \cdot M^{(k-1)} \cdot B^T = rotate(A) \cdot M^{(k)} \cdot rotate(B)^T$, which means
by right and down shifting $M^{(k-1)}$ we can get $M^{(k)}$.\\
In $F_{2^3}$ constructed by $\alpha^3 + \alpha + 1$, let $\beta = \alpha^3$, $N = \{ \beta, \beta^2, \beta^4\}$ 
is a normal basis. $0$-th $\lambda$-Matrix
\begin{equation}
M^{(0)} = \left(
\begin{array} {lcr}
0 & 1 & 0\\
1 & 0 & 1\\
0 & 1 & 1
\end{array} \right).
\end{equation}
i.e.,
\begin{equation}
c_0 = (a_0\  a_1\  a_2)\left(
\begin{array} {lcr}
0 & 1 & 0\\
1 & 0 & 1\\
0 & 1 & 1
\end{array} \right)\left(
\begin{array} {lcr}
b_0\\
b_1\\
b_2
\end{array} \right).
\end{equation}

\indent \\
$\lambda$-Matrix is defined with cross-product terms from multiplication, which is 
\begin{equation}
Product C = (\sum_{i=0}^{n-1}a_i\beta^{2^i})(\sum_{j=0}^{n-1}b_j\beta^{2^j}) = \sum_{i=0}^{n-1}\sum_{j=0}^{n-1}a_ib_j\beta^{2^i}\beta^{2^j}
\end{equation}
The expressions $\beta^{2^i}\beta^{2^j}$ are referred to as cross-product terms, and can be represented by
normal basis, i.e.
\begin{equation}
\beta^{2^i}\beta^{2^j} = \sum_{k=0}^{n-1}\lambda_{ij}^{(k)}\beta^{2^k}, \ \ \lambda_{ij}^{(k)} \in F_2.
\end{equation}
Substitution yields, result is an expression for k-th digit of product as showed in \textit{Example B.3}:
\begin{equation}
c_k = \sum_{i=0}^{n-1}\sum_{j=0}^{n-1}\lambda_{ij}^{(k)}a_ib_j
\end{equation}
$\lambda_{ij}^{(k)}$ is the entry with coordinate $(i,j)$ in $k$-th $\lambda$-Matrix.\par

\subsection{Optimal Normal Basis (ONB)}

The number of non-zero entries in $\lambda$-Matrix is named as  \emph{Complexity$(C_N)$}. \\
To define optimal normal basis, it is necessary to find out the minimum number
of non-zero terms in $\lambda$-matrix.\\
\textit{Theorem C.1}\ \ If N is a normal basis for $F_{p^n}$ with $\lambda$-matrix $M^{(k)}$, then non-zero entries in 
matrix $C_N\geq 2n-1$.\\
\begin{proof}
Let $N = \{\beta, \beta^p, \beta^{p^2},\dots, \beta^{p^{n-1}}\}$, denote $\beta^{p^i}$ by $\beta_i$. 
Then $\sum_{i=0}^{n-1} \beta_i = trace\ \beta$. \\
Denote $trace\ \beta$ with $b$, consider $n\times n$ matrix $M^{(0)}$. Then\\
$b\beta_0 = \sum_{i=0}^{n-1} \beta \beta_i$.\\
Therefore, the sum of all rows in $M^{(0)}$ is an n-tuple with $b$ as the first element and zeros elsewhere.
So the first column must contain at least 1 nonzero element, and other columns must get a sum of zero 
while keeping every row linearly independent so there should be at least 2 non-zero elements each column.
\end{proof}
If there exists a set of normal basis satisfying $C_N = 2n - 1$, this normal basis is named as
\emph{Optimal Normal Basis}(ONB).\\
%For $F_{2^n}$, easy to figure out the first row of multiplication table must contain one "1" because 
%$\beta \beta^{2^0} = \beta^{2^1}$. for the rest rows $\beta \beta^{2^i} \neq \beta^{2^{i+1}}$,
%so there must be at least 2 "1"s to fulfil the equation. So for multiplication table, $C_N\geq 2n-1$ also holds.\\

\textit{Example C.1:}\ \ In $F_{2^4}$ constructed with $\alpha^4 + \alpha + 1$, 2 sets of normal basis can be found:
$\beta = \alpha^3, N = \{ \alpha^3, \alpha^6, \alpha^{12}, \alpha^9\}$ and 
$\beta = \alpha^7, N = \{\alpha^7, \alpha^{14}, \alpha^{13}, \alpha^{11}\}$. Multiplication table for $\beta = \alpha^3$:
\begin{equation}
T_1 = \left(
\begin{array}{lccr}
0 & 1 & 0 & 0\\
0 & 0 & 0 & 1\\
1 & 1 & 1 & 1\\
0 & 0 & 1 & 0
\end{array} \right)
\end{equation}
Complexity $C_N$ = 7, this is optimal. For $\beta = \alpha^7$:
\begin{equation}
T_2 = \left(
\begin{array}{lccr}
0 & 1 & 0 & 0\\
1 & 1 & 0 & 1\\
1 & 0 & 1 & 0\\
1 & 0 & 1 & 1
\end{array} \right)
\end{equation}
Complexity $C_N$ = 9, this is NOT optimal.\\

There is an efficient method to construct optimal normal basis by transforming minimal polynomials based on
special property of optimal normal basis\cite{ECC}.

\subsection{Construction of Type-I Optimal Normal Basis}
Rules for constructing Type-I ONB in $F_{2^n}$ are:
\begin{itemize}
\item n+1 must be prime.
\item 2 must be primitive in $\mathbb{Z}_{n+1}$.
\end{itemize}
Second rule means 2 raise to any power 0 $\sim$ n-1 (mod n+1) must cover every integer 1 $\sim$ n.\\
$\lambda_{ij}^{(k)}$ is the entry with coordinate $(i,j)$ from k-th $\lambda$-Matrix. Then the crossproduct
term can be written as
\begin{equation}
\beta^{2^i}\beta^{2^j} = \sum_{k=0}^{n-1} \lambda_{ij}^{(k)} \beta^{2^k}
\end{equation}
Suppose we only care about k=0. So simplified to following equations:
\begin{numcases}{}
\beta^{2^i}\beta^{2^j} = \beta\\
\beta^{2^i}\beta^{2^j} = 1 \ \ (if\ \  2^i \equiv 2^j\ \  mod\ \  n+1)
\end{numcases}
solution $(i,j)$ implies entries equal to "1" in matrix. If $\beta$ is already an optimal normal element, 
$\beta^{2^i}$ counts through all powers of $\beta$ and generates our basis. So
\begin{numcases}{}
2^i + 2^j = 1 \ \ mod\ \  n+1\\
2^i + 2^j = 0 \ \ mod \ \ n+1
\end{numcases}
have the same solution for $M^{(0)}$. Note this method can be applied even both primitive polynomial and normal element 
are unkown (but existance and adoption of optimal normal basis must be guaranteed).\\
By repeating above steps, equations for multiplication table can also be obtained. Using this multiplication table, 
a transformation relation 
between $\lambda$-Matrix and Multiplication table can be concluded.
Example C.1 shows a type-I ONB in $F_{2^4}$.


	\subsection{Construction of Type-II Optimal Normal Basis}
Rules for constructing Type-II ONB over $F_{2^n}$ are:
\begin{itemize}
\item 2n+1 must be prime. And either
\item 2 must be primitive in $\mathbb{Z}_{2n+1}$, OR
\item $2n+1 \equiv 3 \ \ mod\ \  4$ AND 2 generates the quadratic residues in $\mathbb{Z}_{2n+1}$
\end{itemize}
The last rule means: The last 2 bits are set in the binary representation of prime 2n+1; even if $2^k \ \ mod\ \  2n+1$ does not
generate every element in 0 $\sim$ 2n, we can at least take the square root $mod\ \  2n+1$ of $2^k$. \\
To generate a Type-II ONB, first pick an element $\gamma$ of order 2n+1 in $F_{2^{2n}}$, then use this to find optimal normal 
element from $F_{2^n}$: $\beta = \gamma + \gamma^{-1}$. Crossproduct terms
\begin{equation}
\begin{split}
\beta^{2^i}\beta^{2^j} &= (\gamma^{2^i} + \gamma^{-2^i})(\gamma^{2^j} + \gamma^{-2^j})\\ &= 
(\gamma^{2^i+2^j} + \gamma^{-(2^i+2^j)}) + (\gamma^{2^i-2^j} + \gamma^{-(2^i-2^j)})\\ &= 
\begin{cases}
\beta^{2^k} + \beta^{2^{k\prime}}\ \  if\ 2^i \neq 2^j \ \ mod\ \  2n+1\\
\beta^{2^k}\ \  if\ 2^i \equiv 2^j \ \ mod\ \  2n+1
\end{cases}
\end{split}
\end{equation}
$k$ and $k\prime$ are the 2 possible solutions to multiplication of any 2 basis elements. That is why this normal basis
 is optimal: it has the minimum number of possible terms. In the case of $2^i \neq 2^j \ \ mod\ \  2n+1$, at least one of these
equations:
\begin{equation}
\begin{cases}
2^i + 2^j = 2^k \ \ mod\ \  2n+1\\
2^i + 2^j = -2^k \ \ mod \ \ 2n+1
\end{cases}
\end{equation}
will have a solution, and at least one of these equations
\begin{equation}
\begin{cases}
2^i - 2^j = 2^{k\prime} \ \ mod \ \ 2n+1\\
2^i - 2^j = -2^{k\prime} \ \ mod \ \ 2n+1
\end{cases}
\end{equation}
also has a solution.\\
In the case of $2^i \equiv \pm 2^j \ \ mod\ \  2n+1$, at least one of the following 4 equations has a solution:
\begin{equation}
\begin{cases}
2^i + 2^j = 2^k \ \ mod \ \ 2n+1\\
2^i + 2^j = -2^k \ \ mod \ \ 2n+1\\
2^i - 2^j = 2^k \ \ mod\ \  2n+1\\
2^i - 2^j = -2^k \ \ mod \ \ 2n+1
\end{cases}
\end{equation}
In the first set of equations, there are two possible solutions, and in the second set of equations,
there is only one possible solution. It is easy to see that the equations are all similar, so instead of 
working with 2 different sets we can combine them and work with just one group of 4 equations. To build 
our $\lambda$-Matrix, we set $k=0$ and find solutions to:
\begin{equation}
\begin{cases}
2^i + 2^j = 1 \ \mod\ \  2n+1\\
2^i + 2^j = -1 \ \ mod \ \ 2n+1\\
2^i - 2^j = 1 \ \ mod\ \  2n+1\\
2^i - 2^j = -1 \ \ mod \ \ 2n+1
\end{cases}
\end{equation}

\textit{Example E.1:} \ \ Following this criteria, easy to find $M^{(0)}$ for $F_{2^5}$:
\begin{equation}
M^{(0)} = \left(
\begin{array}{lcccr}
0 & 1 & 0 & 0 & 0\\
1 & 0 & 0 & 1 & 0\\
0 & 0 & 0 & 1 & 1\\
0 & 1 & 1 & 0 & 0\\
0 & 0 & 1 & 0 & 1
\end{array} \right)
\end{equation}
