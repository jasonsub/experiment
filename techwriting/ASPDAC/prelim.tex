\section{Preliminaries}
\label{sec:prelim}
%\subsection{Computer Algebra Preliminaries}
\subsection{Modeling of Constrains by Ideal}
A state can be described by a set of constrains. Generally, a constrain is a function mapping 
from field variables to boolean values ($f:X\to bool,\ X=\{x_0,\dots,x_n\}$); meanwhile this "function" can be
written in form of polynomial about field variables. If the constrain is satisfied, i.e. $f:\sigma(X)\to \perp$,
accordingly the evaluation of this polynomial is 0:
$$f(x_0,\dots,x_n) = a_0x_0+\cdots + a_nx_n = 0$$
If a state is valid under some interpretation, then all of its constrains need to be satisfied:
\begin{equation}
\left\{
\begin{array}{ll}
f_0(x_0\dots x_n) = 0\\
\vdots\\
f_k(x_0\dots x_n) = 0
\end{array}\right.
\nonumber
\end{equation}
It is reasonable to consider the solution vector $\langle x_0,\dots,x_n\rangle$ for this system of 
equations. However in most cases, solving the system is impossible and unnecessary; instead people
tend to generate a simpler constrain/polynomial as an over-approximation (also known as \emph{abstraction})
from heuristic methods and then check if it satisfies all constrains of this state, which means
$$\forall j, f_j(\langle x_i\rangle) = 0\implies\F(\langle x_i\rangle) = 0  $$
One heuristic is to take $\F$ as composition of $\langle f_j\rangle$ as follows: if
$$\F = c_0f_0+c_1f_1+\cdots+c_kf_k$$
then above implication can hold. Here coefficients $c_j$ could be any polynomials
or real numbers. We introduce a new concept to represent this expression:
\begin{Definition}
{\bf Ideals:}\ \ An \emph{ideal} $J$ generated by polynomials $f_0,\dots,f_k\in \mathbb{F}_q[x_0,\dots,x_n]$
is defined as $$J = \langle f_0,\dots,f_k\rangle = \{\sum_{i=0}^{k}c_i\cdot f_i\ |\ c_i\in\mathbb{F}_q[x_0,
\dots,x_n]\}$$
$f_0,\dots,f_k$ are called \emph{generators} of ideal $J$.
\end{Definition}
Thus we conclude following theorem:
\begin{Theorem}
If $f_0\sim f_k$ are all satisfied and $\F$ belongs to the \emph{ideal} generated by $\{f_0,\dots,f_k\}$, then
$\F$ is also satisfied.
\end{Theorem}

\subsection{Ideal Membership and Gr\"obner Basis}
As discussed above, in order to check whether $\F$ is a valid abstraction, it is necessary to check if 
$\F\in J=\langle x_0,\dots,x_n\rangle$, which requires a technique called \emph{ideal membership test}.
To execute this technique, $\langle f_0,\dots,f_k\rangle$ need to be transformed to \emph{Gr\"obner Basis}
$\langle g_0,\dots,g_s\rangle$.

Gr\"obner basis can generate the same ideal as from original generators. Meanwhile it has an important property
that the leading term from an arbitrary polynomial from this ideal can be divided by at least one Gr\"obner
basis polynomial's leading term. More specifically,
\begin{Definition}
{\bf Gr\"obner Basis:}\ \ For a monomial ordering $>$, a set of non-zero polynomials $G=\{g_0,\dots,g_s\}$
contained in an ideal $J$, is called a \emph{Gr\"obner basis} for $J$ if and only if:
$$\forall f\in J, f\neq 0, \exists g_i\in G: lm(g_i)\ |\ lm(f)$$
where $lm$ denotes the leading monomial of a polynomial.
\end{Definition}

This property is vital when operating following "multi-division":
\begin{Definition}
{\bf Multi-division:}\ \ A polynomial $f$ divided by an ideal $J$(actually its generators, a finite set of
polynomials) follows rules from multi-division $f\xrightarrow{J}$:\\
for each term $t$ in $f$, search in all generators of $J$ such that the leading monomial of $g_i$ can divide
this term's monomial, then update $f$:
$$f = f - \frac{t}{lt(g_i)}g_i$$
$lt$ means leading term. Every time $\frac{t}{lt(g_i)}$ is recorded as one term of quotient $c_i$ w.r.t. $g_i$.
When it is impossible to update $f$, the value of $f$ is recorded as \emph{remainder} $r$, denoted as
$f\xrightarrow{J}_{+}r$.
\end{Definition}

If $J = \langle f_0,\dots,f_k\rangle$, $\F\xrightarrow{J}_+r$, we can deduce that
$$\F = c_0f_0+c_1f_1+\cdots+c_kf_k + r$$
if the remainder $r=0$, then $\F$ is the desired constrain ($\F\in J$). According to the property of
Gr\"obner basis, if $\F\in J$ then during all steps updated $\F$ also belongs to $J$. Based on this result,
for every
term in original $\F$ we can find corresponding $g_i$ whose leading term can divide it, thus the remainder
will be 0. Reversely if remainder is 0, we can also deduce $\F\in J$.

\begin{Theorem}
{\bf Ideal membership:}\ \ For ideal $J=\langle g_0,\dots,g_s\rangle$ and $\{ g_0,\dots,g_S\}$
is a Gr\"obner basis, polynomial $\F\in J$ if and only if $\F\xrightarrow{J}_+0$.
\end{Theorem}
\subsection{Buchberger's Algorithm}
\label{sec:buch}
Buchberger's algorithm
shown in Algorithm 1 computes a Gr\"obner
basis in a field. Given polynomials $F = \{f_1, \dots, f_s\}$, 
the algorithm computes the Gr\"obner basis $G = \{g_1, \dots, 
g_t\}$.
% such that ideal $I = \langle f_1, \dots, f_s\rangle = \langle g_1,
% \dots, g_t\rangle$. 
In the algorithm,  

\begin{equation}
       Spoly(f,g)=\frac{L}{lt(f)}\cdot f - \frac{L}{lt(g)}\cdot g \nonumber
      \label{eqn:spoly}
\end{equation}
where $L = \text{LCM}(lm(f), lm(g))$, where $lm(f)$ is the leading
monomial of $f$, and $lt(f)$ is the leading term of $f$. 


\begin{algorithm}[hbt]
\label{alg:gb}
\SetAlgoNoLine
 \KwIn{$F = \{f_1, \dots, f_s\}$}
 \KwOut{$G = \{g_1,\dots ,g_t\}$\\} %, a Gr\"{o}bner basis
  $G:= F$\;
  \Repeat{$G = G'$}
  {
  	$G' := G$\;
  	\For{ each pair $\{f, g\}, f \neq g$ in $G'$} 
	{
		$Spoly(f, g) \stackrel{G'}{\textstyle\longrightarrow}_+r$ \;
		\If{$r \neq 0$}
		{
			$G:= G \cup \{r\}$ \;
		}
	}
   }
\caption{Buchberger's Algorithm}
\end{algorithm}

{\it Fermat's little theorem:} For any $ \alpha \in \mathbb
F_{q}, \alpha^q = \alpha$. Therefore, the polynomial $x^q - x$
vanishes ($=0$) over $\Fq$, and is called a vanishing polynomial. We
denote by $J_0 = \langle x_1^q - x_1, \dots, x_d^q - x_d \rangle$ the
ideal of all vanishing polynomials in $\Fq[x_1, \dots, x_d]$. When $q
= 2^k, x^q - x = x^q + x$ as $-1 = +1$ over $\Fkk$.

Gr\"obner bases can be used to {\it eliminate} variables from an
ideal. Given ideal $J = \langle f_1,\dots,f_s\rangle \subset \mathbb
F_{q}[x_1,\dots,x_d]$, the $l^{th}$ elimination ideal $J_l$ is the
ideal of $\Fq[x_{l+1}, \dots, x_d]$ defined by $J_l = J \cap
\Fq[x_{l+1}, \dots, x_d]$. Variable elimination can be achieved 
by computing a Gr\"obner basis of $J$ w.r.t. elimination orders: 
\begin{Theorem}
(Elimination Theorem\cite{ideals:book}) Let $J\subset \mathbb
  F_{2^k}[x_1,\dots,x_d]$ be an ideal and let $G$ be a Gr\"obner basis
  of $J$ with respect to a lexicographic ordering where
  $x_1>x_2>\cdots>x_d$. Then for every $0\leq l\leq d$, the set $G_l =
  G\cap\mathbb F_{2^k}[x_{l+1},\dots,x_d]$ is a Gr\"obner basis of
  the $l$-th elimination ideal $J_l$.
\end{Theorem}

\subsection{Polynomial Abstraction with Elimination \& Gr\"obner Bases}

The authors of \cite{pruss:dac14} showed that for any combinational
logic block, a canonical word-level polynomial representation can be
derived through \Grobner bases computed with elimination
orders. Our approach is based on their result, which we reproduce
here:
\begin{Lemma}
(From \cite{pruss:dac14}) Given a combinational circuit $C$ with $k$-bit
  input $A = (a_0, \dots, a_{k-1})$ and $k$-bit output $R = (r_0, \dots,
  r_{k-1})$. Denote by $x_1, \dots, x_d$ all the bit-level
  variables of   $C$. Let $J = \langle f_1, \dots, f_s \rangle \subset
  \Fkk[x_1, \dots, x_d, R, A]$ denote all the polynomials corresponding to the
  logic gates of the circuit. Let $J_0 = \langle x_1^2 - x_1, \dots,
  x_d^2 - x_d, R^q - R, A^q - A \rangle$ be the vanishing ideal, so
  that $J + J_0 = \langle f_1, \dots, f_s, ~~ x_1^2 - x_1, \dots,
  x_d^2 - x_d, R^q - R, A^q - A \rangle$. Compute \Grobner basis $G =
  GB(J + J_0)$ w.r.t. lex term order with $x_1 > x_2 > \dots > x_d > R
  > A$. Then $G_d = G \cap \Fkk[R, A]$ eliminates the internal
  variables $x_1, \dots, x_d$ of the circuit. $G_d$ also contains the
  word-level polynomial $R = \F(A)$ which canonically represents the
  function of the circuit.  
\end{Lemma}

The authors referred to the elimination (lex) order with
$\{\textit{internal variables} \ x_1 > \dots > x_d\} > \{\textit{word-level
  output} \ R\} > \{\textit{word-level input} \ A\}$ as the {\it abstraction
  term order (ATO)}. We will now show how to repeatedly apply \Grobner
basis computations with ATO to verify sequential GF circuits.

