\section{Some Questions on Gr\"obner Basis Complexity}

Another problem that we need to find an answer to is one of 
computational complexity of our approach of \cite{lv:tcad2013} (and
its derivatives \cite{pruss:dac14}) that we have been using to verify
Galois field circuits. Here are some interesting results that are
known:

First of all, computing a Gr\"obner Basis $G$ of an ideal $I$ is known
to be doubly-exponential in $n$, the number of variables, and
polynomial in $d$, the degree of the ideal. A tight bound was derived
by T. W. Dube in \cite{Dube:gb-complexity}. I read his technical
report to try to understand the results at a high-level. Dube performs
a disjoint decomposition/partitioning on a homogeneous ideal, uses the
concepts of the Hilbert function and Macaulay constants (I don't
really know what they are) and derives the bound using combinatorial
arguments. He gives a long sequence of deductions which I glossed
over, and I just went to his final result. The result, given in the last
paragraph of the technical report, is summarized below: 

\begin{Result}
Let $F = \{f_1, \dots, f_r\}$ be a homogeneous basis for an ideal $I
\subset K[x_1, \dots, x_n]$. Assume without loss of generality that
$f_1$ has the highest degree $deg(f_1) = d$. Then the degree of
polynomials in the Gr\"obner basis $G$ of $I$ is bounded by 
$2(\frac{d^2}{2}+d)^{2^{n-2}}$. This provides an upper bound on the
{\it degree of power products} which are required to generate the
leading term ideal $LT(I)$, and these power products have to appear in
the Gr\"obner basis\footnote{You may recall that $LT(I)$ is the set of
leading terms of all elements of $I$, i.e. $LT(I) = \{cx^{\alpha}:
\exists f \in I ~\text{with} ~ LT(f) = cx^{\alpha}\}$. Then $\langle
LT(I) \rangle = \langle LT(f_1), \dots, LT(f_r) \rangle$ if and only
if the set $\{f_1, \dots, f_r\}$ is a Gr\"obner basis --- this is one
of the many definitions of a Gr\"obner basis.}. 
\end{Result}

Clearly, the expression $2(\frac{d^2}{2}+d)^{2^{n-2}}$ is polynomial
in $d$ and double-exponential in $n$, and represents the {\it
complexity of the Gr\"obner basis problem}. This should not be
confused for the complexity of Buchberger's algorithm. Dube's
argument is combinatorial --- it just states that in the worst case
$G$ may require polynomials of degree $\leq
2(\frac{d^2}{2}+d)^{2^{n-2}}$ in the Gr\"obner basis. Buchberger's
algorithm, however, solves the problem using S-polynomials and their
reduction $Spoly(f_i, f_j) \xrightarrow{I}_+ r$, and keeps on pairing
the remainder $r$ with the current basis. To the best of my knowledge,
{\it the complexity of Buchberger's algorithm is still a mystery.}
This is because no one has been able to ``predict'' how many new
remainders $r$ may be generated by Buchberger's
algorithm. Buchberger's algorithm will definitely terminate --- thanks
to Dickson's Lemma --- but it is hard to predict when it will
terminate. I am unfamiliar with any new developments in the past few
years, but any serious development would have made headlines.... so I
am assuming this is still a correct assessment. 

The above degree bound of $2(\frac{d^2}{2}+d)^{2^{n-2}}$ for the
Gr\"obner basis is in a {\it general setting} of a ring with
coefficients from a field $K[x_1, \dots, x_n]$. It turns out that life
might be simpler when working over zero-dimensional ideals (ideals
with finite varieties). Papers such as \cite{Lakshman:complexity_gb},
and more importantly \cite{Lakshman:complexity_gb_91}, state that the
number of arithmetic operations required to compute a reduced
Gr\"obner basis of a zero-dimensional ideal is {\it polynomial in
  $d^n$} --- this is a single exponential bound. Lakshman
\cite{Lakshman:complexity_gb_91} works over rationals (ring $R =
\Q[x_1, \dots, x_n]$) and says that {\it ``the   number of arithmetic
  operations required to compute reduced GB of $I   \subset R$ is
  polynomially bounded by the dimension of the residue   class ring
  $R/I$ seen as a rational vector space''}.  

We are interested in Galois fields $\Fkk$. In \cite{gao:gf-gb-ms},
Sicun Gao shows that the complexity of computing a $GB(J + \langle
x_i^q - x_i\rangle)$ over $\Fq[x_1, \dots, x_n]$ is $q^{O(n)}$.
Analogous to Lakshman's result, this is also single exponential in
$n$. [See  section 4.1 in his MS thesis   \cite{gao:gf-gb-ms}]. This
is not surprising because $J + \langle x_i^q - x_i \rangle$ is
zero-dimensional.  In our case,  $q = 2^k$, so a complexity of
$2^{k\cdot n}$ is still prohibitive for practical verification. 

To overcome this complexity, Jinpeng came up with a specialized term
ordering derived from the given circuit. In this term ordering, the
variables of the circuit are ordered in a {\it reverse topological
  order} and a lex term order is imposed. As a result, the polynomials
of our circuit themselves constitute a Gr\"obner basis. We formulate
the verification problem as an ideal membership test: is $f 
\xrightarrow{J+J_0}_+0$? Since we already have a Gr\"obner basis of
$J+J_0$ directly from the circuit, our approach obviates the need to
construct a Gr\"obner basis and performs verification only by a
sequence of polynomial divisions. Please refer to Section VI,
Proposition 6.1, and Theorem 6.1, in our paper \cite{lv:tcad2013} for
more details. 

Note that every polynomial in our Gr\"obner basis is of the form 
$f_i = x_i + P_i$ where: 
\bi
\item $x_i$ is the output of a logic gate in the
design; 
\item $P_i = \text{tail}(f_i)$ corresponds to the function
implemented by the gate; and 
\item each variable $x_j$ that appears in $P_i$ is less than
  $x_i$. 
\ei

So, if the Gr\"obner basis is of the form $f_i: x_i + P_i$, then what
is the complexity of the division: $f \xrightarrow{x_i + P_i}_+r$?  
It turns out that in our experiments in \cite{lv:tcad2013}, the
polynomial $f$ (given specification polynomial to be verified against
a circuit) is a ``sparse'' polynomial of low degrees, e.g. $f: Z -
A*B$ for a multiplier circuit has degree = 2. Dividing low
degree polynomials $f \pmod{ x_i + P_i}$ might not be too
complicated. Jinpeng once (wrongfully) claimed that this complexity is
always bounded polynomially in the size of the Gr\"obner basis. This
is not true: If $f = x^n$, then dividing $x_i^n$ by $x_i + P_i$
requires raising $P_i$ to the $n^{th}$ power --- which is not
polynomial in the size of the Gr\"obner basis. So we need to analyze
this complexity further:  

\begin{Problem}
\label{prob:complex}
Let $F = \{f_1, \dots, f_s\}$ be a set of polynomials in the ring $R =
\Fkk[x_1, \dots, x_n]$ where each polynomial $f_i$ is of the form 
$f_i = x_i + P_i$ where $P_i = \text{tail}(f_i)$. Let ideal 
$J = \langle F \rangle$. Let 
$F_0 = \{x_i^{2^k} - x_i: ~i = 1, \dots,n\}$ and let 
$J_0 = \langle F_0 \rangle$ denote the ideal of vanishing
polynomials. Let $f$ be a polynomial of total degree $d$. What is  the
computational complexity of the multivarite division $f \xrightarrow{F
  \cup F_0}_+ r$?
\end{Problem}

{\bf Here is one way to approach this problem:} 
Since multi-variate division $f\xrightarrow{f_1, \dots, f_s}_+ r$ is
modeled as a series of cancellation of terms $r = f -
\frac{LT(f)}{LT(f_i)} f_i$, it might suffice to derive an upper
bound on the number of terms that would be canceled in the division
process. In the worst case, all terms in $f$ can be
canceled. However, while a term is canceled, other terms may also
be added in the intermediate remainders. So we can further refine
Problem \ref{prob:complex} under these restrictions:

\begin{itemize}
\item First of all, we may assume that in each monomial, {\it each
  variable is raised to a degree  $\leq 2^k -1$}; this is due to the
  set of vanishing polynomials $x_i^{2^k}-x_i$ of $\Fkk$. 
\item Based on our approach \cite{lv:tcad2013}, we may assume a LEX
  term   order. On the other hand, if you prefer to use a degree-based
  (DEGLEX) order, feel free to do so, if it simplifies the analysis.
\item The polynomial $f$ is over $n$ variables $x_1, \dots, x_n$, and
  so the maximum total degree is $d = n(2^k - 1)$. So how many terms,
  at the most, can $f$ have? %Each of these will be canceled. 
\item In the set of polynomials $F = \{f_1, \dots, f_s\}$, we may
  assume that the size (number of terms) of each polynomial $f_i$ is
  bounded by  a known integer $t$. In other words, each $f_i$ has no
  more than $t$ terms. [So, at each ``one-step'' division
    $f\xrightarrow{f_i} r$, a term from $f$ is canceled, and no
    more than $t$ terms are added to $r$.]
\end{itemize}

\begin{Problem}
\label{prob:complex-refine}
Refining Problem \ref{prob:complex} further: If we were to use the
above restrictions, {\it what is the upper bound   on the number of
  terms that are canceled in the division $f   \xrightarrow{F \cup
    F_0}_+ r$?} This gives an estimate of the complexity of
solving verification using Jinpeng's and Tim's approaches
\cite{lv:tcad2013} \cite{pruss:dac14}. 
\end{Problem}

Is it very hard to find the bound in Problem
\ref{prob:complex-refine}? We need to solve this problem.
