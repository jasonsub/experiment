\section{Experiment results}
We have implemented our approach using the \textsc{Singular} symbolic
algebra computation system [v. 3-1-6] \cite{DGPS}. With our
tool implementation, we have performed experiments to extract a minimal UNSAT core from a given set of
polynomials. Our experiments run on a desktop with
3.5GHz Intel $\text{Core}^\text{TM}$ i7-4770K Quad-core CPU, 16 GB RAM and
64-bit Ubuntu Linux OS.

Nowadays most SAT benchmarks are huge, which choke our GB engine. In our experiment we use our customized 
benchmark library: "aim-100" is revised from ramdon 3-SAT benchmark "aim-50/100", "subset" is generated for
some random subset sum problem, "cocktail" is revised from a mixed factorization/random 3-SAT benchmark,
and "phole4/5" are generated by added some redundancies to pigeon hole benchmarks. Meanwhile we create "$3\times3$/$5\times5$ SMPO" benchmarks
by translating miter circuits of unrolled sequential normal basis multiplier \cite{SMPO}
and its specification circuit, and "2$\times$2 MasVMon" benchmark is the miter circuit of Mastrovito multiplier and Montgomery 
multiplier. These benchmarks are in moderate size for our GB engine but not too trivial.

\vspace{-0.1in}
\begin{table}[htb]
\centering
\caption{\small Results of running benchmarks using our tool}
\begin{tabular}{|c||c|c|c|c|c|c|}
\hline
\multirow{4}{2cm}{Benchmark} 
& \multirow{4}{0.9cm}{\# Polys} 
& \multirow{4}{0.8cm}{\# MUS} 
& \multirow{4}{1.8cm}{Size of core by our tool}
 & \multirow{4}{1.6cm}{\# calls of GB engine}
 & \multirow{4}{2cm}{\# redundant polynomials by quotient cancellation}
 & \multirow{4}{1.3cm}{Runtime (sec)} \\
  & & & & & & \\
    & & & & & & \\
      & & & & & & \\
\hline
\hline
$5\times 5$ SMPO & 240  & 137  & 137  & 8  & 57 & 3160\\
\hline
aim-100 & 79 & 22 & 22 & 1  & 0 & 43\\
\hline
$3 \times 3$ SMPO & 17 & 2 & 2 & 1 & 0 & 0.07  \\
$2 \times 2$ MasVMont & 27 & 23 & 23 & 2 & 0 & 2.3 \\
subset & 100 & 6 & 6 & 1 & 0 & 2.4 \\
cocktail & 135 & 4 & 4 & 2 & 1 & 5 \\
phole4 & 104 & 10 & 16 & 1 & 0 & 4.4 \\
phole5 & 169 & 19 & 25 & 3 & 2 & 14 \\
\hline
\end{tabular}
\label{tab:result}  
\end{table} 

In table \ref{tab:result}, we list the details of our experiment results. It shows in most cases, our tool
can get to the minimal core when hitting a fixpoint. Meanwhile, with a limited number of calls to our GB
engine, our tool can remove a major part of redundancies which indicates our tool
can extract a relatively small core efficiently.

From the experiment results we can also make several observations. One observation is our tool generates
minimal core when reaching fixpoint, which reflects the effectiveness of our heuristic. Another fact is,
for some test instances, quite a few redundant polynomials are removed before we start calling
the GB engine. This shows the strong potential for eliminating generator polynomials by our
"cancellation due to quotients" heuristic.