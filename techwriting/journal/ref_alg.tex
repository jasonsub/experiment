\section{Refined Algorithm}
Within the naive algorithm described in last section, Gr\"obner basis is computed for each
iteration to attain the word-level next state polynomial representation. In practice, for 
a sequential circuit with complicated structure and large datapath, Gr\"obner basis computation
is intractable. To overcome the high computational complexity brought by GB computing, 
we managed to find a method obviating directly computing GB; more specifically, we turn 
GB computation into one-step multivariate polynomial division. In this way, our proposed 
approach can be used on larger sequential circuit verification.


\subsection{Refined abstraction term order (RATO)}
A lexicographic order constrained by following relation $>_{r}$: "circuit variables ordered reverse topologically" $>$ 
"designated word-level output" $>$ "word-level inputs" is called the \emph{Refined Abstraction Term Order (RATO)}.

In Buchberger's algorithm, the specification polynomial (Spoly) for each pair is calculated. In RATO, most polynomials
have relatively prime leading terms/monomials (which means $Spoly \xrightarrow{J+J_0}_{+} 0$) except one pair:
word-level polynomial corresponding to outputs and its leading bit-level variable's gate description polynomial.
Its remainder $r$ lets following lemma hold:

\begin{Lemma}
\label{lem:1}
$r$ will only contain primary inputs (bit-level) and word-level output; furthermore, there will be one and
only one term containing word-level output whose monomial is word-level output itself rather than higher order form.
\end{Lemma}

\begin{proof}
First proposition is easy to prove by contradiction. Second part, the candidate pair of polynomials only have one term of
single word-level output variable (say it is $R$) and this term is the last term under RATO, which means there is only one term with
$R$ in Spoly. Meanwhile in other polynomials from $J+J_0$ there is no such term containing $R$, so this term will be
kept to remainder $r$, in first degree.
\end{proof}

\begin{Example}
After adding intermediate bit-level signal $a,b,c,d$, the elimination ideal for example circuit (Ex.\ref{ex:SMPO}) 
could be rewritten under RATO:
\begin{align}
&(t_0,t_1)>(a,b,c,d)\nonumber\\&>T>(x,s_0,s_1)\nonumber
\end{align}
Thus the candidate pair is
$(f_w,f_g), f_w = t_0+a\cdot b+a+b, f_g =t_0+t_1\alpha + T$.
Result after reduction is:
\begin{align}
&Spoly(f_w,f_g) \xrightarrow{J+J_0}_{+}\nonumber\\
&T + s_0 s_1 x+s_0 s_1 \alpha+(1+\alpha)s_0 x+(1+\alpha) s_0+s_1 x+s_1+(1+\alpha) x+1\nonumber
\end{align}
The remainder satisfies Lemma \ref{lem:1}.
\end{Example}

\subsection{Bit-level Variable Substitution (BLVS)}
\label{sec:blvs}
The remainder from \emph{Spoly} contains bit-level PS variables, and our objective is to get a polynomial contains only word-level PS variables
. One possible method is to rewrite bit-level variables in term of function on word-level
variables, i.e.
\begin{equation}
\label{eqn:BLVS}
s_i = \mathcal{G}(S)
\end{equation}
then do substitution. One way to obtain Eqn.\ref{eqn:BLVS} is to apply a Gaussian-elimination-fashion approach, which could 
compute corresponding $\mathcal{G}(S)$ efficiently with time complexity $O(k^3)$.

\begin{Example}
{\bf Objective}:\ Abstract polynomial $s_i + \mathcal{G}_i(S)$ from $f_0: s_0+s_1\alpha+S$.

First, compute $f_0^2: s_0+s_1\alpha^2+S^2$. Apparently variable $s_0$ can be
eliminated by operation 
\begin{align}
f_1 =& f_0 + f_0^2: \nonumber\\
&(\alpha^2+\alpha)s_1+S^2+S\nonumber
\end{align}
Now we can solve univariate polynomial equation $f_1 = 0$ and get solution
$$s_1 = S^2 + S$$
Use this solution we can easily solve equation $f_0 = 0$. The result is
$$s_0 = \alpha S^2+(1+\alpha)S$$
\end{Example}

Another formalized method is to solve following system of equations
\begin{align}
\label{eqn:betamat}
\begin{bmatrix}
S \\
S^2 \\
S^{2^2} \\
\vdots \\
S^{2^{k-1}}
\end{bmatrix}
&=
\begin{bmatrix}
\beta & \beta^{2} & \beta^{2^2} & \cdots & \beta^{2^{k-1}}\\
\beta^{2} & \beta^{2^2} & \beta^{2^3} & \cdots & \beta \\
\beta^{2^2} & \beta^{2^3} & \beta^{2^4} & \cdots & \beta^2\\
\vdots & \vdots & \vdots & \ddots & \vdots \\
\beta^{2^{k-1}} & \beta & \beta^2 & \cdots & \beta^{2^{k-2}}
\end{bmatrix}
\begin{bmatrix}
s_0\\
s_1\\
s_2\\
\vdots\\
s_{k-1}
\end{bmatrix}
\end{align}
Let ${\bf s}$ be a vector of $k$ unknowns $s_0,\dots,s_{k-1}$,
then Eqn.\ref{eqn:betamat} can be solved by using Cramer's rule:
\begin{equation}
\label{eqn:Cramer}
s_i = \frac{\Mi}{\M}, \ \ 0\leq i\leq k-1, \M \neq 0
\end{equation}
where ${\bf M_i}$ denotes a coefficient matrix replacing $i$-th column in ${\bf M}$ with
vector ${\bf S} = [S~~S^2~~\cdots~~S^{2^{k-1}}]^T$.

Notice that ${\bf M}$ is constructed by squaring a row and assigning it to next row,
therefore its determinant belongs to a special sort of determinants:
\begin{Definition}
Let $\{\alpha_0,\alpha_1,\dots,\alpha_{k-1}\}$ be a set of $k$ elements of $\Fpk$. Then the determinant
\begin{align}
\det{M(\alpha_0,\dots,\alpha_{k-1})} &= 
\begin{vmatrix}
\alpha_0 & \alpha_1 & \cdots & \alpha_{k-1} \\
\alpha_0^p & \alpha_1^p & \cdots & \alpha_{k-1}^p \\
\vdots & \vdots & \ddots & \vdots \\
\alpha_0^{p^{k-1}} & \alpha_1^{p^{k-1}} & \cdots & \alpha_{k-1}^{p^{k-1}}
\end{vmatrix}
\end{align}
is called the {\bf Moore determinant} of set $\{\alpha_0,\dots,\alpha_{k-1}\}$.
\end{Definition}
Moore determinant can be written as an explicit expression
\begin{equation}
\label{eqn:Moore}
\det{M(\alpha_0,\dots,\alpha_{k-1})} = \alpha_0 \prod_{i=1}^{k-1} \prod_{c_0,\dots,c_{i-1}\in \Fp} (\alpha_{i} - \sum_{j=0}^{i-1}c_j\alpha_j)
\end{equation}
We use an example to help understanding the notations in Eqn.\ref{eqn:Moore}:
\begin{Example}
Let $\{\alpha_0,\alpha_1,\alpha_2\}$ be a set of elements of $\mathbb F_{2^3}$. Then
\begin{align}
\det{M(\alpha_0,\alpha_1,\alpha_2)} &= 
\begin{vmatrix}
\alpha_0 & \alpha_1 & \alpha_2 \\
\alpha_0^2 & \alpha_1^2 & \alpha_2^2 \\
\alpha_0^4 & \alpha_1^4 & \alpha_2^4
\end{vmatrix} \\ \nonumber
=& \alpha_0 \prod_{i=1}^{2} \prod_{c_0,\dots,c_{i-1}\in \Ftwo} (\alpha_{i} - \displaystyle\sum_{j=0}^{i-1}c_j\alpha_j)
\end{align}
First, let $i=1$,  we obtain $c_0\in \Ftwo$. When $c_0=0$, the product term equals to $\alpha_1$; when $c_0=1$ it equals to $(\alpha_1-\alpha_0)$.
Then let $i=2$, we obtain ${c_0,c_1}\in \Ftwo$, they can take value from $\{0,0\},\{0,1\},\{1,0\}$ and $\{1,1\}$.
We add 4 more product terms $\alpha_2,(\alpha_2-\alpha_1),(\alpha_2-\alpha_0),(\alpha_2-\alpha_0-\alpha_1)$ respectively.

Thus, the result is
\begin{equation}
\det{M(\alpha_0,\alpha_1,\alpha_2)} = \alpha_0\alpha_1(\alpha_1-\alpha_0)\alpha_2(\alpha_2-\alpha_1)(\alpha_2-\alpha_0)(\alpha_2-\alpha_0-\alpha_1)
\end{equation}
\end{Example}
We discover through investigation that $\M$ has a special property when the set of elements forms a basis. The proof is given below:
\begin{Lemma}
\label{lemma:Moore}
Let $\{\alpha_0,\alpha_1,\dots,\alpha_{k-1}\}$ be a normal basis of $\Fpk$ over $\Fp$. Then
\begin{equation}
\det{M(\alpha_0,\alpha_1,\dots,\alpha_{k-1})} = 1
\end{equation}
\end{Lemma}
\begin{proof}
According to the definition Eqn.\ref{eqn:Moore}, the Moore determinant consists of all possible linear combinations of 
$\{\alpha_0,\alpha_1,\dots,\alpha_{k-1}\}$ with coefficients over $\Fp$. If $\{\alpha_0,\alpha_1,\dots,\alpha_{k-1}\}$
is a (normal) basis, then all product terms are distinct and represents all elements in the field $\Fpk$. Since the product 
of all elements of a field equals to 1, the Moore determinant $\M=1$.
\end{proof}
Applying Lemma \ref{lemma:Moore} to Eqn.\ref{eqn:Cramer} gives
\begin{equation}
s_i = \Mi,~~0\leq i\leq k-1
\end{equation}
where $\Mi$ can be easily computed using Laplace expansion method, with complexity $O(k!)$.

In this approach it is easy to get word-level variable representation for each bit-level PS variables. 
By substitution, a new polynomial in word-level PS/NS variables could be attained.

After processing with RATO and BLVS techniques, we get a polynomial in form of $f_T = T+\mathcal{F}(S,x)$
denoting the transition function. We add a polynomial about $S$ to define the present states $f_S$,
as well as a polynomial guarantee free-end signals for primary inputs $f_x = x^2+x$, using elimination term order $S>x>T$,
we can generate a GB out of elimination ideal $\langle f_T,f_S,f_x\rangle$, which contains a univariate
polynomial denoting next states. The new scalable algorithm can be depicted as following:

\begin{algorithm}[hbt]
\SetAlgoNoLine
 \KwIn{Input-output circuit characteristic polynomial ideal $J_{ckt}$, initial state polynomial $\F(S)$}

  $from^0 = reached = \F(S)$\;
  $f_T = $Reduce(spoly$(f_w,f_s), J_{ckt})$\;
  \tcc{Compute S-poly for the critical pair, then reduce it with circuit ideal under RATO}
  \Repeat{$\langle new^i\rangle == \langle 1\rangle$}
  {
  	$i \gets i + 1$\;
  	$G \gets$GB($\langle f_T ,f_x, from^{i-1}\rangle$)\;
  	\tcc{Compute Gr\"obner basis with elimination term order: $T$ smallest; $f_x$ is polynomial covers all possible inputs from PIs}
	$\langle to^i \rangle \gets G\cup \Fkk[T]$\;
	\tcc{There will be a univariate polynomial in $G$ denoting next state in word-level variable $T$}
	$\langle new^i\rangle \gets \langle to^i\rangle + (\langle T^{2^k}-T\rangle:\langle reached\rangle)$\;
	\tcc{Use quotient of ideals to attain complement of reached states, then use sum of ideals to attain an intersection with next state}
  	$\langle reached\rangle \gets \langle reached\rangle \cdot \langle new^i\rangle$\;
  	\tcc{Use product of ideals to attain a union of newly reached states and formerly reached states}
	$from^i \gets new^i(S\setminus T)$\;
	\tcc{Start a new iteration by replacing variable $T$ in newly reached states with current state variable $S$}
  }
  \tcc{Loop until a fixpoint reached: newly reached state is empty}
\Return{$\langle reached\rangle$}
\caption {Refined Algebraic Geometry based Traversal Algorithm}\label{alg:refined}
\end{algorithm}

\subsection{PI Partition}
Using above techniques we can get a remainder polynomial with only word-level PS/NS variables. However in most 
cases, the number of bit-level PIs will be too large for the last-step Gr\"obner basis computation. Therefore it is necessary
to convert bit-level PIs to word-level PI variables. 

Assume we have $k$-bit datapath and $n$-bit PIs. In Galois field, we need to carefully partition $n$ PIs
to guarantee states of all partitions can be covered by single univariate polynomials.

\begin{Proposition}
Divide $n$-bit PIs to partitions $n_1,n_2,\dots, n_s$ and let $n_1,n_2,\dots,n_s$ bit word-level variables
represent their evaluations in $\mathbb F_{2^k}$ accordingly. Then there always be $n_1,n_2,\dots,n_s~|~k$.
\end{Proposition}

Again, assume a partition $n_i~|~k$ and corresponding word-level variable is $P$. Then we can use polynomial
$P^{2^n_i}-P$ to represent all signals at free-end PIs, according to following theorem about {\bf composite
field}:
\begin{Theorem}
\label{thm:composite}
Let $k = m\dot n_i$, such that $\mathbb F_{2^k} = \mathbb F_{(2^{n_i})^m}$. Let $\alpha$ be primitive root of 
$\mathbb F_{2^k}$, $\beta$ be primitive root of the ground field $\mathbb F_{2^{n_i}}$. Then
$$\beta = \alpha^\omega,~~\text{where}~~\omega = \frac{2^k-1}{2^{n_i}-1}$$
\end{Theorem}
\begin{Example}
In a sequential circuit, PS/NS inputs/outputs are 4-bit signals, which means we will use $\mathbb F_{2^4}$
as working field. PIs are partitioned to 2-bit vectors, which means the ground field is $\mathbb F_{2^2}$.
In ground field we can represent all possible evaluations of this PI partition $\{p_0,p_1\}$ with
$$P^4+P,~~\text{where}~~P=p_0+p_1\cdot\beta$$
Using Thm.\ref{thm:composite} we get $\beta = \alpha^5$, so we can redefine word $P$ as element from $\mathbb F_{2^4}$:
$$P = p_0 + p_1\cdot\alpha^5$$
\end{Example}
Using this method we can efficiently partition large size PIs to small number of word-level PI variables.
One limitation of this approach is PIs cannot be partitioned when $k$ is prime.