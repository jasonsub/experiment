\begin{abstract}
This PhD dissertation proposal addresses the problem of sequential circuit verification at the word-level and is based on concepts derived from algebraic geometry. Analyzing the sequential circuits at word level is an efficient way
of {\it abstraction}, which may  lower the complexity of verification by efficient representation of the state-space. We propose to model the verification properties and the gate-level sequential circuit implementations over Galois fields of the type ${\mathbb{F}}_{2^k}$ by means of polynomial ideals and their canonical representations --- Gr\"obner bases --- at the level of $k$-bit words. Subsequently, techniques from algebraic geometry can be used to reason about the state-space by analyzing varieties of these ideals. 

We propose to apply these techniques to traverse a finite state machine (FSM) for reachability analysis at the word-level, and also to 
implicitly unroll sequential arithmetic circuits to verify their function. Moreover, as unsatisfiable (UNSAT) cores play an important role in modern abstraction-refinement techniques for verification, we propose to investigate a word-level analogue of the UNSAT core problem using the  Gr\"obner basis algorithm. The proposal will not only derive new algorithms and techniques, but will also consider efficient CAD implementations to target sequential equivalence and model checking problems. 
\end{abstract}
