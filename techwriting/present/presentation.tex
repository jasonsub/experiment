%\documentclass[pdf,final,colorBG,slideColor]{prosper}
\documentclass[xcolor=dvipsnames]{beamer}
%%\usetheme{default}
\usecolortheme[named=Maroon]{structure}
%\usetheme{Boadilla}
\usetheme{Madrid}
\useoutertheme{default}%[footline=empty]{infolines}

% \usepackage{helvet}
% \usepackage{enumerate}
% \usepackage{amsmath}
% \usepackage{amsfonts}
% \usepackage{graphicx}
% \usepackage{ulem}
% \usepackage{multirow}
\usepackage{comment}
\usepackage{xspace}

\usepackage[absolute,overlay]{textpos}
\usepackage[ruled]{./algorithm2e}
%%for algorithm2e package, label has to be following caption in the same line!!!
\renewcommand{\algorithmcfname}{ALGORITHM}
\SetAlFnt{\small}
\SetAlCapFnt{\small}
\SetAlCapNameFnt{\small}
\SetAlCapHSkip{0pt}
\IncMargin{-\parindent}

%\RequirePackage{algorithmic}
%\RequirePackage{algorithm}
% \renewcommand{\algorithmicrequire}{\textbf{Inputs:}}
% \renewcommand{\algorithmicensure}{\textbf{Outputs:}}


%\newtheorem{theorem}{Theorem}
%\newtheorem{lemma}{lemma}
%\newtheorem{corollary}{Corollary}
%\newtheorem{proposition}{Proposition}
%\newtheorem{Q}{Question}
%\newtheorem{Exa}{Example}
%\newtheorem{Definition}{Definition}


\newcommand{\Fq}{{\mathbb{F}}_{q}}
\newcommand{\Fkk}{{\mathbb{F}}_{2^k}}
\newcommand{\Zkk}{{\mathbb{Z}}_{2^k}}
\newcommand{\Fkkx}[1][x]{\ensuremath{\mathbb{F}}_{2^k}[#1]\xspace}
\newcommand{\Grobner}{Gr\"{o}bner\xspace}
%\newcommand{\Grobner}{Gr\"{o}bner}
\newcommand{\bi}{\begin{itemize}}
\newcommand{\ei}{\end{itemize}}
\newcommand{\F}{{\mathcal{F}}}
\newcommand{\B}{{\mathbb{B}}}


\title[Ph.D Proposal]{Sequential Circuit Verification at Word Level using Algebraic Geometry}

\author[Xiaojun Sun]{Xiaojun Sun}

%\email{rostamian@umbc.edu}
\institute[Univ. of Utah]{
%\includegraphics[height=17mm]{/Users/Kalla/teaching/Comp-Algebra-Course/lectures/old_ulogo.eps}\\
Ph.D Candidate\\
Electrical and Computer Engineering, University of Utah\\
xiaojuns@ece.utah.edu\\
\ \\
\ \\
{\bf Ph.D's Dissertation Proposal}\\
}


\date{}
%\slideCaption{}

%% Images
%\pgfdeclareimage[width=.4in]{fg:logo}{/Users/Kalla/teaching/Comp-Algebra-Course/lectures/old_ulogo.eps} 

%%%%%%%%%%%%%%%%%%%%%%%%%%%%%%%%%%%%%%%%%%%%%%%%%%
%%%%%%%%%%%%%%%%%%%%%%%%%%%%%%%%%%%%%%%%%%%%%%%%%%
%%%%%%%%%%%%%%%%%%%%%%%%%%%%%%%%%%%%%%%%%%%%%%%%%%
\begin{document}


%----------- titlepage ----------------------------------------------%
\begin{frame}[plain]
  \titlepage

\end{frame}

%\maketitle
%%%%%%%%%%%%%%%%%%%%%%%%%%%%%%%%%%%%%%%%%%%%%%%%%%


\begin{frame}{\large{Agenda}}

\begin{itemize}
\item Focus
	\begin{itemize}
	\item Implicitly analyze the reachability of a sequential circuit at word level
	\item Use algebraic geometry to assist in sequential circuits abstraction refinement
	\end{itemize}
\item Motivation
	\begin{itemize}
	\item Bit-level to word-level abstraction, BFS traversal of FSMs
	\end{itemize}
\item Target problems 
	\begin{itemize}
	\item Given a sequential FSM, with $k$-bit state variables,
     perform an implicit state enumeration at word level, i.e. use word-level variables from $\Fkk$ to represent reachable states
        \item Extract UNSAT cores efficiently from a given set of CNF clauses
	\end{itemize}
\item Approach: {\bf Algebraic geometry techniques}
	\begin{itemize}
	\item  \Grobner basis methods + Elimination ordering + BFS traversal
        \item  Challenge: Discover efficient algorithmic techniques to implement image computations, 
        set operations, UNSAT proofs, etc. at word level
        \item  Proposed Contribution: Polynomial abstraction as well as algebraic geometry techniques 
        is applied to reachability analysis; A new algorithm based on \Grobner basis computation is explored to extract UNSAT cores
	\end{itemize}
%\item Results \& Conclusions
\end{itemize}
\end{frame}

%%%%%%%%%%%%%%%%%%%%%%%%%%%%%%%%%%%%%%%%%%%%%%%%%%

\begin{frame}{\large {Motivation}}
\vspace{-0.2in}
%\ptsize{10}

\begin{itemize}
\item Importance of reachability analysis in sequential circuits verification
	\begin{itemize}
	\item Circuits $\to$ state machines; errors $\to$ bad states
	\item Bad states are reachable $implies$ errors affect circuit behavior
	\end{itemize}	
\end{itemize}

\begin{itemize}
\item Advantages exploiting word-level verification
	\begin{itemize}
	\item Many circuit datapaths/system models are described at word level
	\item Reduce state space, avoid ``bit-blasting"
	\end{itemize}	
\end{itemize}
% 
% \bi
% \item Bugs in hardware can leak secret keys [{\it Biham et al.}, ``Bug
%   Attacks'', Crypto 2008] 
% \ei

% \begin{itemize}
% \item Type of circuits in ECC
% 	\begin{itemize}
% 	\item Multiplication dominates Galois field computations
% 	\item In cryptography: $99\%$ time in encryption and decryption
% 	\item 
% 	\end{itemize}
% \end{itemize}	

%  \begin{itemize}
%  \item Data-path size in ECC crypto-systems can be very large
%  	\begin{itemize}
%         \item In $\mathbb{F}_{2^k}$, $k = 163, 233, \dots$ (NIST standard)
%         \item ECC-point addition for encryption, decryption, authentication
%  	\item Custom arithmetic architectures -- hard to verify
%         \item Synthesized circuits are ``easier'' to verify
%  	\end{itemize}
%  \end{itemize}

\begin{itemize}
\item Why use algebraic geometry?
	\begin{itemize}
        \item Provide a symbolic representation for both bit and word level variables in one unified framework
	\item Recent work shows it is practical to apply algebraic geometry to circuit verification
	\end{itemize}
\end{itemize}

\end{frame}






\end{document}